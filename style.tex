\documentclass[a4paper,12pt,openany,hyperfootnotes,hidelinks]{scrbook}
\usepackage[utf8]{inputenc}
\usepackage[titletoc,title]{appendix}
\usepackage[linedheaders,parts,pdfspacing,manychapters]{classicthesis}
\usepackage[]{fullpage}
% \usepackage[standardsections]{scrhack} % https://tex.stackexchange.com/questions/511049/conflict-between-titlesec-package-and-scrbook-class-after-most-recent-update-of
\usepackage[section=chapter,numberedsection,acronym,toc, nonumberlist,index]{glossaries}
\usepackage[stylemods=bookindex,abbreviations]{glossaries-extra} % index style
%
\usepackage{amsmath}
\usepackage{amsthm}
\usepackage{amssymb}
\usepackage{style/rest-api}
\usepackage{style/lst}
\usepackage{verbatim}
\usepackage{enumitem}
\setlist{noitemsep}
% \setlist[1]{labelindent=\parindent} % < Usually a good idea
\setlist[itemize]{label={---}}
\usepackage{multirow}
\usepackage{epigraph} % quotations (at start of chapter)
\usepackage{csquotes}
\usepackage{tikz}
\usetikzlibrary{arrows.meta,fit,positioning,shapes}
% \usetikzlibrary{matrix,arrows.meta,fit,positioning,shapes,automata}
% \usepackage{forest}
%
% \usepackage{breakcites} % splits long citations to several lines
% \newcolumntype{L}{>{\raggedright\arraybackslash}X}
%
\setcounter{tocdepth}{2}
%
\let\headheight=\baselineskip
\setlength{\parindent}{0pt}
\setlength{\parskip}{\baselineskip}
\interfootnotelinepenalty=10000 %% Completely prevent breaking of footnotes
%
% %%%%%%%%%%%%%%%%%%%%%%%%%%%%%%%%%%%%%%%%%%%%%
% centered verbatim command
\usepackage{varwidth}
\newenvironment{centerverbatim}{%
  \par
  \centering
  \varwidth{\linewidth}%
  \verbatim
}{%
  \endverbatim
  \endvarwidth
  \par
}
%%%%%%%%%%%%%%%%%%%%%%%%%%%%%%%%%%%%%%%%%%%%%
% make first use \emph
% see: https://tex.stackexchange.com/questions/449104/how-to-style-first-acronym-and-first-glossary-appearance
\newcommand{\firstuseformat}[1]{\emph{#1}}
%
\renewcommand{\glslinkpresetkeys}{% requires v1.26+
  \ifglsused{\glslabel}%
  {\letcs\glstextformat{@firstofone}}%
  {\let\glstextformat\firstuseformat}%
}
%%%%%%%%%%%%%%%%%%%%%%%%%%%%%%%%%%%%%%%%%%%%%
% make only the first mention of a term in a chapter hyperlinked
% see: https://tex.stackexchange.com/questions/404051/glossaries-hyperlink-only-at-the-first-occurrence-in-every-chapter
\GlsXtrEnableLinkCounting[section]{general,acronym,abbreviation}
%
% disable hyperlink if link count is greater than 1:
\renewcommand*{\glslinkpresetkeys}{%
 \ifnum\GlsXtrLinkCounterValue{\glslabel}>1
  \setkeys{glslink}{hyper=false}%
 \fi
}
%%%%%%%%%%%%%%%%%%%%%%%%%%%%%%%%%%%%%%%%%%%%
\bibliographystyle{apalike}
\setabbreviationstyle{short-em-nolong} % abbreviation / acronym should be just short form and emphasis = short-em-nolong

\longnewglossaryentry{fingerpointing}{name=fingerpointing}{Litigation scheme allowing upstream peers to shift blame to downstream peer.}
% \longnewglossaryentry{certified delivery}{name=certified delivery}{Message delivery with signed receipt from recipient.}
% \longnewglossaryentry{object stream}{name=object stream}{Series of messages passed from one node to its peer.}
% \longnewglossaryentry{handover state}{name=handover state}{Root hash of an outgoing data stream signed against downstream peer and time.}
% \longnewglossaryentry{takeover state}{name=takeover state}{Handover state signed against initial state by downstream peer.}
% \longnewglossaryentry{provable data exchange}{name=provable data exchange}{Peer to peer object streams allowing handover and takeover proofs.}
\longnewglossaryentry{Kademlia topology}{name=Kademlia topology}{A scale free network topology that has guaranteed path between any two nodes in $O(log(n))$ hops.}
\longnewglossaryentry{Kademlia connectivity}{name=Kademlia connectivity}{Connectivity pattern of a node $x$ in forwarding kademlia where (1) there is at least one peer in each PO bins $0<=i<d$, and (2) no peer $y$ in the network such that $\mathit{PO}(x,y) >= d$ and $y$ is not connected to $x$. }
% \longnewglossaryentry{request and disappear}{name=request and disappear}{Property of warranted service provision networks, whereby a service task request gains service guarantees in a single instant swap exchange.}
\longnewglossaryentry{mutable resource update}{name=mutable resource update}{}
\longnewglossaryentry{resource update chunk}{name=resource update chunk}{}
\longnewglossaryentry{forwarding Kademlia}{name=forwarding Kademlia}{Recursive flavour of Kademlia routing involving message relay.}
% \longnewglossaryentry{scale-free graphs}{name=}{scale-free graphs}
% \longnewglossaryentry{kademlia}{name=kademlia}{}
\longnewglossaryentry{Kademlia table}{name=Kademlia table}{Indexing of peers based on proximity order of peer address relative to local overlay address.}
\longnewglossaryentry{saturated Kademlia table}{name=saturated Kademlia table}{Nodes with saturated Kademlia table realise Kademlia connectivity.}
\longnewglossaryentry{retrieve request}{name=retrieve request, plural=retrieve requests}{Peer to peer protocol message asking for the delivery of a chunk based on the chunk address.}
\longnewglossaryentry{proximity order}{name=proximity order}{TBD}
% \longnewglossaryentry{proximity bin}{name=proximity bin}{TBD}
\longnewglossaryentry{neighbourhood}{name=neighbourhood}{TBD}
\longnewglossaryentry{retrievability}{name=retrievability}{}
\longnewglossaryentry{routability}{name=routability}{TBD}
% \longnewglossaryentry{practical routability}{name=practical routability}{TBD}
\longnewglossaryentry{redundant retrievability}{name=redundant retrievability}{A chunk is said to be redundantly retrievable with degree $r$ if it is retrievable and would remain so after any $r$ nodes responsible for it leave the network.}
% \longnewglossaryentry{practical retrievability}{name=practical retrievability}{TBD}
\longnewglossaryentry{synchronisation}{name=synchronisation}{TBD}
\longnewglossaryentry{history sync state}{name=history sync state}{TBD}
\longnewglossaryentry{live sync state}{name=live sync state}{TBD}
\longnewglossaryentry{chunk}{name=chunk, plural=chunks}{TBD}
% \longnewglossaryentry{availability lag}{name=availability lag}{TBD}
\longnewglossaryentry{proximity}{name=proximity}{TBD}
% \longnewglossaryentry{most proximate bin}{name=most proximate bin}{TBD}
\longnewglossaryentry{saturation depth}{name=saturation depth}{TBD}
% \longnewglossaryentry{age of saturation}{name=age of saturation}{TBD}
% \longnewglossaryentry{age of health}{name=age of health}{TBD}
\longnewglossaryentry{neighbourhood depth}{name=neighbourhood depth}{TBD}
\longnewglossaryentry{area of responsibility}{name=area of responsibility}{TBD}
\longnewglossaryentry{radius of responsibility}{name=radius of responsibility}{TBD}
\longnewglossaryentry{history syncing}{name=history syncing}{TBD}
\longnewglossaryentry{session syncing}{name=session syncing}{TBD}
\longnewglossaryentry{sync lag}{name=sync lag}{TBD}
\longnewglossaryentry{maturity}{name=maturity}{TBD}
\longnewglossaryentry{guaranteed delivery}{name=guaranteed delivery}{TBD}
\longnewglossaryentry{eventual consistency}{name=eventual consistency}{TBD}
\longnewglossaryentry{erasure code}{name=erasure code}{TBD}
\longnewglossaryentry{storer node}{name=storer node, plural=storer nodes}{TBD}
\longnewglossaryentry{relaying node}{name=relaying node, plural=relaying nodes}{TBD}
\longnewglossaryentry{syncing}{name=syncing}{TBD}
\longnewglossaryentry{chunking}{name=chunking}{TBD}
% \longnewglossaryentry{distributed preimage archive}{name=distributed preimage archive}{TBD}
\longnewglossaryentry{swap}{name=swap}{TBD}
\longnewglossaryentry{plausible deniability}{name=plausible deniability}{TBD}
% \longnewglossaryentry{bee-line delivery}{name=bee-line delivery}{TBD}


% \longnewglossaryentry{zzz}{name=zzz}{TBD}

\longnewglossaryentry{dapps}{name=dapps}{TBD}
\longnewglossaryentry{underlay network}{name=underlay network}{The nodes in the network connect using a peer to peer network protocol as their transport layer on the lowest level.}
\longnewglossaryentry{enode}{name=enode}{TBD}
\longnewglossaryentry{network ID}{name=network ID}{TBD}
\longnewglossaryentry{overlay topology}{name=overlay topology}{The connectivity graph realising a particular topology over the underlay network connections.}
\longnewglossaryentry{kademlia}{name=kademlia}{TBD}
\longnewglossaryentry{proximity order bins}{name=proximity order bins}{TBD}
\longnewglossaryentry{churn}{name=network churn, text=churn}{Attrition of nodes from the network.}
\longnewglossaryentry{hive protocol}{name=hive protocol}{The protocol nodes joining a decentralised network use to discover their peers.}
\longnewglossaryentry{address book}{name=address book}{Kademlia table of known peer addresses.}
\longnewglossaryentry{chunks}{name=chunks}{TBD}
\longnewglossaryentry{immutable}{name=immutable chunk store, text=immutable}{No replace/update operation is available on chunks.}
\longnewglossaryentry{content addressed chunk}{name=content addressed chunk, plural=content addressed chunks}{TBD}
\longnewglossaryentry{single owner chunk}{name=single owner chunk, plural=single owner chunks}{TBD}
\longnewglossaryentry{assoc}{name=assoc}{See \gloss{addressed and signed single-owner content}.}
\longnewglossaryentry{garbage collection}{name=garbage collection}{The process that purges chunks from their local storage.}
\longnewglossaryentry{garbage collection strategy}{name=garbage collection strategy}{The process that dictates which chunks are chosen for garbage collection.}
\longnewglossaryentry{direct delivery}{name=direct delivery}{Chunk delivery happens in one step via some lower level network protocol.}
\longnewglossaryentry{pass-back delivery}{name=pass-back delivery}{TBD}
\longnewglossaryentry{routed delivery}{name=routed delivery}{TBD}
\longnewglossaryentry{opportunistic caching}{name=opportunistic caching}{When a forwarding node receives a chunk, then the chunk is saved in case it will be requested again.}
\longnewglossaryentry{light node}{name=light node}{Concept of light node refer to a special mode of operation necessitated by poor bandwidth environments, e.g., mobile devices on low throughput networks or devices allowing only transient or low-volume storage.}
\longnewglossaryentry{chequebook}{name=chequebook}{TBD}
\longnewglossaryentry{global balance}{name=global balance}{TBD}
\longnewglossaryentry{newcomer}{name=newcomer}{TBD}
\longnewglossaryentry{insider}{name=insider}{TBD}
\longnewglossaryentry{postage stamp}{name=postage stamp}{Proof of payment for pre-paid delivery and storage.}
\longnewglossaryentry{batch}{name=batch}{ A
group of chunks referenced under an intermediate node.}
\longnewglossaryentry{chunk span}{name=chunk span}{TBD}
\longnewglossaryentry{Swarm manifest}{name=Swarm manifest}{A structure that defines a mapping between arbitrary paths and files to handle collections.}
\longnewglossaryentry{manifest entry}{name=manifest entry}{Contains a reference to the Swarm root chunk of the representation of a file and also specifies the media mime type of file.}
\longnewglossaryentry{root access}{name=root access}{TBD}
\longnewglossaryentry{granted access}{name=granted access}{TBD}
\longnewglossaryentry{encrypted reference}{name=encrypted reference}{TBD}
\longnewglossaryentry{access key}{name=access key}{TBD}
\longnewglossaryentry{session key}{name=session key}{TBD}
\longnewglossaryentry{lookup key}{name=lookup key}{TBD}
\longnewglossaryentry{access key decryption key}{name=access key decryption key}{TBD}
\longnewglossaryentry{trojan chunk}{name=trojan chunk}{A chunk containing a disguised message while appearing no different to other chunks. }
\longnewglossaryentry{update notification request}{name=update notification request}{TBD}
\longnewglossaryentry{RLP}{name=RLP}{TBD}

\longnewglossaryentry{Distributed Immutable Store for Chunks}{name=Distributed Immutable Store for Chunks}{TBD}
\longnewglossaryentry{DISC}{name=DISC}{See \gloss{Distributed Immutable Store for Chunks}.}
\longnewglossaryentry{distributed hash table}{name=distributed hash table,plural=distributed hash tables}{TBD}
\longnewglossaryentry{DHT}{name=DHT}{TBD See \gloss{distributed hash table}}
\longnewglossaryentry{BMT chunk}{name=BMT chunk}{TBD}
\longnewglossaryentry{binary Merkle tree hash}{name=binary Merkle tree hash}{TBD}

\longnewglossaryentry{BMT hash}{name=BMT hash}{TBD See \gloss{binary Merkle tree hash}}

\longnewglossaryentry{swarm base account}{name=swarm base account}{TBD}
\longnewglossaryentry{bzz account}{name=bzz account}{TBD}
\longnewglossaryentry{enode url scheme}{name=enode url scheme}{TBD}

\longnewglossaryentry{mining chunks}{name=mining chunks}{TBD}
\longnewglossaryentry{addressed and signed single-owner content}{name=addressed and signed single-owner content}{A structure of chunk content composed of identifier, payload and signature attesting to the association of identifier and payload.}
\longnewglossaryentry{chunk value}{name=chunk value}{TBD}
\longnewglossaryentry{postage lottery}{name=postage lottery}{TBD}
\longnewglossaryentry{postage batch references}{name=postage batch references}{TBD}
\longnewglossaryentry{security deposit}{name=security deposit}{TBD}
\longnewglossaryentry{litigation}{name=litigation}{TBD}
\longnewglossaryentry{challenge}{name=challenge}{TBD}

\longnewglossaryentry{access control trie}{name=access control trie}{TBD}
\longnewglossaryentry{pinning}{name=pinning}{TBD}
\longnewglossaryentry{SWINDLE}{name=SWINDLE}{TBD}
\longnewglossaryentry{Secured With INsurance Deposit Litigation and Escrow}{name=Secured With INsurance Deposit Litigation and Escrow}{TBD See \gloss{SWINDLE}.}
\longnewglossaryentry{update notification}{name=update notification}{TBD}
\longnewglossaryentry{upload tag}{name=upload tag}{TBD}
\longnewglossaryentry{singleton manifest}{name=singleton manifest}{TBD}
\longnewglossaryentry{range queries}{name=range queries}{TBD}
%\longnewglossaryentry{zzz}{name=zzz}{TBD}
\longnewglossaryentry{BZZ network ID}{name=BZZ network ID}{TBD}
\longnewglossaryentry{Kademlia}{name=Kademlia}{TBD}
\longnewglossaryentry{reference count}{name=reference count}{TBD}
\longnewglossaryentry{recovery identifier}{name=recovery identifier}{TBD}
%\longnewglossaryentry{zzz}{name=zzz}{TBD}
%\longnewglossaryentry{zzz}{name=zzz}{TBD}
%\longnewglossaryentry{zzz}{name=zzz}{TBD}
%\longnewglossaryentry{zzz}{name=zzz}{TBD}
%\longnewglossaryentry{zzz}{name=zzz}{TBD}
%\longnewglossaryentry{zzz}{name=zzz}{TBD}
%\longnewglossaryentry{zzz}{name=zzz}{TBD}




% not yet confirmed
\longnewglossaryentry{requestor node}{name=requestor node}{TBD}
\longnewglossaryentry{anonymous retrieval}{name=anonymous retrieval}{Not disclosing the identity of requestor node while retreiving a chunk.}
\longnewglossaryentry{push syncing}{name=push syncing}{Network protocol responsible for delivering a chunk to its storer after it is uploaded to an arbitrary node.}
\longnewglossaryentry{statement of custody receipt}{name=statement of custody receipt}{A receipt from the storer node to the uploader after successful \gloss{push syncing} of a chunk.}
\longnewglossaryentry{anonymous uploads}{name=anonymous uploads}{Uploading leveraging the forwarding Kademlia and keeping the identity of the uploader hidden.}
\longnewglossaryentry{pull syncing}{name=pull syncing}{TBD}
\longnewglossaryentry{downstream peer}{name=downstream peer}{TBD}
\longnewglossaryentry{upstream peer}{name=upstream peer}{TBD}
\longnewglossaryentry{forwarding node}{name=forwarding node, plural=forwarding nodes}{TBD}
\longnewglossaryentry{net user}{name=net user, plural=net users}{A node that is using more resources of the Swarm network than it is providing.}
\longnewglossaryentry{net provider}{name=net provider, plural=net providers}{A node that is providing more resources to the Swarm network than it is using.}
\longnewglossaryentry{repeated dealings}{name=repeated dealings}{TBD}
\longnewglossaryentry{honey token}{name=honey token}{TBD}
\longnewglossaryentry{spurious hop}{name=spurious hop}{Relaying traffic to a node without increasing proximity to the target address.}
%\longnewglossaryentry{zzz}{name=zzz}{TBD}
%\longnewglossaryentry{zzz}{name=zzz}{TBD}
%\longnewglossaryentry{zzz}{name=zzz}{TBD}
%\longnewglossaryentry{zzz}{name=zzz}{TBD}
%\longnewglossaryentry{zzz}{name=zzz}{TBD}


\newacronym{TTL}{TTL}{time to live}

% \begin{figure}[htbp]
%   \centering
%   \caption{}
%   \label{fig:}
% \end{figure}

% definitions of acronyms
% (some) are linking to main terms following the example described here:
% https://tex.stackexchange.com/questions/8946/how-to-combine-acronym-and-glossary
% this is so they are listed in index under main terms

\newglossaryentry{WWW}{
    type=\acronymtype,                   % put it into acronyms glossary (not main)
    name={WWW\glsadd{World Wide Web}},   % the short / non-first label
    description={\gls{World Wide Web}},  % what appears as description in glossary
    first={WWW\glsadd{World Wide Web}}   % what is shown on first appearance
    }
% NOTE: above is example of acronym also described in main glossary, hence all the links

\newglossaryentry{TLD}{
    type=\acronymtype,
    name={TLD},
    description={top level domain},
    first={top level domain (TLD)}
    }
% NOTE: above is example of acronym not described in main glossary, just in acronyms list
    
\newglossaryentry{TTL}{
    type=\acronymtype,
    name={TTL\glsadd{time to live}},
    description={\gls{time to live}},
    first={TTL\glsadd{time to live}}
    }

\newglossaryentry{AC}{
    type=\acronymtype,
    name={AC\glsadd{access control}},
    description={\gls{access control}},
    first={access control (AC)\glsadd{access control}}
    }
    
\newglossaryentry{ENS}{
    type=\acronymtype,
    name={ENS\glsadd{Ethereum Name Service}},
    % short={ENS\glsadd{Ethereum name service}},
    description={\gls{Ethereum Name Service}},
    first={ENS\glsadd{Ethereum Name Service}}
    }
    

\newglossaryentry{URL}{
    type=\acronymtype,
    name={URL},
    description={Uniform Resource Locator},
    first={Uniform Resource Locator (URL)}
    }

\newglossaryentry{BMT}{
    type=\acronymtype,
    name={BMT\glsadd{binary Merkle tree}},
    description={\gls{binary Merkle tree}},
    first={BMT\glsadd{binary Merkle tree}}
    }

\newglossaryentry{DISC}{
    type=\acronymtype,
    name={DISC\glsadd{distributed immutable store for chunks}},
    description={\gls{distributed immutable store for chunks}},
    first={DISC\glsadd{distributed immutable store for chunks}}
    }

\newglossaryentry{DHT}{
    type=\acronymtype,
    name={DHT\glsadd{distributed hash table}},
    description={\gls{distributed hash table}},
    first={DHT\glsadd{distributed hash table}},
    plural={DHTs\glsadd{distributed hash table}}
    }

\newglossaryentry{SWINDLE}{
    type=\acronymtype,
    name={SWINDLE},
    description={Secured With INsurance Deposit Litigation and Escrow, see \gls{swindle}},
    first={SWINDLE\glsadd{swindle} (Secured With INsurance Deposit Litigation and Escrow)},
    plural={SWINDLE\glsadd{swindle}}
    }

\newglossaryentry{SWEAR}{
    type=\acronymtype,
    name={SWEAR},
    description={Secure Ways of Ensuring ARchival or Swarm Enforcement And Registration, see \gls{swear}},
    first={SWEAR\glsadd{swear} (Secure Ways of Ensuring ARchival or Swarm Enforcement And Registration)},
    plural={SWEAR\glsadd{swear}}
    }

\newglossaryentry{SWAP}{
    type=\acronymtype,
    name={SWAP},
    description={Swarm Accounting Protocol, see \gls{swap}},
    first={SWAP\glsadd{swap}},
    plural={SWAP\glsadd{swap}}
    }

\newglossaryentry{ISP}{
    type=\acronymtype,
    name={ISP},
    description={internet service provider },
    first={internet service provider (ISP)}
    }

\newglossaryentry{P2P}{
    type=\acronymtype,
    name={P2P\glsadd{peer-to-peer}},
    description={\gls{peer-to-peer}},
    first={P2P\glsadd{peer-to-peer}}
    }

\newglossaryentry{HTTP}{
    type=\acronymtype,
    name={HTTP},
    description={Hypertext Transfer Protocol},
    first={Hypertext Transfer Protocol (HTTP)}
    }

\newglossaryentry{dapp}{
    type=\acronymtype,
    name={dapp},
    description={\gls{distributed web application}},
    first={dapp\glsadd{distributed web application}},
    plural={dapps\glsadd{distributed web application}}
    }

\newglossaryentry{IPFS}{
    type=\acronymtype,
    name={IPFS\glsadd{InterPlanetary File System}},
    description={\gls{InterPlanetary File System}},
    first={IPFS\glsadd{InterPlanetary File System}}
    }

\newglossaryentry{PO}{
    type=\acronymtype,
    name={PO\glsadd{proximity order}},
    description={\gls{proximity order}},
    first={PO\glsadd{proximity order}}
    }

\newglossaryentry{EVM}{
    type=\acronymtype,
    name={EVM},
    description={\gls{Ethereum Virtual Machine}},
    first={\glsadd{Ethereum Virtual Machine} (EVM)}
    }

\newglossaryentry{BMT chunk}{
    type=\acronymtype,
    name={BMT chunk\glsadd{binary Merkle tree chunk}},
    description={\gls{binary Merkle tree chunk}},
    first={BMT chunk\glsadd{binary Merkle tree chunk}}
    }

\newglossaryentry{BMT hash}{
    type=\acronymtype,
    name={BMT hash\glsadd{binary Merkle tree hash}},
    description={\gls{binary Merkle tree hash}},
    first={BMT hash\glsadd{binary Merkle tree hash}}
    }

\newglossaryentry{ACT}{
    type=\acronymtype,
    name={ACT\glsadd{access control trie}},
    description={\gls{access control trie}},
    first={ACT\glsadd{access control trie}}
    }


\newglossaryentry{PSS}{
    type=\acronymtype,
    name={PSS\glsadd{postal service on Swarm}},
    description={\gls{postal service on Swarm}},
    first={PSS\glsadd{postal service on Swarm}}
    }
    
\newglossaryentry{prod}{
    type=\acronymtype,
    name={prod\glsadd{prompt recovery of data}},
    description={\gls{prompt recovery of data}},
    first={prod\glsadd{prompt recovery of data}}
    }
 
\newglossaryentry{X3DH}{
    type=\acronymtype,
    name={X3DH\glsadd{extended triple Diffie--Hellmann key exchange}},
    description={\gls{extended triple Diffie--Hellmann key exchange}},
    first={X3DH\glsadd{extended triple Diffie--Hellmann key exchange}}
    }

\newglossaryentry{RS}{
    type=\acronymtype,
    name={RS\glsadd{Reed-Solomon coding}},
    description={\gls{Reed-Solomon coding}},
    first={RS\glsadd{Reed-Solomon coding}}
    }

\newglossaryentry{race}{
    type=\acronymtype,
    name={race\glsadd{raffle--apply--claim--earn}},
    description={\gls{raffle--apply--claim--earn}},
    first={race\glsadd{raffle--apply--claim--earn}}
    }
    
    
\newglossaryentry{RAID}{
    type=\acronymtype,
    name={RAID},
    description={Redundant Array of Inexpensive Disks},
    first={RAID}
    }

\newglossaryentry{MAC}{
    type=\acronymtype,
    name={MAC},
    description={message authentication code},
    first={MAC}
    }

\newglossaryentry{ECDH}{
    type=\acronymtype,
    name={ECDH\glsadd{elliptic curve Diffie-Hellman}},
    description={\gls{elliptic curve Diffie-Hellman}},
    first={ECDH\glsadd{elliptic curve Diffie-Hellman}}
    }

\newglossaryentry{BMT proof}{
    type=\acronymtype,
    name={BMT proof},
    description={binary Merkle tree proof},
    first={BMT proof},
    plural={BMT proofs}
    }

\newglossaryentry{API}{
    type=\acronymtype,
    name={API},
    description={application programming interface},
    first={API},
    plural={APIs}
    }

\newglossaryentry{CAC}{
    type=\acronymtype,
    name={CAC},
    description={content addressed chunk},
    first={CAC},
    plural={CACs}
    }

\newglossaryentry{PAC}{
    type=\acronymtype,
    name={PAC},
    description={packed addresses chunk},
    first={PAC},
    plural={PACs}
    }

\newglossaryentry{SOC}{
    type=\acronymtype,
    name={SOC},
    description={single owner chunk},
    first={SOC},
    plural={SOCs}
    }


\newglossaryentry{dream}{
    type=\acronymtype,
    name={dream},
    description={deniable, revocable, expirable, addressable, malleable},
    first={dream},
    plural={dreams}
    }
%
\newcommand\gloss[1]{\gls{#1}} %singular
\newcommand\glossupper[1]{\Gls{#1}} %singular uppercase
\newcommand\glossplural[1]{\glspl{#1}} % plural
\newcommand\glossupperplural[1]{\Glspl{#1}} % plural uppercase
%%%%%%%%%%%%%%%%%%%%%%%%%%%%%%%%%%%%%%%%%%%%%%%%%%%%%%
%%% make condition to render text based on it
%%% following example here: https://tex.stackexchange.com/questions/16711/minimal-option-for-cv/21642#21642
\newif\ifdraft
\drafttrue % Uncomment to compile whole draft document / comment to compile only finished sections.
\ifdraft
\else 
\fi
\newif\iftodo
% \todotrue % Uncomment to compile whole draft document / comment to compile only finished sections.
\iftodo
\else 
\fi
%%%
%
% make page number a link to jump to toc only in draft mode
\ifdraft
\renewcommand\pagemark{{%
  \usekomafont{pagenumber} \hyperref[sec:toc]{\thepage}
}}
\fi
% Statuses only shown in draft mode
\newcommand\wip[1]{
\iftodo \todo[inline,size=\Large]{work in progress: #1}\fi }
\newcommand\orange[1]{
\iftodo \todo[inline,size=\Large]{work in progress: #1}\fi}
\newcommand\green[1]{
\iftodo \todo[inline,size=\Large,backgroundcolor=green]{complete }\fi}
\newcommand\yellow[1]{
\iftodo \todo[inline,size=\Large,backgroundcolor=yellow]{incomplete:  #1}\fi}
\newcommand\red[1]{
\iftodo \todo[inline,size=\Large,backgroundcolor=red]{incomplete}\fi}
% \newcommand\red[1]{
% \todo[inline,size=\Large,backgroundcolor=red]{incomplete}}
%
% statuses in section names
% for error explanation see: https://tex.stackexchange.com/questions/64298/error-with-xcolor-package
\usepackage{xcolor}
\colorlet{GREEN}{green}
\colorlet{ORANGE}{orange}
\colorlet{RED}{red}
% none of these work
% \ifdraft
% \usepackage{bbding}
% \usepackage{pifont}
% \newcommand{\cmark}{\ding{51}}%
% \newcommand{\xmark}{\ding{55}}%
% \usepackage{unicode-math}
% \newcommand{\cmark}{\setmainfont{Linux Libertine O}\symbol{"2714}\par}
% \newcommand{\xmark}{\setmainfont{Linux Libertine O}\symbol{"2718}\par}
% \usepackage{fontawesome}
% \def\cmark{\FA\symbol{"F00C}}
% \def\xmark{\FA\symbol{"F00D}}
% \fi
% \newcommand{\cmark}{${\checkmark}$}
% \newcommand{\xmark}{${\times}$}%'
\newcommand{\cmark}{***}
\newcommand{\xmark}{xxx}
\newcommand\statusgreen{\ifdraft \textcolor{green}{\cmark}\fi} 
\newcommand\statusyellow{\ifdraft \textcolor{green}{?}\fi} 
\newcommand\statusorange{\ifdraft \textcolor{orange}{WIP}\fi} 
\newcommand\statusred{\ifdraft \textcolor{red}{\xmark}\fi} 
% \newcommand\statusred{ \textcolor{red}{\xmark}} 
\newcommand\statuspriority{\ifdraft \textcolor{blue}{PRIORITY}\fi}
%
%%%%%%%%%%%%%%%%%%%%%%%%%%%%%%%%%%%%%%%%%%%%%%%%%%%%%%%%%%%%%
\newcommand\PO{\mathit{PO}}
\newcommand\Keys{\mathcal{K}}
\newcommand\nil{\varnothing}
% \newcommand\swip[1]{\url[SWIP-#1]{https://github.com/ethersphere/SWIPs/blob/master/SWIPs/swip-#1.md}}
\newcommand\defeq{\overset{\textrm{\tiny  def}}{=}}
\providecommand{\xor}{\veebar}
\newcommand{\idx}[1]{\texttt{[}\/#1\/\texttt{]}}
\DeclareMathOperator{\concat}{\operatorname{\oplus}}
\DeclareMathOperator{\Concat}{\operatornamewithlimits{\bigoplus}}
\DeclareMathOperator{\rangedel}{:}
%%%%%%%%%%%%%%%%%%%%%%%%%%%%%%%%%%%%%%%%%%%%%%%%%%%%%%%%%%%%%%
% new math environments
\usepackage{thmtools}
\makeatletter
\def\ll@definition{%
  \protect\numberline{\csname the\thmt@envname\endcsname}%
  \ifx\@empty\thmt@shortoptarg
    \thmt@thmname
  \else
    \thmt@shortoptarg
  \fi}
\def\l@thmt@definition{} 
\makeatother
\declaretheoremstyle[
spaceabove=6pt, spacebelow=6pt,headfont=\normalfont\bfseries,notefont=\mdseries, notebraces={}{},bodyfont=\normalfont,postheadspace=1em]{theorem}
\declaretheorem[style=theorem]{definition}
\declaretheorem[style=theorem]{theorem}
%\theoremstyle{definition}
%\newtheorem{definition}{Definition}%[section]
%%
% \newtheorem{theorem}{Theorem}[section]
% \newtheorem{lemma}{Lemma}[section]
% \newtheorem{corollary}{Corollary}[section]
\renewcommand{\listtheoremname}{List of definitions}
%
\makeatletter % legacy /sc command support
%\DeclareOldFontCommand{\rm}{\normalfont\rmfamily}{\mathrm}
%\DeclareOldFontCommand{\sf}{\normalfont\sffamily}{\mathsf}
%\DeclareOldFontCommand{\tt}{\normalfont\ttfamily}{\mathtt}
%\DeclareOldFontCommand{\bf}{\normalfont\bfseries}{\mathbf}
%\DeclareOldFontCommand{\it}{\normalfont\itshape}{\mathit}
%\DeclareOldFontCommand{\sl}{\normalfont\slshape}{\@nomath\sl}
\DeclareOldFontCommand{\sc}{\normalfont\scshape}{\@nomath\sc}
\makeatother