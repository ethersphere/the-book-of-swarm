
The Swarm specification is presented in four chapters. Chapter \ref{spec:convention} enlists all the conventions relating data types, formats and algorithms using the \lstinline{buzz} language as pseudo-code.
Chapter \ref{spec:protocol} ( Protocols) specifies the wire protocols: message formats, serialisation and encapsulation are hard requirements that are crucial for  cross client compatibility. This chapter benefits from \lstinline{protobuf},%
%
\footnote{\url{https://developers.google.com/protocol-buffers}}
%
which is a generic language-neutral, platform-neutral, extensible mechanism for serializing structured data.
The chapter on Strategies (\ref{spec:strategy}) is meant to present recommended incentive-aligned behaviour.
APIs (\ref{spec:api}) gives a formal specification of the high level interfaces of Swarm. Different client implementations are  to be tested against a standard suite of API tests. The API specifications benefit from \lstinline{OpenAPI},%
%
\footnote{\url{https://swagger.io/}}
%
which is an API description format for REST APIs.


\listoftheorems[ignoreall,show={definition}]

\chapter{Data types and algorithms}\label{spec:convention}

\orange{}

\section{Built-in primitives \statusyellow}\label{spec:format:builtin}
\subsection{Crypto \statusgreen}\label{spec:format:crypto}

This section describes the crypto primitives used throughout the specification. They  are exposed as  buzz built-in  functions.
The modules are hashing, random number generation, key derivation, symmetric and asymmetric encryption (ECIES), mining (i.e., finding a nonce), elliptic curve key generation, digital signature (ECDSA), Diffie--Hellman shared secret (ECDH) and Cauchy-Reed-Solomon  (CRS) erasure coding.

Some of the built-in crypto primitives (notably, sha3 hash, and ECDSA ecrecover) are replicating crypto functionality of the Ethereum VM. These are defined here with the help of ethereum api calls to a smart contract. This smart contract just implements the primitives of "buzz" and only has read methods.

\subsubsection{Hashing}

The base hash function implements Keccak256 hash as used in Ethereum.

\begin{definition}[Hashing]\label{def:hash}
\begin{lstlisting}[language=buzz1]
// /crypto

define function hash @input []byte
    ?and/with @suff
    return segment
as
    ethereum/call "sha3" with @input append= @suff
         on context contracts "buzz" 
\end{lstlisting}
\end{definition}  


\subsubsection{Random number generation}

\begin{definition}[Random number generation]\label{def:rng}
\begin{lstlisting}[language=buzz1]
// /crypto

define function random type 
    return [@type size]byte 

\end{lstlisting}
\end{definition}    


\subsubsection{Scrypt key derivation}

The crypto key derivation function implements \lstinline{scrypt} \cite{percival2009stronger}.

\begin{definition}[Scrypt key derivation]\label{def:scrypt}
\begin{lstlisting}[language=buzz1]
// /crypto

define type salt as [segment size]byte

define type key as [segment size]byte

// params for scrypt key derivation function
// scrypt.key(password, salt, n, r, p, 32) to generate key

define type kdf
    n int // 262144
    r int // 8
    p int // 1

define function scrypt from @password
    with   salt 
    using  kdf
    return key

\end{lstlisting}
\end{definition}  

\subsubsection{Mining helper}

This module provides a very simple helper function that finds a nonce that when given as the single argument to a mining function returns true.

\begin{definition}[Mining a nonce]\label{def:mine}
\begin{lstlisting}[language=buzz1]
// /crypto

define type nonce as [segment size]byte

define function mine @f function of nonce return bool
as
    @nonce = random key
    return @nonce if call @f @nonce
    self @f
    
\end{lstlisting}
\end{definition}  

\subsubsection{Symmetric encryption}

Symmetric encryption uses a modified blockcipher with 32 byte blocksize in counter mode.
The segment keys are generated by hashing the chunk-specific encryption key with the counter and hash that again. This second step is required so that a segment can be selectively disclosed in a 3rd party provable way yet without compromising the security of the rest of the chunk.

The module provides input length preserving blockcipher encryption.

\begin{definition}[Blockcipher]\label{def:crypt}
\begin{lstlisting}[language=buzz1]
// /crypto

// two-way (en/de)crypt function for segment
define function crypt.segment segment
    with key
    at @i uint8
as
    hash @key and @i        // counter mode 
        hash                // extra hashing
        to @segment length  // chop if needed
        xor @segment        // xor key with segment

// two-way (en/de)crypt function for arbitrary length 
define function crypt @input []byte
    with key
    return [@input length]byte
as
    @segments = @input each segment size    // iterate segments of input
        go crypt.segment at @i++ with @key  // concurrent crypt on segments
    return wait for @segments               // wait for results
        join                                // join (en/de)crypted segments
        
\end{lstlisting}
\end{definition}    

\subsubsection{Elliptic curve keys}

Public key cryptography is the same as in Ethereum, it uses the secp256k1 elliptic curve.  


\begin{definition}[Elliptic curve key generation]\label{def:ec-keys}
\begin{lstlisting}[language=buzz1]
// /crypto
define type pubkey  as [64]byte
define type keypair
    privkey [32]byte
    pubkey
    
define type address as [20]byte

define function address pubkey
    return address
as 
    hash pubkey 
        from 12

define function generate 
p    ?using entropy
as
    @entropy = random segment if no @entropy
    http/get "signer/generate?entropy=" append @entropy 
        as keypair

\end{lstlisting}
\end{definition}    

\subsubsection{Asymmetric encryption}

Asymmetric encryption implements ECIES based on  the secp256k1 elliptic curve. 
%  TODO: this needs more detail

\begin{definition}[Asymmetric encryption]\label{def:asymmetric-encryption}
\begin{lstlisting}[language=buzz1]
// /crypto

define function encrypt @input []byte 
    for pubkey
    return [@input length]byte

define function decrypt @input []byte 
    with keypair
    return [@input length]byte

\end{lstlisting}
\end{definition}  


\subsubsection{Signature}

Crypto's  built-in signature module implements secp2156k1 elliptic curve based ECDSA. The actual signing happens in the external signer running as a separate process (possibly within the secure enclave). As  customary  in Ethereum, the  signature is represented and serialised using the r/s/v format,

\begin{definition}[Signature]\label{def:signature}
\begin{lstlisting}[language=buzz1]
// /crypto

define type signature
    r segment
    s segment
    v uint8
    signer private keypair
    

define type doc 
    preamble []byte
    context  []byte
    asset    segment
    
define function sign @input []byte 
    by keypair
    return signature
as
    @doc = doc{ "swarm signature", context caller, @input }
    @sig = http/get "signer/sign?text=" append @doc 
        append "&account=" append @keypair pubkey address
            as signature 
    @sig signer = @keypair
    @sig
    
define function recover signature
    with @input []byte
    from @caller []byte
    return pubkey
as
    @doc =  doc{ "swarm signature", @caller, @input } as bytes
    ethereum/call "ecrecover" with 
        on context contracts "buzz" 
            as pubkey

\end{lstlisting}
\end{definition}  


\subsubsection{Diffie-Hellmann shared secret}

The shared secret module implements elliptic curve based Diffie--Helmann shared secret (ECDH) using the usual secp256k1 elliptic curve.
The actual DH comes from the external signer which is then hashed together with a salt.

\begin{definition}[Shared secret]\label{def:dh}
\begin{lstlisting}[language=buzz1]
// /crypto

define function shared.secret between keypair
    and pubkey
    using salt
    return [segment size]byte
as
    http/get "signer/dh?pubkey=" append @pubkey append "&account=" @keypair address
        hash with @salt

\end{lstlisting}
\end{definition}  

\subsubsection{Erasure coding}\label{spec:format:erasure}

Erasure coding interface provides wrappers \lstinline{extend/repair} for the encoder/decoder that work directly on a list of chunks.%
%
\footnote{Cauchy-Reed-Solomon erasure codes based on \url{https://github.com/klauspost/reedsolomon}.
}

Assuming $n$ out of $m$ coding.
\lstinline{extend} takes a list of $n$ data chunks and an argument for the number of required parities. It returns the parity chunks only.
\lstinline{repair} takes a list of $m$ chunks (extended with \emph{all} parities) and an argument for the number of parities $p=m-n$, that designate the last $p$ chunks as parity chunks. It returns the list of $n$ repaired data chunks only.
The encoder does not know which parts are invalid, so missing or invalid chunks should be set to \lstinline{nil} in the argument to repair.
If parity chunks are needed to be repaired, you call \lstinline{repair @chunks with @parities; extend with @parities}

\begin{definition}[CRS erasure code interface definition]\label{def:crs}
\begin{lstlisting}[language=buzz1]
// /crypto/crs

define function extend @chunks []chunk 
    with @parities uint
    return [@parities]chunk

define function repair @chunks []chunk   
    with @parities uint
   return [@chunks length - @parities]chunk

\end{lstlisting}
\end{definition}

\begin{definition}[CRS erasure coding parameters]\label{def:crs-params}
\begin{lstlisting}[language=buzz1]
// /crypto/crs
define strategy as "race"|"fallback"|"disabled"

define type params 
    parities uint
    strategy 
     
\end{lstlisting}
\end{definition}


\subsection{State store \statusgreen}\label{spec:format:statestore}
% \input{specs/api/statestore.tex}

\begin{definition}[State store]\label{def:state-store}
\begin{lstlisting}[language=buzz1]
// /statestore

define type key []byte
define type db  []byte
define type value []byte

define function create db

define function destroy db

define function put value
    to db
    on key
    
define function get key 
    from db
    return value
\end{lstlisting}
\end{definition}


\subsection{Local context and configuration \statusgreen}\label{spec:format:local}
% \input{specs/api/local.tex}


\begin{definition}[Context]\label{def:scontext}
\begin{lstlisting}[language=buzz1]
// /context

define type contract as "buzz"|"chequebook"|"postage"|""

define type context
    contracts [contract]ethereum/address 
\end{lstlisting}
\end{definition}

\section{Bzz address\statusgreen}\label{spec:format:bzzaddress}
\subsection{Overlay address \statusyellow}

Swarm's overlay network uses 32-byte addresses. In order to help  uniform utilisation of the address space,  these addresses must be derived using a hash function. A Swarm node must be associated with an Ethereum account called the \gloss{Swarm base account} or \gloss{bzz account}.
The node's overlay address is derived as hash of the corresponding  public key.
 of the 

\begin{definition}{Swarm overlay address of node A}\label{def:overlay}
\begin{equation}
\mathit{overlayAddress}(A) \defeq \mathit{Hash}(\mathit{ethAddress}|\mathit{bzzNetworkID})         
\end{equation}
where
\begin{itemize}
\item $\mathit{Hash}$ is the 256-bit Keccak SHA3 hash function
\item \emph{ethAddress} - the ethereum address  (bytes,  not hex) derived from the node's base account public key: $\mathit{account}\defeq\mathit{PubKey}(K_A^\mathit{bzz})[12:32]$), where
    \begin{itemize}
    \item \emph{PubKey} is the \emph{uncompressed} form of the public key of a keypair \emph{including} its $04$ (uncompressed) prefix.
    \item $K_A^\mathit{bzz}$ refers to the node's bzz account key pair
    \end{itemize}
\item \emph{bzzNetworkID} is the bzz network id of the swarm network serialised as a little-endian binary \emph{uint64}.
\end{itemize}
\end{definition}

\subsection{Underlay address \statusyellow}

To enable peers to locate the a node on the network, the overlay address is paired with an underlay address. The underlay address is a string representation of the node's network location on the underlying transport layer. 

\begin{definition}{underlay}\label{def:underlay}
\begin{lstlisting}[]
// ID: /swarm/handshake/1.0.0/
}
\end{lstlisting}
\end{definition}

% As long as swarm runs of devp2p, we use the \gloss{enode URL scheme} representation:

% \begin{definition}{devp2p underlay address of a node}\label{def:underlay}
% $\mathit{underlay}(n) \defeq $
% \begin{verbatim}
%     "enode://"<NODEID>"@"<HOST>(":"<TCPPORT>("?udp="<UDPPORT>))
% \end{verbatim}
% \end{definition}


\subsection{BZZ address \statusyellow}

Bzz address is functionally the pairing of overlay and underlay addresses. In order to ensure that an address is derived from an account the node possesses as well as verifiably attest to an underlay address a node can be called on, bzz addresses are communicated in the following transfer format:

\begin{definition}{Swarm bzz address transfer format} \label{def:bzz-address}
\begin{lstlisting}[]
// ID: /swarm/handshake/1.0.0/bzzaddress
// Serialization: Varint delimited Protobuf
syntax = "proto3";

message BzzAddress {
    bytes Underlay = 1;
    Signature Sig  = 2;
    bytes Overlay  = 3; 
}
\end{lstlisting}
\end{definition}

Here the signature is attesting to an underlay address for a network:

\begin{definition}{signed underlay address of node A}\label{def:signed-underlay}
\begin{equation}
\mathit{signedUnderlay}(A) \defeq \mathit{Sign}(\mathit{underlay}|\mathit{bzzNetworkID})         
\end{equation}
\end{definition}

of the underlay with bzz network ID appended as plaintext and hashes the resulting public key together with the bzz network ID. 

\begin{definition}{bzz types}\label{def:bzz-types}
\begin{lstlisting}[language=buzz1]
// /swarm/bzz

define type overlay as [segment.size]byte

define type account as crypto/keypair

define type eth.address of account 
as
    hash @account public key from 12

define function address of account 
    for uint64 as @network
as
    hash @account eth.address
        and @network 
            as overlay

\end{lstlisting}
\end{definition}

Note that the overlay address is not explicitly part of the construct. It is not needed since it can be calculated from signature, underlay and the network ID.



In order to get the overlay address from the transfer format peer info, one recovers from signature the peer's base account public key using the plaintext that is constructed as per \ref{def:signed-underlay}. From the public key, the overlay can be calculated as in \ref{def:overlay}.


\section{Chunks, encryption and addressing\statusyellow}
\subsection{Content addressed chunks \statusgreen}\label{spec:format:chunks}
First let us define some basic types, such as \emph{payload, span, segment}. These fixed length byte slices enables verbose expression of fundamental units like \lstinline{segment size} or \lstinline{payload size}.

\begin{definition}[segment, payload, span, branches]\label{def:chunk-constants}
\begin{lstlisting}[language=buzz1]
// /chunk

define type segment as [32]byte    // unit for type definitions
define type payload as [:4096]byte // variable length max 4Kilobyte
define type span as uint64         // little endian binary marshalled

define function branches 
    as payload size / segment size

\end{lstlisting}
\end{definition}

Now  let's turn to the definition of \emph{address,  key} and \emph{reference}:                                             

\begin{definition}[Chunk reference]\label{def:chunk-reference}
\begin{lstlisting}[language=buzz1]
// /chunk

define type address as [segment size]byte

define type reference 
    address                   // result of bmt hash
    key if context.encryption // decryption key optional (context dependent)

\end{lstlisting}
\end{definition}

Now, define chunk as a object with span  and payload.

\begin{definition}[Content addressed chunk]\label{def:chunks}
\begin{lstlisting}[language=buzz1]
// /chunk

define type chunk
    span      // length of data span subsumed under node
    payload   // max 4096 bytes 

define function address of chunk
as
    @chunk payload bmt/hash with @chunk span 

define function create from payload
    ?over span
as
    @span = @payload length if no @span
    @chunk = chunk{ @span, @payload }
    return @chunk if no context encryption
    @key = encryption.key for @chunk 
    @chunk encrypt with @key
\end{lstlisting}
\end{definition}

Where length is the content of the length field and reference  size is the sum of size of the referencing hash value and that of the decryption key, which is currently 64, as we use 256-bit hashes and 256-bit keys.

In order to remove the padding after decryption before returning the plaintext chunk. 

\begin{definition}[Span to payload length]\label{def:span}
\begin{lstlisting}[language=buzz1]
// /chunk

define function payload.length of span
as
    while @span >= 4096 
        @span = @span + 4095
           / 4096
           * reference size
    return @span
\end{lstlisting}
\end{definition}

Finally, we can define the public API of chunks  for retrieval and storage.

\begin{definition}[Chunk API retrieval]\label{def:retrieve}
\begin{lstlisting}[language=buzz1]
// /chunk

define function retrieve reference
has api GET on "chunk/<reference>"
as 
    retrieve @reference address as chunk
        (decrypt with @reference key if @reference key)
\end{lstlisting}
\end{definition}


\begin{definition}[Chunk API: storage] \label{def:store}
\begin{lstlisting}[language=buzz1]
// /chunk


define function store payload
    ?over span 
has api POST on "chunk/(?span=<span>)"
    from payload as body
as 
    @chunk = create from @payload over @span
    reference{ @chunk address, @chunk key }
\end{lstlisting}
\end{definition}



\subsection{Single owner chunk \statusgreen}\label{spec:format:soc}
Single owner chunks are the second type of chunk in swarm (see \ref{sec:single-owner-chunks}). They constitute the basis of swarm feeds.


\begin{definition}{single owner chunks}\label{def:soc}
\begin{lstlisting}[language=buzz1]
// /soc

// data structure for single owner chunk
define type soc 
    id         [segment size]byte // id 'within' owner namespace
    signature  crypto/signature   // owner attests to <content, id> 
    chunk                         // content: embeds a content chunk

// constructor for single owner chunks
define function create from chunk 
    by @owner crypto/keypair
    on @id [segment size]byte
 as
    @sig = crypto/sign @id and @chunk address by @owner
    soc{ @id, @sig, @chunk }

define function address soc
as
    hash @soc id and @soc signature signer address
    
\end{lstlisting}
\end{definition}

\begin{definition}{single owner chunk retrieval}\label{def:soc-retrieve}
\begin{lstlisting}[language=buzz1]
// /soc

define function retrieve @id [segment size]byte
    by @owner ethereum/address
    ?with key
has api GET on "soc/<owner>/<id>(?key=<key>)"
as
    retrieve hash @id and @owner 
        as soc
            chunk (decrypt with @key if @key)
\end{lstlisting}
\end{definition}


\begin{definition}{single owner chunk storage}\label{def:soc-store}
\begin{lstlisting}[language=buzz1]
// /swarm/soc

define function store payload
    on @id [segment size]byte
    by @owner ethereum/address 
    ?over span
has api POST on "soc/<owner>/<id>?span=<span>&encrypt=<encrypt>"
    from <payload> as body
as 
    @span = @payload length if no @span
    @chunk = chunk{ @span, @payload }
    if context encryption then
        @key = encryption.key for @chunk 
        @chunk encrypt= with @key
    @soc = create from @chunk on @id by private key of @owner
    reference{ @soc store address, @key }
             
\end{lstlisting}
\end{definition}

\subsection{Binary Merkle Tree Hash \statusyellow}\label{spec:format:bmt}
The hashing method used to obtain the address of the default content addressed chunk is called the \gloss{binary Merkle tree hash}, or \emph{BMT Hash} for short. 

\subsubsection{Calculating the BMT hash}

The base segments of the binary tree are subsequences of the chunk content data. 
The size of segments is 32  bytes, which is the digest size of the \emph{base hash} used to construct the tree. 
Given the Swarm hash tree used to represent files (see \ref{spec:format:files}) assumes that intermediate chunks package references to other chunks. 

Obtaining the BMT hash of a sequence involves the following steps:

\begin{enumerate}
\item \emph{padding} - If the content is shorter than the maximum chunk Size  (4096 bytes, \ref{sec:content-addressed-chunks}), it is padded with zeros up to chunk size. Note that this zero padding is only for hashing and does not impact chunk data sizes.
\item \emph{chunk data layer} - Calculate the base hash of \emph{pairs of segments} in the padded chunk, i.e., segment size ($2 * 32$) units of data and concatenate the results.
\item \emph{building the tree} - Repeat previous step on the result until the result is just one section
\item \emph{calculate span} - Calculate the span of the data, i.e., the size of the data that is subsumed under the chunk represented by the unpadded data as a 64-bit little-endian integer value (see  \ref{spec:format:files})            
\item \emph{integrity protection} - Prepend the span to the root hash of the binary tree and calculate the \emph{base hash} of the data
\end{enumerate}

\begin{definition}[BMT hash]\label{def:bmt-hash}
\begin{lstlisting}[language=buzz1]
// /bmt

define function hash payload 
    with span
as
    @padded = @payload as [:chunk size]byte    // use zero padding 
    // for BMT hashing only
    hash @span and root of @padded over chunk size 
    
define function root of @section []byte
    over @len uint
as
    return hash @section        // data level
        if @len == 2 * segment size
    @len /= 2                  // recursive call
    @children = @section each @len go self over @len
    wait for @children 
        join hash
    

\end{lstlisting}
\end{definition}

\subsubsection{Inclusion proofs}

Having the segments align with the hashes packaged in these chunks one can extend the notion of inclusion proofs to files.
The BMT hash enables compact 3rd party verifiable segment inclusion proofs.

\red{to be specified}
\subsection{Encryption \statusyellow}\label{spec:format:encryption}

Plaintext chunks consist of a 8-byte length field encoding the size of the binary blob accessible through a Merkle structure for which this chunk is the root, followed by the chunk payload which is at most 4096 bytes. Before encrypting, each chunk payload is padded to exactly 4096 bytes, as its actual length can be deduced from the length field as follows:

\begin{lstlisting}

payloadLength := length
while payloadLength > 4096
        payloadLength := payloadLength + 4095
        payloadLength := payloadLength / 4096
        payloadLength := payloadLength * refSize
\end{lstlisting}

Where length is the content of the length field and refSize is the sum of size of the referencing hash value and that of the decryption key, which is currently 64, as we use 256-bit hashes and 256-bit keys.

This procedure can be used to remove the padding after decryption before returning the plaintext chunk. 


It is important to emphasize, that encrypted Swarm chunks are not different from plaintext chunks and therefore no change is needed on the P2P protocol level to accommodate them. The proposed encryption scheme is end-to-end, meaning that encryption and decryption is done on endpoints, i.e., where the http proxy layer runs. This has an important consequence that public gateways cannot be used for encrypted content.

The reference of a single chunk (and the whole content) will be the concatenation of the hash of encrypted data and the decryption key. This means the encrypted  swarm reference (64 bytes) will be longer than the unencrypted one (32 bytes). When a node syncs encrypted chunks, it does not share the full references (or the decryption keys) with the other nodes in any way.  Therefore, other nodes will be unable to access the original data, or in fact, even to detect whether a chunk is encrypted.
        

FIGURE: symmetric encryption in swarm 


The segment keys are generated by hashing the encryption seed with the counter and hash that again. This second step is required so that a segment can be selectively disclosed in a 3rd party provable way yet without compromising the security of the rest of the chunk.

Alternative encryptions of the same chunk can be generated cheaply if the metadata to be prepended to the BMT root hash contains enough entropy.


\section{Files, manifests and data structures\statusyellow}\label{spec:format:data-structures}
\subsection{Files and the Swarm hash \statusyellow}\label{spec:format:files}

This table gives an overview of data sizes a chunk span represents, depending on the level of recursion.

\begin{table}[ht]
\begin{tabular}{|r||r|r|r||r|r|r|}
\hline
&\multicolumn{6}{|c|}{span}\\\hline
&\multicolumn{3}{|c|}{unencrypted}
&\multicolumn{3}{|c|}{encrypted}\\\hline
level & chunks & $\mathit{log}_2$ of bytes & standard & chunks & $\mathit{log}_2$ of bytes & standard \\
\hline\hline
0 & 1 & 12 & 4KB & 1 & 12 & 4KB \\\hline
1 & 128 & 19 & 512KB & 64 & 18 & 256KB \\\hline
2 & 16,384 & 26 & 67MB & 4,096 & 24 & 16MB \\\hline
3 & 2,097,152 &33 & 8.5GB & 262,144 &  30 & 1.07GB\\\hline
4 & 268.44M & 40  & 1.1TB & 16.78M & 36 & 68.7GB\\\hline
5 & 34,359.74M & 47 & 140TB & 1,073.74M & 42  & 4.4TB\\\hline
\end{tabular}
\caption{Size of chunk spans}
\end{table}




\subsubsection{Calculating the Swarm Hash}

Client-side custom redundancy is achieved by CRS erasure coding (see \ref{sec:erasure} and \ref{sec:features}); using it neccessitates 
\begin{definition}{CRS erasure coding parameters}\label{def:crs-params}
\begin{lstlisting}[language=buzz1]
// /swarm/file

define type crs/params 
    parities uint
    strategy _race|_fallback|_disabled
     
\end{lstlisting}
\end{definition}
      
\begin{definition}{the swarm hash tree for files}\label{def:swarm-hash}
\begin{lstlisting}[language=buzz1]
// /swarm/file

define function encode []chunk stream as @levels
    for @level uint
as
    @chunks = @levels at @level     // read chunk stream 
    @crs = context crs              // 
    @m = branches (- @crs parities if @crs) 
    
    @parent = @chunks up to @m      // read up to m chunks from stream
        (append crs/extend with @crs parities if @crs)
            each chunk/store        // package children references
                join as chunk
                
    if @levels length == @level+1 then
        @levels append= stream{}
        go self @levels for @level+1
    
    @levels at @level+1 append= @parent
    if no @chunks then
        close @levels at @level + 1
    else
        self @chunks for @level
            
define function split byte stream as @data
as
    @data each chunk size as chunk

define function  upload byte stream as @data
as
    @data split each branches     // 
        go as chunk 
            encode for 0  // 
    @top = @levels each wait for  // wait for all levels to close
    return @top at 0              // return root hash as address        
\end{lstlisting}
\end{definition}
                   

\begin{definition}{files and file retrieval}\label{def:file-retrieval}
\begin{lstlisting}[language=buzz1]
// /swarm/file

define function download reference
as
    chunk/retrieve @reference       // root chunk retrieval
        decode each as chunk data   // recursion


define function decode chunk
  with @limit uint8
as
    @leaf = @chunk span payload.length <= chunk size
    return @chunk if @leaf or branches @limit 
    
    @crs = context crs
    @all = @m = branches 
    if @crs then
        @m -= @crs parities
        if @crs strategy is not {race} then 
        @all = @m 
        
    @chunks = @chunk segments up to @all 
        each go as reference retrieve 
    wait for @m in @chunks and cancel 

    if @crs then 
        @chunks = @chunks crs/repair with @crs parities 

    return @chunks if @level == @limit 
    @chunks each go self @limit + 1  

\end{lstlisting}
\end{definition}


\subsection{Manifests \statusyellow}\label{spec:format:manifests}
Manifests represent a mapping of strings to references (see \ref{sec:collections}). The primary purpose is to implement document collections (websites) on swarm and enable URL-based addressing of content. This section defines the data structures relevant for manifests as well as the algorithms for lookup and update which implement the manifest API (see \ref{sec:manifests-ux} and \ref{spec:api:manifest}).

A manifest entry can be conceived of as metadata about a file pointed to and retrievable by its reference (see \ref{spec:format:files}). The metadata is quite diverse, ranging from information needed for access control, file information similar to one given on file systems, information needed for erasure coding (see \ref{sec:erasure}, \ref{sec:headers} and \ref{spec:format:erasure}), information for browser, i.e.,  response headers such as content type (MIME info) and most importantly the reference to the file. Using manifests as simple key-value store is exemplified by access control (see \ref{sec:access-control}, \ref{sec:access-control-ux} for discussion as well as \ref{spec:format:access-control} and \ref{spec:api:access-control} for the specification).

\begin{definition}{manifest entry}\label{def:manifest-entry}
\begin{lstlisting}[language=buzz1]
// /swarm/manifest

// manifest entry encodes attributes 
define type entry 
    info                 // FS file/dir info
    access.control/info  // access conrrol info
    crs/info             // erasure coding - CRS params
    reference            // reference

define type info
    content.type  [32]byte

\end{lstlisting}
\end{definition}

\begin{definition}{manifest data structure}\label{def:manifests}
\begin{lstlisting}[language=buzz1]
// /swarm/manifest

define type node 
    entry  *entry          // reference to chunk serialised as entry
    forks  [<<256]fork     // sparse array of max 256 fork

// fork encodes a branch
define type fork 
    prefix segment   // compaction 
    node   *node     // reference to chunk serialised as node

// manifest entry encodes attributes 
define type entry 
    info                 // FS file/dir info
    access.control/info  // access conrrol info
    crs/info             // erasure coding - CRS params
    reference            // reference

define type info
    content.type  segment

\end{lstlisting}
\end{definition}

\begin{definition}{manifest path lookup}\label{def:manifests-lookup}
\begin{lstlisting}[language=buzz1]
// /swarm/manifest

define function lookup *node
    on @path []byte
as 
    @context access.control = @node entry access.control 
    // manifest is  a compacted trie
    @fork = @node forks at head @path 
    // if @path empty, the  paths matched return the entry
    if no @path then 
        return @node entry

    if @fork prefix is prefix of @path then // including == 
      return self @fork node 
        on @path from @fork prefix length

   fail with "not found"

\end{lstlisting}
\end{definition}


\begin{definition}{manifest update}\label{def:manifest-update}
\begin{lstlisting}[language=buzz1]
define function add reference  
    to *node 
    on @path []byte 
as
    // if called on nil call on zero value
    if no @node then 
        self @reference to node{} on @path
        return

    // if empty path then change entry field of node
    if no @path then
        @node entry = @reference
        @node store 
            return

    // lookup the fork based on the first byte of path
    @fork = @node forks at head @path
    // if no fork yet, add the singleton node 
    if no @fork then
        @node forks at head @path =
            fork{@path, store node{@reference as entry}}
        @node store 
            return

    @common = prefix of @path and @fork prefix // common cannot be empty
    @rest = @fork prefix from @common length
    @newnode = node{}
    @newnode forks at head @rest = fork{@rest, @fork node}
    @midnode = self @reference to @newnode on @path from @common length 
    @node forks at head @path = fork{ @midnode, @common } 
    @node store
\end{lstlisting}
\end{definition}
% \section{Entanglement coding \statusred}\label{spec:format:entanglements}
%\input{specs/format/entanglement.tex}
% composite api, resolver, tags
\subsection{Resolver}\label{spec:format:bzz-api}


\begin{definition}[Resolver]\label{def:resolver}
\begin{lstlisting}[language=buzz1]
// /soc

             
\end{lstlisting}
\end{definition}


\subsection{Tags}\label{spec:format:tags}
\subsection{Storage}\label{spec:format:bzz-api}


\section{Access Control \statusgreen}\label{spec:format:access-control}
\begin{definition}[Access control]\label{def:ac}
\begin{lstlisting}[language=buzz1]
// /access

define type auth as "pass"|"pk" 
define type hint as [segment size]byte

// access control parameters
define type params
	auth                          // serialises uint8
	publisher crypto/pubkey       // 65 byte
	salt                          // salt for scrypt/dh
	hint                          // hint to link identity
	act       *node               // reference to act manifest root
	kdf                           // params for scrypt
} 

// root access manifest
define type root
    params
    reference

\end{lstlisting}
\end{definition}


\begin{definition}[Session key and auth with credentials]\label{def:session-key}
\begin{lstlisting}[language=buzz1]
// /access

define function session.key.pass from hint 
    with  salt 
    using kdf
as
    crypto/scrypt from input/password using @hint
        with @salt using @kdf

define function session.key.pk from hint
    with crypto/pubkey
    using salt
as
    crypto/shared.secret between 
        input/select key by @hint
            and @pubkey 
        hash with @salt
      
define function session.key 
    using params
as
    if @params auth == "pass" then
        return session.key.pass from  @params hint 
            with @params salt using @params kdf
                
    session.key.pk from @params hint 
        with @params publisher using @params salt
\end{lstlisting}
\end{definition}


\begin{definition}[Access key]\label{def:access-key}
\begin{lstlisting}[language=buzz1]
// /access
define function access.key 
    using params
as 
    @key = session.key using @params 
    return @key if no @params act
    act.lookup @key in @params act
   
define function act.lookup key
    in @act *node
as
        manifest/lookup hash @key and 0
            in @act
                xor hash @key and 1


\end{lstlisting}
\end{definition}


\begin{definition}[Access control API]\label{def:ac-api}
\begin{lstlisting}[language=buzz1]
// /access

// control 
define function lock reference
    using params
has api POST on "/access/<address>"
    with @params as body
as
    @key = hash @reference address and @key 
    @encrypted = @reference as bytes 
        crypto/crypt with @key
    root{ @params, @encrypted } 
        store
        

define function unlock address
has api GET on "access/<address>"
as
    @root = retrieve @address 
        try as root otherwise return @address
    @key = access.key using @root params
    @root encrypted crypto/crypt with @key
        as *node
    
\end{lstlisting}
\end{definition}

\begin{definition}[ACT manipulation API]\label{def:act-api}
\begin{lstlisting}[language=buzz1]
// /access

define type act as manifest/node

define function add @keys []crypto/pubkey
    to *root
has api PUT on "/access/<root>/" 
    with @keys as body
as 
    // get params from the root access structure
    @params = retrieve @root as root params
    @access.key = access.key using @params                      
    @keys each @key
        @session.key = session.key using @params
        manifest/add @access.key xor hash @session.key with 1
            to @act on hash @session.key with 0  
            

define function remove @keys []crypto/pubkey
    from *root   
has api DELETE on "/access/<root>" 
    with @keys as body
as 
    // get params from the root access structure
    @params = retrieve @root as root params
    @keys each @key
        @session.key = session.key using @params
        manifest/remove hash @session.key with 0  
            from @params act 
             

\end{lstlisting}
\end{definition}

\section{PSS \statusyellow}

\subsection{PSS message\statusgreen}
\label{spec:format:pss-messsage}
\subsection{Pss message and trojan chunk}

Pss has two fundamental types, a message and a trojan chunk structure which wraps the encrypted serialised message and contains a nonce that is mined to make the resulting chunk's content address (BMT hash) to match the targets.


\begin{definition}{pss message and trojan chunk}\label{def:pss-message}
\begin{lstlisting}[language=buzz1]
// /swarm/pss


define type topic        as [32]byte       // obfuscated topic matcher
define type targets      as [][]byte       // overlay prefixes 
define type recipient    as crypto/keypair

// pss message
define type message 
    seal    segment            
    payload [!4030]byte    // varlength padded to 4030B
    
// trojan chunk
define type trojan 
    nonce   segment           // the nonce to mine 
    message [4064]byte        // encrypted msg 
\end{lstlisting}
\end{definition}


The message is encoded in a way that allows integrity checking and at the same time obfuscates the topic. The operation to package the payload with a topic is called \emph{sealing}


\begin{definition}{sealing the message}\label{def:pss-sealing}
\begin{lstlisting}[language=buzz1]
// /swarm/pss

define function seal []byte as @payload
    with topic
as
    @seal = hash @payload and @topic // obfuscate topic
        xor @topic          
    return message{ @seal, @payload }

define function unseal message
    with topic 
as
    @seal = hash @message payload and @topic 
    if @topic == @seal xor @message seal then // check 
        return @payload 
    return nil
    
\end{lstlisting}
\end{definition}

Functions \lstinline{wrap/unwrap} transform between message and trojan chunk. \lstinline{wrap} takes an optional recipient public key to asymmetrically encrypt the message.
The targets are a list of overlay address prefixes derived from overlay addresses of recipients, with length specified to guarantee that a chunk matching it will end up with the recipient solely as a result if  push-syncing.   

\begin{definition}{pss message wrapping and mining trojan chunk}\label{def:wrap}
\begin{lstlisting}[language=buzz1]
// /pss

// wraps the message in a trojan chunk
define function wrap message 
    for recipient
    to  targets
as 
    @msg = @message 
        (crypto/encrypt for @recipient if @recipient) 

    @nonce = crypto/mine @n such that
        @targets any is prefix of
            trojan{@n, @msg} as chunk address 
    trojan{@nonce, @msg} as chunk 

\end{lstlisting}
\end{definition}

As a chunk arrives at the node,  \lstinline{pss/deliver} is called as a hook by the storage component.
First the message is unwrapped using the recipient private key and unsealed with all the topics subscribed to by API clients. If the unsealing is successful, message integrity as well as topic matching is proven so the payload is written into the stream registered for the topic in question.

\begin{definition}{pss message unwrapping and local trojan delivery}\label{def:unwrap}
\begin{lstlisting}[language=buzz1]
// /pss

// unwrap trojan chunk 
define function unwrap chunk
    for recipient
as
    @chunk bytes 
        (crypto/decrypt  for @recipient if @recipient)
            as message

// mailbox is a handler type, expects payload
// sent sealed with the topic to be delivered via the stream 
define type mailbox
    topic
    deliveries stream of []byte 
    
define context mailboxes as []mailbox

define function deliver chunk
    @msg = @chunk unwrap for context recipient
    mailboxes each @mailbox 
        @payload = unseal @msg  with @mailbox topic
        if @payload then 
            write @msg payload 
                to @mailbox deliveries 
    

\end{lstlisting}
\end{definition}


\begin{definition}{pss/send API endpoints}\label{def:send}
\begin{lstlisting}[language=buzz1]
// /pss
// send API endpoint
define function send payload
    about topic
    to    targets
has api POST on "/pss/<topic>"
as 
    context tag = tag/tag{}
    seal @payload with @topic        // seal with topic
        wrap for @recipient          // encrypt if given recipient
            to @targets              // mine nonce and returns trojan chunk
                store                // to be sent by push-sync
    return tag                       // tag to monitor status 
    
\end{lstlisting}
\end{definition}

\begin{definition}{pss/receive API endpoint}\label{def:receive}
\begin{lstlisting}[language=buzz1]
// /pss
// send API endpoint
define function receive about topic 
    on uint64 @channel
as 
    @stream = open @channel
    context mailboxes append= mailbox{ @topic, @stream }
    
define function cancel topic 
\end{lstlisting}
\end{definition}

\subsection{Envelopes}

\begin{definition}{envelope}\label{def:pss-envelope}
\begin{lstlisting}[language=buzz1]
// /pss

define type envelope
    id  [32]byte
    sig crypto/signature
    ps  post/stamp   
    
\end{lstlisting}
\end{definition}

\subsection{Update notifications \statusred}\label{spec:format:update-notifications}
%\input{specs/format/update-notifications.tex}

\subsection{Chunk recovery  \statusyellow}\label{spec:format:recovery}

\begin{definition}{recovery request}\label{def:recovery-request}
\begin{lstlisting}[language=buzz1]
// /swarm/recovery

define type request address
    with @response bool
as  
    if @response then
      @key = local keys get random
      @address = crypto/silly// hash                        
      crypto/sign @address @id by @key 
    
    

\end{lstlisting}
\end{definition}

\begin{definition}{recovery response}\label{def:recovery}
\begin{lstlisting}[language=buzz1]

define function request address
    with @response bool
as                                                      
    pss/send @payload about "RECOVERY" to context targets
    

\end{lstlisting}
\end{definition}

\begin{definition}{envelope}\label{def:recovery-response}
\begin{lstlisting}[language=buzz1]

define function response []byte as @payload
    by @key crypto/keypair
as
    @key = crypto/generate if no @key 
    crypto/mine @nonce such that
        context overlay to @depth is prefix of 
            soc @payload on @nonce by @key
                address 


\end{lstlisting}
\end{definition}
   

\section{Postage stamps \statusorange}\label{spec:format:postage-stamps}

\subsection{D}
\subsubsection{Witness type}

There can be different implementations of postage stamps that differ in the structure and semantics of the \emph{proof of payment}. To allow for new cryptographic mechanisms to be used as they are developed, the \lstinline{witnessType} argument indicates the type of the witness used. 

Witness type $0$ stands for ECDSA witness, which is an ECDSA signature on the byte slice resulting from the concatenation of   1) preamble constant 2) chunk hash 3) batch reference 4) valid until date.% 
%
\footnote{The binary encoding of the ECDSA signature is 65 bytes resulting from the concatenation of the $r$ (32 bytes), $s$ (32 bytes) and $v$ (1 byte) parameters of the signature, in this order. The signature is calculated on the secp256k1 elliptic curve, just like the signatures of Ethereum transactions.}
%
This is the bare minimum that postage stamp contracts and clients must implement.%
%
\footnote{The ECDSA witness is the simplest and cheapest solution both in terms of gas consumed by the stamp verification contract and in terms of computational resources used off chain. Also, it does not rely on cryptographic assumptions in addition to those on which Ethereum critically relies, therefore as long as Ethereum is considered cryptographically secure, no advance in cryptorgraphy can render this witness type insecure. This is the justification for this witness type to be the only mandatory witness type to be implemented.}

Witness type $1$ refers to the RSA witness, which is an RSA signature on the same 128 bytes as above. The binary encoding of the RSA signature is of variable length, and is an Solidity ABI encoded array of the RSA signature $s$.%
%
\footnote{as defined in \emph{PKCS \#1}, \url{https://tools.ietf.org/html/rfc8017} and the RSA public key parameters $n$ (RSA modulus) and $e$ (public exponent).}

The RSA witness is specified so that blind stamping services can be implemented in a simple fashion, in order to mitigate the privacy issues arising from the ability to link chunks signed with the same private key. Even though blind ECDSA signatures also exist, their protocol requires more rounds of communication, making the implementation of such a service more complex, more error-prone and less performant. 

The inclusion of the entire public key in each RSA witness rather than storing the public key in contract state and just referencing it from the witness is justified by reducing the gas costs of interactions with the contract as well as future-proofing the design in case contract state rent is introduced in Ethereum. These considerations are more important than the brevity of postage stamps, marginally reducing the bandwith costs of uploading and forwarding stamped content.

Note that cryptographic advances can render RSA witnesses insecure without rendering Ethereum insecure, therefore RSA witnesses can be phased out in future versions of the protocol, if the security of RSA signatures gets compromised. Note, furthermore, that such blind signing services are not entirely trustless, through the damage they can incur is bounded. Trustless blind stamping services based on ZK proofs are not feasible at this stage, as the current algorithms are not sufficiently performant for the purpose, but given the rapid advances in the field, the development of suitable algorithms can be expected in the future, in which case a corresponding witness type will have to be specified in a separate SWIP.

\subsubsection{Contract Upgrades}

In order to facilitate the upgrade of the contract either in case of a discovered vulnerability or some feature extension (such as adding new witness types), it is recommended that the part holding the funds with the database of payments and the part that verifies witnesses are in separate contracts so that a backwards-compatible upgrade can be performed with minimal disruption.

In order to avoid centralized control, it is also recommended that it is the witness-verifying contract that is referenced in client configuration so that client operators can independently decide for themselves when and whether to switch to a new contract, as they become available.


Nodes participating in the same postage system are configured to reference the same contract on the same blockchain. This contract
*must* have an accessor call with the following ABI signature:

\begin{lstlisting}

```json
{
  "name": "postage",
  "type": "function",
  "inputs": [
    {
      "name": "payloadHash",
      "type": "bytes32"
    },
    {
      "name": "postagePaid",
      "type": "uint256"
    },
    {
      "name": "beginValidity",
      "type": "uint256"
    },
    {
      "name": "endValidity",
      "type": "uint256"
    },
    {
      "name": "witnessType",
      "type": "uint8"
    },
    {
      "name": "witness",
      "type": "bytes"
    }
  ],
  "outputs": [
    {
      "name": "",
      "type": "bool"
    }
  ]
}
```

\end{lstlisting}

This accessor method returns \lstinline{true} if the proof embodied by \lstinline{witness} checks out for all other arguments within the claimed 
validity period, i.e. when \lstinline{block.timestamp} (the output of \lstinline{TIMESTAMP EVM} opcode) is between \lstinline{beginValidity} (inclusive) and 
\lstinline{endValidity} (exclusive). Outside of the validity period, the return value is \lstinline{undefined}.

The binary serialization of the postage stamp consists of a single 5-element array containing the arguments in the same order. 


\begin{definition}{postage stamp}\label{def:postage-stamp}
\begin{lstlisting}[language=buzz1]
// /postage

define type batchid as [32]byte  
define type address as bzz/address
define type witness as crypo/signature

// postage stamp
define type stamp  
    batchid
    address
    witness

define function valid stamp
as 
    // check validity on blockchain
    ethereum call context contracts postage
        is @stamp valid
        
\end{lstlisting}
\end{definition}


% \section{Honey token and multi-chain support}\label{spec:format:honey}
%\input{specs/format/honey.tex}


\chapter{Protocols}\label{spec:protocol}

\section{Introduction \statusorange}\label{spec:protocol:intro}

\subsection{Underlay network}

The \texttt{libp2p} networking stack provides all required properties for the swarm underlay network laid out in \ref{sec:underlay-transport}.

\begin{enumerate}
\item Addressing is provided in the form of so called \emph{multi address} for every node, which is referred here as the underlay address. Every node can have multiple underlay addresses depending on transports and network listening addresses that are configured.
\item Dialing is provided over \texttt{libp2p} supported network transports.
\item Listening is provided by \texttt{libp2p} supported network transports.
\item Live connections are established between two peers and kept open for accepting or sending messages.
\item Channel security is provided with \texttt{TLS} and \texttt{libp2p secio} stream security transport.
\item Protocol multiplexing is provided by \texttt{libp2p mplex} stream multiplexer protocol.
\item Delivery guarantees are provided by using \texttt{libp2p} bidirectional streams to validate the response from the peer on sent message.
\item Serialization is not enforced by \texttt{libp2p}, as it provides byte streams allowing flexibility for every protocol to choose the most appropriate serialization. The recommended serialization is \texttt{Protobuf} with \texttt{varint} delimited messages in streams.
\end{enumerate}

\subsection{Protocols and streams}

Communication between peers is organised in protocols as logical units under a unique name that may define one or more \emph{streams}. \texttt{libp2p} provides streams as the basic channel for communication. Streams are full-duplex channels of bytes, multiplexed over a singe connection between two peers.

Every stream defines:

\begin{itemize}
\item a version that follows semantic versioning in semver form
\item data serialization definitions
\item sequence of data passing between peers over a full-duplex stream
\end{itemize}

Streams are identified by \texttt{libp2p} case-sensitive protocol IDs. The following convention is used construct stream identifiers:

\begin{lstlisting}[basicstyle=\ttfamily]
/swarm/ProtocolName/ProtocolVersion/StreamName
\end{lstlisting}

\begin{itemize}
\item All stream IDs are prefixed with \texttt{/swarm}.
\item \texttt{ProtocolName} is an arbitrary string that identifies the protocol.
\item \texttt{ProtocolVersion} is a string in semver form that is used to specify compatibility between protocol implementations over time.
\item \texttt{StreamName} is an arbitrary string that identifies a stream defined as part of the protocol.
\end{itemize}

\subsection{Data exchange sequences}

A data passing sequence must be synchronous under one opened stream. Multiple streams can be opened at the same time that are multiplexed over the same connection exchanging data independently and asynchronously. Streams may use different data exchange sequences such as:

\begin{itemize}
\item \emph{single message sending} - not waiting for the response by the peer if it is not needed before closing the stream.
\item \emph{multiple message sending} - a series of data that is sent to a peer without reading from it before closing the stream.
\item \emph{request/response} - requires a single response for a single request before closing the stream.
\item \emph{multiple requests/response cycles} - require a synchronous response after every request before closing the stream.
\item \emph{exact message sequence} -  requires multiple message types over a single stream in an exact order (see the handshake protocol in \ref{spec:protocol:hive}).
\end{itemize}

Streams have predefined sequences that are kept as simple as possible for a single purpose. For complex message exchanges, multiple streams should be used.

Streams may be short lived for immediate data exchange or communication, or long lived for notifications if needed.


\section{Swarm protocol basics\statusgreen}\label{spec:protocol:basics}


The bzz handshake protocol is the protocol that is always run after two peers are connected and before any other protocols are established. It communicates information about the  peer's address, network ID and light node capability.

The handshake protocol defines only one stream and three messages:

\begin{definition}[Bzz handshake protocol  messages]\label{def:bzz-messages}

\begin{lstlisting}
// ID: /swarm/handshake/1.0.0/handshake

syntax = "proto3";

message Syn {
    bytes Address = 1;
    int32 NetworkID = 2;
    bool Light = 3;
}

message SynAck {
    Syn Syn = 1;
    Ack Ack = 2;
}

message Ack {
    bytes Address = 1;
}
\end{lstlisting}
\end{definition}

This message sequence is inspired by the TCP three way handshake to ensure message deliverability.

Upon connection a requesting peer constructs a new handshake stream and sends a \lstinline{Syn} message with its Overlay address, Network ID and Lightnode capability flag, and waits for \lstinline{SynAck} response message from the responding peer. In  the  \lstinline{SynAck} message responder sends it own \lstinline{Syn} message as well as an acknowledgement with the received Overlay address. After the requesting peer receives the \lstinline{SynAck} message from the responding peer and validates that the received \lstinline{Ack} information in it is correct, it sends an \lstinline{Ack} message itself as a confirmation to the responding peer. The stream is closed by the responding peer after it receives the \lstinline{Ack} message.

The connection must be terminated if network IDs are do not match or if the  exact  order of messages is not followed.

The bzz address is verified and overlay, underlay and signature are extracted.
light is a boolean field indicating whether the node is operating as a light (as opposed to full) node.

After the handshake,  each peer should remember the following data about the other:

\begin{itemize}
    \item the overlay address - used in forwarding  (see  \ref{spec:strategy:forwarding}),
    \item the underlay address - used for dialing, passed to the underlay network protocol when the connectivity driver needs to connect to the peer (see \ref{spec:strategy:connection}),
    \item the bzz address signature - needed by the hive protocol to pass information about the node to other peers (see \ref{spec:protocol:hive}),
    \item whether the peer is a light node.
\end{itemize}

% \subsection{Encapsulation of price information \statusred}



\section{Hive discovery  \statusgreen}\label{spec:protocol:hive}
The \gloss{hive protocol} enables nodes to exchange information about other peers that are relevant to them in order to bootstrap their connectivity  (see \ref{sec:bootstrapping}) . The information communicated are both overlay and underlay addresses of the known remote peers (see  \ref{spec:format:bzzaddress}). The overlay address serves to select peers to achieve the connectivity pattern needed for the desired network topology. Underlay address is needed to establish the peer connections by dialing selected peers.

\subsection{Streams and messages \statusgreen}


The protocol specifies one stream with two messages:

\begin{definition}[Hive protocol messages]\label{def:hive-messages}

\begin{lstlisting}
// /swarm/hive/1.0.0/peers
syntax = "proto3";

package hive;

message Peers {
    repeated BzzAddress peers = 1;
}

message BzzAddress {
    bytes Underlay = 1;
    bytes Signature = 2;
    bytes Overlay = 3;
}

\end{lstlisting}
\end{definition}

During the lifetime of connection, nodes can broadcast newly received peers to their peers. This is done by sending the \lstinline{Peers} message over the \\\lstinline{/swarm/hive/1.0.0/peers} stream.

Upon receiving a peers message, nodes are meant to store the peer information in their \gloss{address book}, i.e., a data structure containing info about peers known to the node. The address book is meant to be used to suggest peers  to a connectivity manager according to a connection strategy (\ref{spec:strategy:connection}) in order to bootstrap kademlia topology%
% (\ref{sec:kademlia-connectivity})%
. The address book is meant to be persisted across sessions.

\subsubsection{Sending side}

A stream with the appropriate id is created and a \lstinline{Peers} message is sent over the stream. There is no response to this message. The sending node should wait for the receiving side to close its side of the stream, before closing the stream themselves and moving on.

\subsubsection{Receiving side}

When the stream is created, receiving node should wait for a \lstinline{Peers} message. After receiving the message, node should close its side of the stream to let the sender node know that the message was received, and move on with processing. If the new node was not known, it should also be forwarded to all connected peers closer to peer address then the node themselves.

\section{Retrieval  \statusorange}\label{spec:protocol:retrieval}
\subsection{Simple Summary}
Standardisation of the bzz-retrieve protocol which allows for nodes to issue requests for content addressed data and for their counterparts to deliver that content

\subsection{Abstract}
Every node that fully participates in the Swarm is expected to request, deliver or forward requests for chunk retrievals. This is to be done using a simple p2p protocol that negotiates the delivery of those data chunks.

\subsection{Motivation}
Different Swarm implementations must standardise the way chunk delivery is made. In order for the devp2p protocols on these nodes to correctly negotiate and establish normal operations, the underlying protocol messages and exchange rules must be thoroughly defined. Doing so should pave the way for a simplified reimplementation of Swarm in other clients and languages.

\subsection{Specification}
<!--The technical specification should describe the syntax and semantics of any new feature. The specification should be detailed enough to allow competing, interoperable implementations for the current Swarm platform and future client implementations.-->
The Swarm retrieval protocol is defined as a `devp2p` protocol with the following parameters:
Protocol name:    bzz-retrieve
Current version:  2
Max Msg Size:     10 * 1024 * 1024

Types of nodes and their participation in the retrieval protocol:
1. Bootnode - does not participate in the `bzz-retrieve` protocol
2. Light node - participates but does not handle retrieve requests (handles ChunkDelivery message and issues RetrieveRequest messages)
3. Full node - participates fully (issues and handles both RetrieveRequest and ChunkDelivery messages)

Protocol Messages:
1. ChunkDelivery
2. RetrieveRequest

The protocol messages are defined in the following manner:

1. RetrieveRequest is the message used to issue a request for a single chunk over the Swarm. When a chunk is not found locally on the node handling the message from the requester, that node is expected to send another
RetrieveRequest to the node it sees fit in order to get the chunk. In turn, if that node does not find the chunk, another RetrieveRequest is expected to happen and so forth until the chunk is found.
Once a chunk is found, it is sent back to the requesting node which in turn forwards the chunk back to the node which requested it etc, until the whole cascade of requests is resolved with the delivery of the
wanted chunk to the original requester of the chunk.

The definition of the RetrieveRequest message:
```go
type RetrieveRequest struct {
	Ruid uint
	Addr storage.Address
}
```

The RetrieveRequest defines two fields:
a. Ruid - a unique, random `uint32` generated on the requesting node to associate a certain requested chunk with a node
b. Addr - a 32-byte chunk hash to retrieve

2. ChunkDelivery is used in order to deliver chunks over the Swarm. It is the cornerstone of every bit that should be delivered over Swarm. Each ChunkDelivery message represents a delivery of one discrete chunk.
A ChunkDelivery must always be preceded by a RetrieveRequest message. This mitigates the risk of misbehaving nodes spamming other nodes with unwanted content, causing their local storage to be
filled with unsolicited content.

The definition of the ChunkDelivery message:
```go
type ChunkDelivery struct {
	Ruid  uint
	Addr  storage.Address
	SData []byte
}
```

The ChunkDelivery message defines 3 fields:
a. Ruid - a unique, random `uint32` that corresponds to the Ruid of the incoming RetrieveRequest message for which the ChunkDelivery was sent.
b. Addr - the 32-byte chunk address of the delivered chunk
c. SData - an arbitrary byte array containing the chunk data


Notes:
1. The requester node \textit{MUST} check that a Ruid on an incoming ChunkDelivery message exists for the specific peer from which the chunk was delivered. If the given Ruid cannot be found - that peer should be treated as a misbehaving node and should be dropped
2. The requester node \textit{MUST} check that the address on an incoming ChunkDelivery is identical to the requested chunk address that is associated with the request's unique identifier (Ruid). If the chunk addresses do not match - that node should be treated as a misbehaving node and should be dropped
3. The requester node \textit{MUST} verify that the content of the delivered chunk, after hashing, corresponds to the requested chunk address. If the hashes do not match - the node from which the chunk delivery was received should be treated as a misbehaving node and should be dropped.



\section{Push-syncing  \statusorange}\label{spec:protocol:push-sync}

\begin{definition}{Push-sync protocol messages}\label{def:push-syncc-messages}

\begin{lstlisting}[]
// /swarm/push-sync/1.0.0/peers
syntax = "proto3";

message Delivery {
  bytes Address = 1;
  bytes Data = 2;
}

message Receipt {
  bytes Address = 1;
}
\end{lstlisting}
\end{definition}


\section{Pull-syncing \statusorange}\label{spec:protocol:pull-sync}
purpose of pull syncing

neighbourhood syncronisation

maximum resource utilisation strategy



\section{SWAP settlement protocol \statusorange}\label{spec:protocol:swap}


\chapter{Strategies \statusorange}\label{spec:strategy}


The strategies clients follow will have a fundamental effect on the behaviour of the network and if they end up going against the preconceived design, the project may very well fail to suit user expectations. Such scenarios can easily turn fatal to the project, which is why it is instructive to err on the side of caution when change (or initially suggest) strategies for node behaviour.
The scope of strategies are defined as those aspects of the intended 'protocol' which cannot be directly observed or easily verified. As an example contrast the very act of forwarding an incoming retrieve request (strategy) with the act of using correctly formatted and serialised messages when doing so (protocol constraint). The latter can be immediately detected and be responded to by disconnecting and blacklisting offenders. In contrast, whether a node does forwarding in a way conducive to the desired network outcome is subtle to detect. 

Since we choose to work with the narrowest possible assumption of profit-maximising node operators, the choice of strategies ought to be course grained enough to evaluate the consequences of the options. In particular we must not allow for scenarios when even vaguely rational deviations from the recommended strategy have catastrophic effects.

In general incentives should be in place to guarantee that behaviour that is detrimental to the service incurs a risk of subsequent loss, deterrent upfront cost or opens up reciprocal  vulnerability. 

The constraints put on strategic choice are crucial in terms of rendering the game theory feasible to simulations and experiments or express them with simple enough analytical models to aid reasoning. 

\section{Connection  \statusorange}\label{spec:strategy:connection}
%
All new (previously not known) peers found in the \lstinline{Peers} message received from either of the two streams as well as newly connected peers that dialled in should be automatically broadcast to all connected peers that are closer  to them than the node's base overlay address. Formally, 
node $s$ (sender) notifies an existing peer $r$ (recipient) about peer $p$ if $\mathit{PO}(s, r) = \mathit{PO}(s, p)$. 

\subsection{Connectivity and its constraints}

Without resource constraints the optimal connectivity for a proper storer node is full  connectivity, i.e. when a node has all other nodes in the network as their peer. The chequebook contract also prefers a small number of peer connections to be maintained and other network overhead costs for keepalive connections or network socket shortage also implicates that 
above a certain network size full connectivity is prohibitive.

The relative gain is maximised if throughput on all keepalive connections are maximised. If network contention prevents a node from forwarding to a peer insantaneously, another peer must to be chosen. 
Assuming uniformity in the network 
%(see appendix \ref{sec:distribution}) 
throughput maximisation in the context of limited number of peer connections can be achieved with a connectivity pattern where each kademlia bin has a constant cardinality of $2^b$ and peers in the bin are balanced, i.e., match each distinct bit prefix of length $b$.
Therefore, connectivity strategy can be formulated as follows:







\subsection{Light node connection strategy}

Light nodes in the context of connection topology are nodes that do not have kademlia connectivity. In the extreme case, a light node should be able to get away with a single connection. The lack of full connectivity is indicated to the node's peers so that no retrieve requests or  push sync requests are sent to them. Note that if a node falsely indicates its status, that should cause minimal disruption. 

A lightnode wants to identify its neighboorhood so that it connects to at least one of the $R$ storer nodes closest to it (where $R$ is the redundancy parameter determining the minimum neighbourhood size).





\section{Forwarding  \statusorange}\label{spec:strategy:forwarding}





\subsubsection{Routing with full address} \label{sec:routing-with-full-address}

By full address we mean that the recipient address contains the full 32 bytes of the intended recipient node.

\begin{enumerate}
\item If the recipient address falls within the \emph{most proximate bin} of the node, the content should be forwarded to \emph{all} the node's \emph{nearest neighbors}.
\item Otherwise the node \emph{MUST} attempt to send to exactly \emph{one} peer in the bin that is the closest match to the recipient.
\item If this is not successful, the node \emph{MUST} try to forward to each remaining peer in that bin in order from closest to farthest to the recipient, until one is successful.
\item If this is not successful, the node should retain the content and retry later.
\end{enumerate}

\subsubsection{Routing with partial address}

\begin{enumerate}
\item Identify the \emph{farthest} peer that matches the partial address.
\item If that peer is in the \emph{most proximate bin}, the content should be forwarded to \emph{all} the node's \emph{nearest neighbors}.
\item Otherwise, if the bitlength of the recipient address equals the proximity order of its matching bin, the content should be forwarded to \emph{all} peers that are in the matching bin or closer to the recipient\footnote{Notice that in this context, the matching bin becomes the \emph{most proximate bin}}.
\item Otherwise the bitlength of the recipient address is \emph{longer} than the promixity order of its matching bin. In this case the node \emph{MUST} attempt to send to exactly \emph{one} peer in the bin that is the closest match to the recipient address.
\item If this is not successful, the node \emph{MUST} try to forward to each remaining peer in that bin in order from closest to farthest to the recipient address, until one is successful.
\item If this is not successful, the node should retain the content and retry later.
\end{enumerate}

\subsubsection{Routing with empty address}

\begin{enumerate}
\item Content should be forwarded to \emph{all} peers.
\item If \emph{none} of the peers can successfully be forwarded to, the node should retain the content and retry later.
\end{enumerate}

\subsection{Evaluating recipients}

Any node receiving content has to evaluate whether it can be the intended recipient, or one of the indended recipients, of the content.

The first condition that must be fulfilled is a matching recipient address on the content. Matching may happen in one of two forms:

\paragraph{Literal matching}

As described in \ref{sec:routing-with-full-address}, if the full 32-byte recipient address matches the node address, the node is the only intended recipient and \emph{MUST} process the content.

If the recipient address is a partial address, and it matches the node address, the node \emph{may} be one of the intended recipients and \emph{MUST} process the content.

\paragraph{Proximity matching}

This method provides a way to send content with a full 32 byte recipient address to more than one recipient.\footnote{This is the matching method used for the \emph{Push Sync} feature, where content addressed chunks are routed to their neighborhood in the network.} The procedure is as follows.

\begin{enumerate}
\item The proximity order of the recipient address with respect to the node is calculated
\item If the proximity order is within the node's depth, the node \emph{may} be one of the intended recipient and \emph{MUST} process the content.
\end{enumerate}




1. Get rid of `hopcount` and make sure we have forwarding of chunk requests without loops in a DAG
2. Add somehow retry mechanism for chunks in the syncer, possibly through the `FetcherItem` and the `NetStore` retry mechanism used for RetrieveRequests
3. Pass information on who has offered a given chunk from the syncer to the fetchers.
4. Extract all logic that selects the `next` peer at one place - this decision is based on:
    - Kademlia
    - historic information of the Fetcher (such as PeerToSkip ; who has been tried already)
    - syncer information - who has offered a given chunk
    - path info - minimally, who requested the chunk

In particular, 4 is achieved by finding the closest peer to the chunk address such that
    
    
- NEW: 
    - it has not yet been selected earlier by the fetcher
    
- KNOWN SOURCE:
    - peer that offered the chunk via syncing, such a peer is a reat candidate as it guarantees the chunk in one hop
    
- ON ROUTE: 
    - if the chunk falls outside of the area of responsibility, it strictly increases proximity to the chunk (in other words, we choose from the same PO bin the chunk falls into). 
    - if the chunk falls within our area of responsibility, then choose a nearest neighbour - ideally all the nearest neighbours at once - note that there are gonna be nearest neighbours further from the chunk than us

- REVERSE: if the request comes from a node not strictly further from the chunk, then
  - do not select any peer and forward to noone
      this can happen 
      - if we are the requesting peer's nearest neighbours:
        we don't have the chunk and all other neighbours are asked, so makes no sense to forward
      - we offered the hash via syncing to the requesting node. If we get here, it means we deleted a chunk we offered or the  requesor 
      
      This algotithm is strict in that the forwarding path created by nodes following these rules have the properties explicitly used in the paper's claim to logN steps to get the chunk from anywhere. 
      In particular it is guaranteed to be without loops. It is guaranteed to succeed with the sole assumption of kademlia health. However, modifications will need to be made to take care of the situations with unhealthy kademlia.

%peer selection and pricing
\section{Pricing  \statusorange}\label{spec:strategy:pricing}
In the following we describe 3 strategies: each is more complex than the previous.

passive: weak cartell - hard wired fix price table
minimal responsiveness: if margin is not guaranteed, the peer is not forwarded to. 

reactive - respond to downstream price increase with exploring alternative peers. If there is no alternative it raises the price. If downstream peers drop the price, it follows suit.

proactive: respond to 


\subsection{Passive  strategy}\label{sec:pricing-retrieval}
The following strategy is a basic way on how nodes *can* use the protocol described above. The strategy intends to be as easy as possible to implement, while balbalbala

## Margin
A nodes price for a `ChunkDeliveryRequest` is always at least his desired `margin` + the price of the cheapest upstream peer. 

## Filling the priceInformationRegistry
Initially, nodes only know the price for serving content from the most proximate bin (price == `margin`). When this node gets his first request for serving content that is one proximity order away from the border of his most proximate bin, he will first of all check that the price for this request is more than his margin. If this is the case, he will forward the `RetrieveRequest` with a price of 0 (since the state is not initialized yet). If the upstream peer in the appropriate bin did not change his price, this peer will get back a `NewPrice` message, where the price equals the `margin` of the downstream peer. If the original price is at least 2x the `margin`, the node can forward the request with a price of `margin` (and pocket the other marin himself). From this point onwards, the `state` is initialized for one peer and one proximity. Subsequent `RetrieveRequests` to this peer will be send with a `price` of `margin`. Chunks which are further away will have a higher price. The peer will quickly learn about this via this iterative process. If all peers apply the same `margin`, the price of a `ChunkDelivery` equals the proximity order between the requestor and the chunk times the `margin`. If a peer get's a request for a chunk with a price that is less than his `margin` + the price of the cheapest upstream peer in that bin, the request is not forwarded, but immediately answered by a `NewPrice` message with a price equal to the `margin` + the price of the cheapest upstream peer in that bin.

## Reacting to price differences
In a network where no peer ever charges a `margin` different than the default margin, `NewPrice` messages will be only send to peers who have not fully initialized their `state` and, given equal proximity between the requested chunk and the peer in a particular bin, a node does not have a preferred supplier of a chunk based on the price. If any peer in the network would change his price, however, a preference based on price will emerge:

For any chunk request:
- Select the bin to forward the request to
- Select the peers who are closest to the chunk in terms of PO
- Select the peer with the lowest price
- Do a round-robin load balancing if there are multiple peers with the same price

From this, we can conclude that if a peer lowers his price, he is expected to receive more `RetrieveRequests` and if a peer increases his price, he is expected to receive fewer `RetrieveRequests`. A node who is connected to an upstream peer with a lower price may decide to lower his price for this distance as well in order to receive more requests.

## Notifying about price changes
It is in the interest of the network, that when there is a price change this gets propagated as soon as possible to the relevant parties. A price change can be initiated by a change of `margin`, or by a change of the price from upstream peers. In both cases, a peer might decide not to notify other peers about the decrease in price--the only thing which changes is that it will accept `RetrieveRequests` with a lower price than before. Without notifying the downstream peers, however, a node might not get the expected amount of additional clients directly, as the downstream peers might never propose a lower price than before. For this reason, we propose that nodes pro-actively send `NewPrice` messages to his peers, essentially requesting them to update their `priceInformationRegistry`. In the case of pro-active `NewPrice` messages, the `chunkReference` may be synthetic, meaning it does not have to respond to an actual chunk; all that is important is that peers update their `priceInformationRegistry`.

### Overpriced bins
Since the price of a node depends on the price of his upstream peers, it might happen that by change, all his upstream peer-connections in a certain bin are populated by expensive peers—effectively making the node itself to be too expensive as well for his downstream peers. To solve this issue, we propose to do statistical analysis: if a node has sensible prices across his bins, it is expected that he receives an equal amount of requests for all his bins. If a node receives statistically significantly less requests for one bin compared to his other bins, he marks the supposedly overpriced peers as non-functional and the hive protocol should kick in to suggest new peer connections. 

### Reacting to demand increases/decreases

A node that decreases its price such that he is the cheapest, is expected to get much more traffic. Without an adequate response to an increase in traffic, a node might be forced to go offline as D

## Configuration options
The following configuration options will be added. 

| name | unit | default |
| -------- | -------- | -------- |
| `margin`     | base currency of chequebook     | 0     |
| `period`      | duration in seconds    | 300     |
| `maximum_upstream_bandwidth`      | bandwidth usage in bytes per `period`     | TODO     |
|`high_water_mark`    | percentage: (upstream bandwidth used / `period`) / (`maximum_upstream_bandwidth`)     | 80%     |
| `low_water_mark`    | percentage: (upstream bandwidth used / `period`) / (`maximum_upstream_bandwidth`)      | 20%     |
| `margin_change`    | percentage    | 1%     |
See below for how these veriables are used.


## Behavior of a node
1) Given a particular `chunkReference`, a node chooses to send the `ChunkRequestMessage`, usually, to the peer with the lowest price of all peers in that bin.
2) If a `newPrice` message is returned from a `chunkRequestMessage`, the node updates the `priceInformationRegistry` and sends the `chunkRequestMessage` again, choosing a peer on the basis of the updated `priceInformationRegistry`. 
4) When a peer requests a certain `ChunkRequestMessage` and does not have the `chunkContent` in its local storage, it forwards the `chunkRequestMessage` (see 1) with a price equal to the `chunkRequestMessage` it received minus his `margin`.
5) A node keeps track of his `bandWidth_usage` ((upstream bandwidth used / `period`) / (`maximum_upstream_bandwidth`))
6) Whenever `bandWidth_usage` > `high_water_mark`, the `margin` will be (1+`margin_change`) * `margin`
7) Whenever `bandWidth_usage` < `low_water_mark`, the `margin` will be (1-`margin_change`) * `margin`. `margin_change` indicates the 
8) Whenever a node can get a `chunk` for a cheaper price than requested by a `ChunkRequestMessage` (upstream peers changed prices, amount of hops are less than the average request for the same distance), he answers with a `ChunkDeliveryMessage`, which will be priced based on the `ChunkRequestMessage`. 
9) Whenever a node expects to be able to deliver long-term lower prices than currently known by his peers, he sends a `NewPrice` message to his peers. 


\section{Accounting and settlement  \statusorange}\label{spec:strategy:swap}

payment threshold

disconnect threshold

receiving cheque

cashing strategy

chequebook balance 

topup strategy



\section{Push-syncing  \statusorange}\label{spec:strategy:push-sync}

\begin{definition}{Push-sync protocol messages}\label{def:push-syncc-messages}

\begin{lstlisting}[]
// /swarm/push-sync/1.0.0/peers
syntax = "proto3";

message Delivery {
  bytes Address = 1;
  bytes Data = 2;
}

message Receipt {
  bytes Address = 1;
}
\end{lstlisting}
\end{definition}


\section{Pull-syncing  \statusorange}\label{spec:strategy:pull-sync}
purpose of pull syncing

neighbourhood syncronisation

maximum resource utilisation strategy



\section{Garbage collection \statusorange}\label{spec:strategy:garbage-collection}

Let's  index  the epochs with negative integers with an ordering respecting recency, i.e., the current epoch is 0, the most recent $-1$, the previous one $-2$, etc. 
Let $n$ be the cutoff memory size for past epochs.
Let $\mathit{Hits}(e)$ denote the number of hits a chunk is served during epoch $e$. 

Let's build a popularity predictive model based on past observations using the assumption of exponential decay with time in predictive power. 

\begin{equation}
    \mathit{Hits}(0) \defeq \frac{\sum_{e=0}^n \frac{\mathit{Hits}(e)}{c^e}}{c^{n-1}}
\end{equation}

\chapter{API-s}\label{spec:api}


\begin{definition}[HTTP status codes used in swarm]
\begin{lstlisting}
\end{lstlisting}
\begin{tabular}{c|p{0.25\linewidth}|p{0.6\linewidth}}
103 & Checkpoint & returns temporary root hash for resumable uploads
\\\hline
200 & ok &
\\
201 & Created & returned by POST requests upon successful creation of file/manifest/tag/stamp/
\\
206 & Partial Content &
\\\hline
400 & Bad request & returned if request or its parameters are not well formed or missing
\\
401 & Unauthorized & returned by access control if auth fails
\\
402 & Payment Required & returned if no swap balance or missing/invalid postage stamp
\\
403 & Forbidden & returned if retrieved chunk is encrypted  but reference is. missing decryption key
\\
404 & Not Found &
returned if a local prerequisite is not found
\\
405 & Method Not Allowed & HTTP verb  not allowed for this endpoint
\\
406 & Not Acceptable &
the Accept header explicit in the request.
\\
408 & Request Timeout & retrieve requests fallback error after TTL passed
\\
411 & Length Required & returned by chunk upload API if length of file uploaded is beyond limit
% 412 & Precondition Failed (RFC 7232)
\\
413 & Payload Too Large  &
length of payload to upload beyond (chunk API)
\\
414 & URI Too Long  & manifest path  $\leq $ 32
\\
416 & Range Not Satisfiable  & offset in range query out of range
\\
420 & Enhance your calm & returned when recovery was initiated but retrieval timed out
\\
422 & Unprocessable Entity & returned by the blockchain external API if eth api returns an error 
% 417 & Expectation Failed
% 421 & Misdirected Request (RFC 7540)
% 451 & Unavailable For Legal Reasons (RFC 7725)
\end{tabular}
\end{definition}


\section{External API requirements\statusorange}\label{spec:api:external}

\subsection{Signer}\label{spec:api:signer}
% \input{specs/api/signer.tex}

% \UseRawInputEncoding
\begin{apiRoute}{GET}{/sign/\param{id}/\param{document}}{ECDSA signature}
 {
}
{ }
\begin{routeParameter} 
\routeParamItem{id}{hex string - eth address}
\routeParamItem{document}{(hex) string -  document to sign (is prefixed and hashed before signing)}
\end{routeParameter} \begin{routeResponse}{application/json}
\responseItem{200}{OK}{}
\responseItem{401}{Unauthorised}{failed authentication on existing identity} 
\responseItem{404}{Not Found}{unknown identity} 
\end{routeResponse} 
\end{apiRoute}

\begin{apiRoute}{GET}{/dh/\param{id}/\param{pubkey}}{Diffie-Hellman shared secret}
 {
}
{ }
\begin{routeParameter} \routeParamItem{id}{hex string - eth address}
\routeParamItem{pubkey}{hex string - represents the remote party in the shared secret arrangement}
\end{routeParameter} \begin{routeResponse}{application/json}
\responseItem{200}{OK}{} 
\responseItem{401}{Unauthorised}{failed authentication on existing identity}
\responseItem{404}{Not Found}{unknown identity}
\end{routeResponse} \end{apiRoute}


\subsection{Blockchain \statusgreen}\label{spec:api:blockchain}
% \input{specs/api/blockchain.tex}

\begin{apiRoute}{GET}{/eth/\param{contract}/\param{function}/\param{args}}
{ethereum API call}
 {
}
{ }
\begin{routeParameter}
\routeParamItem{contract}{hex string, eth address of contract}
\routeParamItem{function}{endpoint within contract}
\routeParamItem{args}{arguments for the eth API call}
\end{routeParameter} \begin{routeResponse}{application/json}
\responseItem{200}{ok}{}  \responseItem{400}{Bad request}{unknown contract or function endpoint given} 
\responseItem{404}{Not Found}{unknown contract or function endpoint} \responseItem{422}{Unprocessable entity}{incorrect ABI,  error by eth API}
\end{routeResponse} \end{apiRoute}

\begin{apiRoute}{POST}{/eth/\param{contract}/\param{function}/\param{args}}
{ethereum API send transaction}
 {
}
{ }
\begin{routeParameter}
\routeParamItem{contract}{hex string, eth address of contract}
\routeParamItem{function}{endpoint within contract}
\routeParamItem{args}{arguments for the eth API call}
\end{routeParameter} \begin{routeResponse}{application/json}
\responseItem{200}{OK}{transaction data in response body}
\responseItem{400}{Bad Request}{unknown contract or function endpoint given} 
\responseItem{401}{Unauthorised}{failure signing transaction}
\responseItem{404}{Not Found}{unknown contract or function endpoint} 
\responseItem{422}{Unprocessable entity}{incorrect ABI,  error by eth API}
\end{routeResponse} \end{apiRoute}


\subsection{User input}\label{spec:api:input}


\begin{apiRoute}{GET}{/input/\param{id}}
 {
}
{ }
\begin{routeParameter}
\routeParamItem{id}{hex string - eth address of the persona the input is expected from}
\routeParamItem{note}{the question to be answered or instruction to select}
\end{routeParameter} \begin{routeResponse}{application/json}
\responseItem{200}{ok}{}
\end{routeResponse} \end{apiRoute}



\section{Storage API \statusyellow}\label{spec:api:storage}

All URIs are relative to \emph{http://localhost:8500/bzz/}

\begin{longtable}[]{@{}lll@{}}
\toprule
Method & HTTP request & Description\tabularnewline
\midrule
\endhead
\href{DefaultApi.md\#bzzUploadFile}{\textbf{bzzUploadFile}} &
\textbf{POST} /upload & Swarm Upload API\tabularnewline
\bottomrule
\end{longtable}

\# \textbf{bzzUploadFile} \textgreater{} Object
bzzUploadFile(SWARMPOSTAGESUBSCRIPTION, SWARMTAG, SWARMENCRYPTED,
SWARMPIN, SWARMCRSPARITY, file)

Swarm Upload API

\hypertarget{parameters}{%
\subsubsection{Parameters}\label{parameters}}

\begin{longtable}[]{@{}llll@{}}
\toprule
Name & Type & Description & Notes\tabularnewline
\midrule
\endhead
\textbf{SWARMPOSTAGESUBSCRIPTION} & \textbf{String} & & {[}default to
null{]}\tabularnewline
\textbf{SWARMTAG} & \textbf{String} & & {[}optional{]} {[}default to
null{]}\tabularnewline
\textbf{SWARMENCRYPTED} & \textbf{String} & & {[}optional{]} {[}default
to null{]}\tabularnewline
\textbf{SWARMPIN} & \textbf{Boolean} & & {[}optional{]} {[}default to
null{]}\tabularnewline
\textbf{SWARMCRSPARITY} & \textbf{String} & & {[}optional{]} {[}default
to null{]}\tabularnewline
\textbf{file} & \textbf{List} & & {[}optional{]} {[}default to
null{]}\tabularnewline
\bottomrule
\end{longtable}

\hypertarget{return-type}{%
\subsubsection{Return type}\label{return-type}}

\href{..//Models/object.md}{\textbf{Object}}

\hypertarget{authorization}{%
\subsubsection{Authorization}\label{authorization}}

No authorization required

\hypertarget{http-request-headers}{%
\subsubsection{HTTP request headers}\label{http-request-headers}}

\begin{itemize}
\item
  \textbf{Content-Type}: multipart/form-data, file, application/x-tar
\item
  \textbf{Accept}: application/json, application/problem+json
\end{itemize}%


\subsection{Tags \statusyellow}\label{spec:api:tags}

\begin{apiRoute}{POST}{/tags}{Create new tag}
{
}
{ }

\begin{routeResponse}{application/json}
\responseItem{201}{Created}{ID in response body}
\end{routeResponse}
\end{apiRoute}




\begin{apiRoute}{GET}{/tags?offset=\param{offset}\&length=\param{length}}{Get all tags}
{
}
{ }

\begin{queryParameter} 
\queryParamItem{offset}{integer}
\queryParamItem{length}{integer}
\end{queryParameter}
\begin{routeResponse}{application/json}
\responseItem{200}{OK}{json array of tag details struct}
\end{routeResponse}
\end{apiRoute}




\begin{apiRoute}{GET}{/tags/\param{id}}{View details tag with given ID}
{
}
{ }

\begin{routeParameter} 
\routeParamItem{id}{hex string}
\end{routeParameter}
\begin{routeResponse}{application/json}
\responseItem{200}{OK}{json struct in response body}
\responseItem{400}{Bad Request}{ID not well formed}
\responseItem{404}{Not Found}{tag with ID does not exists.}
\end{routeResponse}
\end{apiRoute}



\begin{apiRoute}{DELETE}{/tags/\param{id}}{Path description}
{
}
{ }

\begin{routeParameter} 
\routeParamItem{id}{hex string}
\end{routeParameter}
\begin{routeResponse}{application/json}
\responseItem{204}{No Content}{successful deletion}
\responseItem{400}{Bad Request}{ID not well formed}
\responseItem{404}{Not Found}{tag with ID does not exists.}

\end{routeResponse}
\end{apiRoute}



\subsection{Pinning
\statusyellow}\label{spec:api:pinning}

\begin{apiRoute}{GET}{/pin/?offset=\param{offset}\&length=\param{length}}{List pinned content and metadata}
{
}
{ }

\begin{queryParameter} 
\queryParamItem{offset}{integer}
\queryParamItem{length}{integer}
\end{queryParameter}
\begin{routeResponse}{application/json}
\responseItem{200}{OK}{json array in response body}
\end{routeResponse}
\end{apiRoute}


\begin{apiRoute}{GET}{/pin/\param{id}}{View pinned content and metadata}
{
}
{ }

\begin{routeParameter} 
\routeParamItem{id}{hex string}
\end{routeParameter}
\begin{routeResponse}{application/json}
\responseItem{200}{OK}{pin in response body}
\responseItem{400}{Bad Request}{ID not well formed}
\responseItem{404}{Not Found}{Pin with ID does not exists.}
\end{routeResponse}
\end{apiRoute}




\begin{apiRoute}{PUT}{/pin/\param{id}}{Pin an already uploaded content}
{
}
{ }

\begin{routeParameter} 
\routeParamItem{id}{hex string}
\end{routeParameter}
\begin{routeResponse}{application/json}
\responseItem{200}{OK}{json struct in response body}
\responseItem{400}{Bad Request}{ID not well formed}
\responseItem{404}{Not Found}{Pin with ID does not exists.}
\end{routeResponse}
\end{apiRoute}




% \begin{apiRoute}{PUT}{/pin/\{id\}}{Path description?}
% {
% }
% { }

% \begin{routeParameter} 
% \routeParamItem{id}{}
% \end{routeParameter}
% \begin{routeResponse}{application/json}
% \begin{routeResponseItem}{200}{Ok}
% \begin{routeResponseItemBody}
  
% \end{routeResponseItemBody}
% \end{routeResponseItem}
% \end{routeResponse}
% \end{apiRoute}




\begin{apiRoute}{DELETE}{/pin/\param{id}}{Remove pinning from content}
{
}
{ }

\begin{routeParameter} 
\routeParamItem{id}{hex string}
\end{routeParameter}
\begin{routeResponse}{application/json}
\responseItem{204}{No Content}{successful deletion}
\responseItem{400}{Bad Request}{ID not well formed}
\responseItem{404}{Not found}{pin with ID does not exists}
\end{routeResponse}
\end{apiRoute}



\subsection{Swap and chequebook\statusorange}\label{spec:api:swap}
% Agents
\def\IssuerLocal{A client}
\def\IssuerSwapContract{A swap}
\def\BeneficiarySwapContract{B swap}
\def\BeneficiaryLocal{B client}

% Message Flows
\def\Issue{issue cheque}% \def\Cheque{Cheque}
\def\Redeem{redeem cheque} %\def\Cheque{Cheque}
\def\Clear{clear ETH} %\def\ETH{ETH}
\def\NW{withdrawal event} %\def\Msg{Log Event}
\def\ND{deposit event} %\def\Msg{Log Event}

% Legend 
\def\LegendOnChain{On-chain}
\def\LegendOffChain{Off-chain}


\begin{tikzpicture}[every node/.style={font=\small,
  minimum height=.35cm,minimum width=.5cm},]

%
% Matrix
\node [matrix, very thin,column sep=1.0cm,row sep=0.2cm] (matrix) at (0,0) {
  & \node(0,0) (\IssuerLocal) {};   &                         & \node(0,0) (\IssuerSwapContract) {};   & & \node(0,0) (\BeneficiarySwapContract) {};   & & \node(0,0) (\BeneficiaryLocal) {}; \\
  & & & & & & & \\
  & & & & & & & \\
  & & & & & & & \\
  & \node(0,0) (\IssuerLocal 1) {}; & \node(0,0) (\Issue) {}; & \node(0,0) (\IssuerSwapContract 1) {}; & & \node(0,0) (\BeneficiarySwapContract 1) {}; & & \node(0,0) (\BeneficiaryLocal 1) {}; \\
  & & & & & & & \\
  & & & & & & & \\
  & \node(0,0) (\IssuerLocal 2) {}; &                         & \node(0,0) (\IssuerSwapContract 2) {}; & & \node(0,0) (\BeneficiarySwapContract 2) {}; & \node(0,0) (\Redeem) {};  &  \node(0,0) (\BeneficiaryLocal 2) {}; \\ 
  & & & & & & & \\
  & & & & & & & \\
  & \node(0,0) (\IssuerLocal 3) {}; &                         & \node(0,0) (\IssuerSwapContract 3) {}; & \node(0,0) (\Clear) {}; & \node(0,0) (\BeneficiarySwapContract 3) {}; &                    ; & \node(0,0) (\BeneficiaryLocal 3) {}; \\
  & & & & & & & \\
  & \node(0,0) (\IssuerLocal 4) {}; & \node(0,0) (\NW) {}   ; & \node(0,0) (\IssuerSwapContract 4) {}; &                         & \node(0,0) (\BeneficiarySwapContract 4) {}; & \node(0,0) (\ND) {}; & \node(0,0) (\BeneficiaryLocal 4) {}; \\
  & & & & & & & \\
  & \node(0,0) (\IssuerLocal 5) {}; &                         & \node(0,0) (\IssuerSwapContract 5) {};  & & \node(0,0) (\BeneficiarySwapContract 5) {};& & \node(0,0) (\BeneficiaryLocal 5) {}; \\
  & & & & & & & \\
  & \node(0,0) (\IssuerLocal 6) {}; &                         & \node(0,0) (\IssuerSwapContract 6) {};  & & \node(0,0) (\BeneficiarySwapContract 6) {};& & \node(0,0) (\BeneficiaryLocal 6) {}; \\
  & & & & & & & \\
  & \node(0,0) (\IssuerLocal 7) {}; & \node(0,0) (\LegendOnChain) {};  & & & & & \\
  & \node(0,0) (\IssuerLocal 8) {}; & \node(0,0) (\LegendOffChain) {}; & & & & & \\
};

% Agents labels
\fill 
	(\IssuerLocal) node[draw,fill=white] {\IssuerLocal}
	(\IssuerSwapContract) node[draw,fill=white] {\IssuerSwapContract}
	(\BeneficiarySwapContract) node[draw,fill=white] {\BeneficiarySwapContract}
	(\BeneficiaryLocal) node[draw,fill=white] {\BeneficiaryLocal};

\draw [dashed] 
  (\IssuerLocal) -- (\IssuerLocal 6)
  (\BeneficiaryLocal) -- (\BeneficiaryLocal 6)
  (\IssuerSwapContract) -- (\IssuerSwapContract 6)
  (\BeneficiarySwapContract) -- (\BeneficiarySwapContract 6);

% Horizontal flows (Monetary interactions)
%\draw [-latex] (\IssuerLocal 1) -- (\IssuerSwapContract 1.west) arc(180:0:0.37cm) -- (\BeneficiarySwapContract 1.west) arc(180:0:0.37cm) -- (\BeneficiaryLocal 1);
\draw [-{Latex[length=1.5mm,width=2.5mm]}] (\IssuerLocal 1) -- (\BeneficiaryLocal 1);
\draw [-{Latex[length=1.5mm,width=2.5mm]}] (\BeneficiaryLocal 2) -- (\IssuerSwapContract 2);
%\draw [-latex] (\BeneficiaryLocal 2) -- (\BeneficiarySwapContract 2.east) arc(0:180:0.37cm) -- (\IssuerSwapContract 2);
\draw [-{Latex[length=1.5mm,width=2.5mm]}] (\IssuerSwapContract 3) -- (\BeneficiarySwapContract 3);
\draw [-{Latex[length=1.5mm,width=2.5mm]}] (\IssuerSwapContract 3) -- (\IssuerLocal 4);
\draw [-{Latex[length=1.5mm,width=2.5mm]}] (\BeneficiarySwapContract 3) -- (\BeneficiaryLocal 5);

% Flows Labels 
\fill
  (\Issue) 
    node[above, font=\footnotesize ] {\Issue}
  (\Redeem) 
    node[above, font=\footnotesize] {\Redeem}
  (\Clear) 
    node[above, font=\footnotesize] {\Clear}
  (\NW) 
    node[above, font=\footnotesize, text width=1.4cm, text height=1.5cm, align=center,fill=white] {\NW}
  (\ND) 
    node[above, font=\footnotesize, text width=1.5cm,align=center,fill=white] {\ND};

% Interaction points 
\draw 
  (\IssuerLocal 1) node[minimum size=0.25cm, draw,circle,fill=red!20] {}
  (\BeneficiaryLocal 1) node[minimum size=0.25cm, draw,circle,fill=red!20] {}
  (\BeneficiaryLocal 2) node[minimum size=0.25cm, draw,circle,fill=red!20] {}
  (\IssuerSwapContract 2) node[minimum size=0.25cm, draw,circle,fill=green!20] {}
  (\IssuerSwapContract 3) node[minimum size=0.25cm, draw,circle,fill=green!20] {}
  (\IssuerLocal 4) node[minimum size=0.25cm, draw,circle,fill=red!20] {}
  (\BeneficiarySwapContract 3) node[minimum size=0.25cm, draw,circle,fill=green!20] {}
  (\BeneficiaryLocal 5) node[minimum size=0.25cm, draw,circle,fill=red!20] {}
  (\IssuerLocal 7) node[minimum height=.1cm, minimum size=0.1cm, draw,circle,fill=green!20] {}
  (\IssuerLocal 8) node[minimum height=.1cm, minimum size=0.1cm, draw,circle,fill=red!20] {};

% Vertical lifelines
\draw [-{Latex[length=1.5mm,width=2mm]}] (\IssuerSwapContract 2) -- (\IssuerSwapContract 3);
 Legend labels
\draw
	(\LegendOnChain) node[minimum height=.1cm] {\LegendOnChain}
	(\LegendOffChain) node[minimum height=.1cm] {\LegendOffChain};
\end{tikzpicture}


\subsection{Postage stamps \statusorange}\label{spec:api:postage}

\begin{apiRoute}{GET}{/stamp}{View all postage stamps}
{
}
{ }

\begin{routeResponse}{application/json}
\responseItem{200}{ok}{}
\end{routeResponse}
\end{apiRoute}




\begin{apiRoute}{GET}{/stamp/\param{id}}{View postage stamp with id}
{
}
{ }

\begin{routeParameter} 
\routeParamItem{id}{}
\end{routeParameter}
\begin{routeResponse}{application/json}
\responseItem{200}{ok}{}
\end{routeResponse}
\end{apiRoute}




\begin{apiRoute}{PUT}{/stamp/\param{id}}{Top up postage stamp with id}
{
}
{ }

\begin{routeParameter} 
\routeParamItem{id}{}
\end{routeParameter}
\begin{routeResponse}{application/json}
\responseItem{200}{ok}{}
\end{routeResponse}
\end{apiRoute}



\begin{apiRoute}{DELETE}{/stamp/{id}}{Drain and expire stamp with id}
{
}
{ }

\begin{routeParameter} 
\routeParamItem{id}{}
\end{routeParameter}
\begin{routeResponse}{application/json}
\responseItem{200}{ok}{}
\end{routeResponse}
\end{apiRoute}


\begin{apiRoute}{POST}{/stamp/{id}}{Create a new postage stamp}
{
}
{ }

\begin{routeParameter} 
\routeParamItem{id}{}
\end{routeParameter}
\begin{routeResponse}{application/json}
\responseItem{200}{ok}{}
\end{routeResponse}
\end{apiRoute}



\subsection{Access Control  \statusgreen}\label{spec:api:access-control}

\begin{apiRoute}{POST}{/access/\param{address} }{Lock ACT for address as in \ref{def:ac-api}}
{
}
{ }

\begin{routeParameter} 
\routeParamItem{address}{hex string}
\end{routeParameter}
\begin{routeResponse}{application/json}
\responseItem{201}{Created}{root access manifest reference in response body}
\responseItem{400}{Bad Request}{Encrypted content but no decryption key in reference}
\responseItem{403}{Forbidden}{Encrypted content but no decryption key in reference}
\responseItem{404}{Not found}{}
\responseItem{408}{Request Timeout}{Timeout retrieving referenced manifest}
\responseItem{420}{Enhance your calm}{Recovery initiated but request timed out}
\end{routeResponse}
\end{apiRoute}



\begin{apiRoute}{GET}{/access/\param{address} }{Unlock ACT for address as in \ref{def:ac-api}}
{
}
{ }

\begin{routeParameter} 
\routeParamItem{address}{hex string}
\end{routeParameter}
\begin{routeResponse}{application/json}
\responseItem{200}{ok}{}
\responseItem{400}{Bad Request}{Address not well formed}
\responseItem{401}{Unauthorized}{Access denied: AC unlock failed}
\responseItem{403}{Forbidden}{Encrypted content but no decryption key in reference}
\responseItem{408}{Request Timeout}{Timeout retrieving referenced manifest}
\responseItem{420}{Enhance your calm}{Recovery initiated but request timed out}\end{routeResponse}
\end{apiRoute}




\begin{apiRoute}{PUT}{/access/\param{root}/\param{pubkey}}{Add entry for pubkey to the  ACT referred in the root access manifest \ref{def:act-api}}
{
}
{ }

\begin{routeParameter} 
\routeParamItem{root}{hex string - reference to root access manifest}
\routeParamItem{pubkey}{hex string - public key of grantee}
\end{routeParameter}
\begin{routeResponse}{application/json}
\responseItem{201}{Created}{Reference to new manifest root in response body}
\responseItem{400}{Bad Request}{Address or public key not well formed}
\responseItem{401}{Unauthorized}{Permission denied: creating session key failed}
\responseItem{403}{Forbidden}{Encrypted content but no decryption key in reference}
\responseItem{408}{Request Timeout}{Timeout retrieving referenced manifest}
\responseItem{420}{Enhance your calm}{Recovery initiated but request timed out}
\end{routeResponse}
\end{apiRoute}


\begin{apiRoute}{DELETE}{/access/\param{root}/\param{pubkey}}{Remove entry for pubkey from ACT referred in the root access manifest, see \ref{def:act-api}}
{
}
{ }

\begin{routeParameter} 
\routeParamItem{root}{hex string - reference to root access manifest}
\routeParamItem{pubkey}{hex string - public key of grantee}
\end{routeParameter}
\begin{routeResponse}{application/json}
\responseItem{201}{Created}{Reference to new manifest root in response body}
\responseItem{400}{Bad Request}{Address or public key not well formed}
\responseItem{401}{Unauthorized}{Permission denied: creating session key failed}
\responseItem{403}{Forbidden}{Encrypted content but no decryption key in reference}
\responseItem{408}{Request Timeout}{Timeout retrieving referenced manifest}
\responseItem{420}{Enhance your calm}{Recovery initiated but request timed out}
\end{routeResponse}
\end{apiRoute}



% \subsection{Recovery\statusorange}\label{spec:api:recovery}

\section{Communications  \statusorange}\label{spec:api:communications}
\input{specs/api/comms.tex}

\subsection{PSS \statusyellow}\label{spec:api:trojan}

\begin{apiRoute}{POST}{/pss/send/\param{topic}(?targets=\param{targets}\&recipient=\param{recipient})}{Send private message with topic to targets, encrypted for recipient if public key given. If public key parameter missing, the message is encrypted using the topic as as the key for the asymmetric encryption so that whoever knows and expects messages on the topic can read,  see \ref{def:send}}. 
{
}
{ }

\begin{routeParameter} 
\routeParamItem{topic}{string}
\end{routeParameter}
\begin{queryParameter} 
\queryParamItem{recipient}{hex string - recipient public key for encryption}
\queryParamItem{targets}{hex string - comma separated list of targets, these correspond to alternative target overlays for the same recipient and the first mined trojen chunk  matching any  target will be sent.} 
\end{queryParameter} 
\begin{headerParameter} 
\headerParamItem{SWARM-TAG}{hex string, to monitor delivery status}
\headerParamItem{SWARM-STAMP}{hex string}
\end{headerParameter}
\begin{requestBody}
the message payload plaintext
\end{requestBody}
\begin{routeResponse}{application/json}
\responseItem{209}{Sent}{Tag ID to monitor delivery with}
\responseItem{400}{Bad Request}{Topic, targets or recipient not well formed.}
\end{routeResponse}
\end{apiRoute}




\begin{apiRoute}{POST}{/pss/subscribe/\param{topic}/(?on=\param{channel})}{Subscribe to messages with topic to be delivered on given channel, see \ref{def:receive}}
{
}
{ }

\begin{routeParameter} 
\routeParamItem{topic}{string}
\end{routeParameter}
\begin{queryParameter} 
\queryParamItem{on}{hex string - channel ID}
\end{queryParameter} \begin{routeResponse}{application/json}
\responseItem{201}{Created}{}
\responseItem{400}{Bad Request}{topic or channel not well formed.}
\end{routeResponse}
\end{apiRoute}

 
\begin{apiRoute}{DELETE}{/pss/subscribe/\param{topic}/(?on=\param{channel})}{Unsubscribe for topic on channel, see \ref{def:receive}}
{
}
{ }

\begin{routeParameter} 
\routeParamItem{topic}{string}
\end{routeParameter}
\begin{queryParameter} 
\queryParamItem{on}{hex string - channel ID}
\end{queryParameter} \begin{headerParameter} 
\headerParamItem{SWARM-TAG}{hex string}
\headerParamItem{SWARM-STAMP}{hex string}
\end{headerParameter}
\begin{routeResponse}{application/json}
\responseItem{204}{No Content}{successful unsubscribe}
\responseItem{400}{Bad Request}{topic or channel not well formed.}
\end{routeResponse}
\end{apiRoute}




\subsection{Feeds \statusorange}\label{spec:api:feeds}



% \subsection{The pss URL scheme}

