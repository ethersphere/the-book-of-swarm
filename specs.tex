
The Swarm specification is presented in four chapters. Chapter \ref{spec:convention} enlists all the conventions relating data types, formats and algorithms using the \lstinline{buzz} language as pseudo-code.
Chapter \ref{spec:protocol} ( Protocols) specifies the wire protocols: message formats, serialisation and encapsulation are hard requirements that are crucial for  cross client compatibility. This chapter benefits from \lstinline{protobuf},%
%
\footnote{\url{https://developers.google.com/protocol-buffers}}
%
which is a generic language-neutral, platform-neutral, extensible mechanism for serializing structured data.
The chapter on Strategies (\ref{spec:strategy}) is meant to present recommended incentive-aligned behaviour.
APIs (\ref{spec:api}) gives a formal specification of the high level interfaces of Swarm. Different client implementations are  to be tested against a standard suite of API tests. The API specifications benefit from \lstinline{OpenAPI},%
%
\footnote{\url{https://swagger.io/}}
%
which is an API description format for REST APIs.

\newpage\phantomsection 
\addcontentsline{toc}{chapter}{List of definitions}
\listoftheorems[ignoreall,show={definition}]

\chapter{Data types and algorithms}\label{spec:convention}

\orange{}

\section{Built-in primitives \statusyellow}\label{spec:format:builtin}
\subsection{Crypto \statusgreen}\label{spec:format:crypto}

This section describes the crypto primitives used throughout the specification. They  are exposed as  buzz built-in  functions.
The modules are hashing, random number generation, key derivation, symmetric and asymmetric encryption (ECIES), mining (i.e., finding a nonce), elliptic curve key generation, digital signature (ECDSA), Diffie--Hellman shared secret (ECDH) and Reed-Solomon  (RS) erasure coding.

Some of the built-in crypto primitives (notably, sha3 hash, and ECDSA ecrecover) are replicating crypto functionality of the Ethereum VM. These are defined here with the help of ethereum api calls to a smart contract. This smart contract just implements the primitives of "buzz" and only has read methods.

\subsubsection{Hashing}

The base hash function implements Keccak256 hash as used in Ethereum.

\begin{definition}[Hashing]\label{def:hash}
\begin{lstlisting}[language=buzz1]
// /crypto

define function hash @input []byte
    ?and/with @suff
    return segment
as
    ethereum/call "sha3" with @input append= @suff
         on context contracts "buzz" 
\end{lstlisting}
\end{definition}  


\subsubsection{Random number generation}

\begin{definition}[Random number generation]\label{def:rng}
\begin{lstlisting}[language=buzz1]
// /crypto

define function random type 
    return [@type size]byte 

\end{lstlisting}
\end{definition}    


\subsubsection{Scrypt key derivation}

The crypto key derivation function implements \lstinline{scrypt} \cite{percival2009stronger}.

\begin{definition}[Scrypt key derivation]\label{def:scrypt}
\begin{lstlisting}[language=buzz1]
// /crypto

define type salt as [segment size]byte

define type key as [segment size]byte

// params for scrypt key derivation function
// scrypt.key(password, salt, n, r, p, 32) to generate key

define type kdf
    n int // 262144
    r int // 8
    p int // 1

define function scrypt from @password
    with   salt 
    using  kdf
    return key

\end{lstlisting}
\end{definition}  

\subsubsection{Mining helper}

This module provides a very simple helper function that finds a nonce that when given as the single argument to a mining function returns true.

\begin{definition}[Mining a nonce]\label{def:mine}
\begin{lstlisting}[language=buzz1]
// /crypto

define type nonce as [segment size]byte

define function mine @f function of nonce return bool
as
    @nonce = random key
    return @nonce if call @f @nonce
    self @f
    
\end{lstlisting}
\end{definition}  

\subsubsection{Symmetric encryption}

Symmetric encryption uses a modified blockcipher with 32 byte blocksize in counter mode.
The segment keys are generated by hashing the chunk-specific encryption key with the counter and hash that again. This second step is required so that a segment can be selectively disclosed in a 3rd party provable way yet without compromising the security of the rest of the chunk.

The module provides input length preserving blockcipher encryption.

\begin{definition}[Blockcipher]\label{def:crypt}
\begin{lstlisting}[language=buzz1]
// /crypto

// two-way (en/de)crypt function for segment
define function crypt.segment segment
    with key
    at @i uint8
as
    hash @key and @i        // counter mode 
        hash                // extra hashing
        to @segment length  // chop if needed
        xor @segment        // xor key with segment

// two-way (en/de)crypt function for arbitrary length 
define function crypt @input []byte
    with key
    return [@input length]byte
as
    @segments = @input each segment size    // iterate segments of input
        go crypt.segment at @i++ with @key  // concurrent crypt on segments
    return wait for @segments               // wait for results
        join                                // join (en/de)crypted segments
        
\end{lstlisting}
\end{definition}    

\subsubsection{Elliptic curve keys}

Public key cryptography is the same as in Ethereum, it uses the secp256k1 elliptic curve.  


\begin{definition}[Elliptic curve key generation]\label{def:ec-keys}
\begin{lstlisting}[language=buzz1]
// /crypto
define type pubkey  as [64]byte
define type keypair
    privkey [32]byte
    pubkey
    
define type address as [20]byte

define function address pubkey
    return address
as 
    hash pubkey 
        from 12

define function generate 
    ?using entropy
as
    @entropy = random segment if no @entropy
    http/get "signer/generate?entropy=" append @entropy 
        as keypair

\end{lstlisting}
\end{definition}    

\subsubsection{Asymmetric encryption}

Asymmetric encryption implements ECIES based on  the secp256k1 elliptic curve. 
%  TODO: this needs more detail

\begin{definition}[Asymmetric encryption]\label{def:asymmetric-encryption}
\begin{lstlisting}[language=buzz1]
// /crypto

define function encrypt @input []byte 
    for pubkey
    return [@input length]byte

define function decrypt @input []byte 
    with keypair
    return [@input length]byte

\end{lstlisting}
\end{definition}  


\subsubsection{Signature}

Crypto's  built-in signature module implements secp2156k1 elliptic curve based ECDSA. The actual signing happens in the external signer running as a separate process (possibly within the secure enclave). As  customary  in Ethereum, the  signature is represented and serialised using the r/s/v format,

\begin{definition}[Signature]\label{def:signature}
\begin{lstlisting}[language=buzz1]
// /crypto

define type signature
    r segment
    s segment
    v uint8
    signer private keypair
    

define type doc 
    preamble []byte
    context  []byte
    asset    segment
    
define function sign @input []byte 
    by keypair
    return signature
as
    @doc = doc{ "swarm signature", context caller, @input }
    @sig = http/get "signer/sign?text=" append @doc 
        append "&account=" append @keypair pubkey address
            as signature 
    @sig signer = @keypair
    @sig
    
define function recover signature
    with @input []byte
    from @caller []byte
    return pubkey
as
    @doc =  doc{ "swarm signature", @caller, @input } as bytes
    ethereum/call "ecrecover" with 
        on context contracts "buzz" 
            as pubkey

\end{lstlisting}
\end{definition}  


\subsubsection{Diffie-Hellmann shared secret}

The shared secret module implements elliptic curve based Diffie--Helmann shared secret (ECDH) using the usual secp256k1 elliptic curve.
The actual DH comes from the external signer which is then hashed together with a salt.

\begin{definition}[Shared secret]\label{def:dh}
\begin{lstlisting}[language=buzz1]
// /crypto

define function shared.secret between keypair
    and pubkey
    using salt
    return [segment size]byte
as
    http/get "signer/dh?pubkey=" append @pubkey append "&account=" @keypair address
        hash with @salt

\end{lstlisting}
\end{definition}  

\subsubsection{Erasure coding}\label{spec:format:erasure}

Erasure coding interface provides wrappers \lstinline{extend/repair} for the encoder/decoder that work directly on a list of chunks.%
%
\footnote{Cauchy-Reed-Solomon erasure codes based on \url{https://github.com/klauspost/reedsolomon}.
}

Assuming $n$ out of $m$ coding.
\lstinline{extend} takes a list of $n$ data chunks and an argument for the number of required parities. It returns the parity chunks only.
\lstinline{repair} takes a list of $m$ chunks (extended with \emph{all} parities) and an argument for the number of parities $p=m-n$, that designate the last $p$ chunks as parity chunks. It returns the list of $n$ repaired data chunks only.
The encoder does not know which parts are invalid, so missing or invalid chunks should be set to \lstinline{nil} in the argument to repair.
If parity chunks are needed to be repaired, you call \lstinline{repair @chunks with @parities; extend with @parities}

\begin{definition}[CRS erasure code interface definition]\label{def:crs}
\begin{lstlisting}[language=buzz1]
// /crypto/crs

define function extend @chunks []chunk 
    with @parities uint
    return [@parities]chunk

define function repair @chunks []chunk   
    with @parities uint
   return [@chunks length - @parities]chunk

\end{lstlisting}
\end{definition}

\begin{definition}[CRS erasure coding parameters]\label{def:crs-params}
\begin{lstlisting}[language=buzz1]
// /crypto/crs
define strategy as "race"|"fallback"|"disabled"

define type params 
    parities uint
    strategy 
     
\end{lstlisting}
\end{definition}


\subsection{State store \statusgreen}\label{spec:format:statestore}
% \input{specs/api/statestore.tex}

\begin{definition}[State store]\label{def:state-store}
\begin{lstlisting}[language=buzz1]
// /statestore

define type key []byte
define type db  []byte
define type value []byte

define function create db

define function destroy db

define function put value
    to db
    on key
    
define function get key 
    from db
    return value
\end{lstlisting}
\end{definition}


\subsection{Local context and configuration \statusgreen}\label{spec:format:local}
% \input{specs/api/local.tex}


\begin{definition}[Context]\label{def:scontext}
\begin{lstlisting}[language=buzz1]
// /context

define type contract as "buzz"|"chequebook"|"postage"|""

define type context
    contracts [contract]ethereum/address 
\end{lstlisting}
\end{definition}

\section{Bzz address\statusgreen}\label{spec:format:bzzaddress}
\input{specs/format/bzzaddress.tex}

\section{Chunks, encryption and addressing\statusyellow}
\subsection{Content addressed chunks \statusgreen}\label{spec:format:chunks}
\input{specs/format/chunks.tex}
\subsection{Single owner chunk \statusgreen}\label{spec:format:soc}
\input{specs/format/soc.tex}
\subsection{Binary Merkle Tree Hash \statusyellow}\label{spec:format:bmt}
\input{specs/format/bmt.tex}
\subsection{Encryption \statusyellow}\label{spec:format:encryption}
\input{specs/format/encryption.tex}

\section{Files, manifests and data structures\statusyellow}\label{spec:format:data-structures}
\subsection{Files and the Swarm hash \statusyellow}\label{spec:format:files}
\input{specs/format/files.tex}
\subsection{Manifests \statusyellow}\label{spec:format:manifests}
\input{specs/format/manifests.tex}
% \section{Entanglement coding \statusred}\label{spec:format:entanglements}
%\input{specs/format/entanglement.tex}
% composite api, resolver, tags
\subsection{Chunks}

\begin{apiRoute}{GET}{/chunks/\{reference\}}{Retrieve chunk as in \ref{def:retrieve}}
{
}
{ }

\begin{routeParameter} 
\routeParamItem{reference}{hex string}
\end{routeParameter}
\begin{routeResponse}{application/json}
\responseItem{200}{OK}{chunk data as response body}
% \responseItem{403}{Forbidden}{chunk encrypted but no decryption key in reference}
\responseItem{408}{Request Timeout}{retrieval of chunk times out}
% \responseItem{420}{Enhance Your Calm}{recovery initiated but request timed out} 
\end{routeResponse}
\end{apiRoute}




\begin{apiRoute}{POST}{/chunks/(?span=\{span\})}{Create chunk as in \ref{def:store}}
{
}
{ }

\begin{queryParameter} 
\queryParamItem{span}{integer}
\end{queryParameter}

\begin{headerParameter} 
\headerParamItem{SWARM-TAG}{hex string}
\headerParamItem{SWARM-STAMP}{hex string}
\headerParamItem{SWARM-ENCRYPTION}{hex string}
\headerParamItem{SWARM-PIN}{bool}
\end{headerParameter}
\begin{requestBody}
chunk data
\end{requestBody}
\begin{routeResponse}{application/json}
\responseItem{204}{Created}{reference in response body}
\responseItem{400}{Bad Request}{span parameter not well formed. }
\responseItem{413}{Payload Too Large}{payload size exceeds span value or 4096} 

\end{routeResponse}
\end{apiRoute}




\begin{apiRoute}{GET}{/soc/\param{owner}/\param{id}}{Retrieve single owner chunk as in \ref{def:soc-retrieve} }
{
}
{ }

\begin{routeParameter} 
\routeParamItem{owner}{eth address of single owner}
\routeParamItem{id}{identifier within owner namespace}
\end{routeParameter}
\begin{routeResponse}{application/json}
\responseItem{200}{OK}{a single owner chunk payload in response body}
\responseItem{400}{Bad request}{if owner, id or key is not well formed}
% \responseItem{403}{Forbidden}{single owner chunk encrypted but no decryption key given}
\responseItem{408}{Request Timeout}{retrieval of soc chunk times out} 
% \responseItem{420}{Enhance Your Calm}{recovery initiated but request timed out}
\end{routeResponse}
\end{apiRoute}




\begin{apiRoute}{POST}{/soc/\param{owner}/\param{id}?span=\param{span}\&sig=\param{sig}}{Create new single owner chunk as in \ref{def:soc-store}. The signature (of the id with the content address of wrapped chunk) can be optionally provided, if missing, the default  signer is called. }
{
}
{ }

\begin{routeParameter} 
\routeParamItem{owner}{eth address of owner}
\routeParamItem{id}{identifier - designator within owner namespace}
\end{routeParameter}
\begin{headerParameter} 
\headerParamItem{SWARM-TAG}{hex string}
\headerParamItem{SWARM-STAMP}{hex string}
\headerParamItem{SWARM-PIN}{bool}
\end{headerParameter}
\begin{queryParameter} 
\queryParamItem{span}{integer}
\queryParamItem{sig}{hex string  of rsv format signature}
\end{queryParameter}
\begin{requestBody}
chunk data
\end{requestBody}
\begin{routeResponse}{application/json}
\responseItem{201}{Created}{reference in response body}
\responseItem{400}{Bad Request}{owner,  id, signature, data or span is not well formed} 
\responseItem{401}{Unauthorized}{signing fails} 
\responseItem{404}{Not Found}{owner keypair is not found} 
\responseItem{413}{Payload Too Large}{payload size exceeds span value or 4096} 
\responseItem{422}{Unprocessable Entity}{owner/id pair already exists} 
\end{routeResponse}
\end{apiRoute}

\subsection{Files \statusgreen}\label{spec:api:file}

\begin{apiRoute}{POST}{/files/}{Path description as in \ref{def:swarm-hash}}
{
}
{ }

\begin{headerParameter} 
\headerParamItem{SWARM-TAG}{hex string}
\headerParamItem{SWARM-STAMP}{hex string}
\headerParamItem{SWARM-ENCRYPTION}{hex string}
\headerParamItem{SWARM-PIN}{bool}
\headerParamItem{SWARM-PARITIES}{integer}
\end{headerParameter}
\begin{requestBody}
file data
\end{requestBody}
\begin{routeResponse}{application/json}
\responseItem{201}{Created}{Minimal manifest entry as response body}


\end{routeResponse}
\end{apiRoute}






\begin{apiRoute}{PUT}{/files/\param{reference} }{Append body to file, returns new reference. Note that any intermediate chunk of a file is a file.}
{
}
{ }

\begin{routeParameter} 
\routeParamItem{reference}{hex string}
\end{routeParameter}
\begin{headerParameter} 
\headerParamItem{SWARM-TAG}{hex string}
\headerParamItem{SWARM-STAMP}{hex string}
\headerParamItem{SWARM-ENCRYPTION}{hex string}
\headerParamItem{SWARM-PIN}{bool}
\headerParamItem{SWARM-PARITIES}{integer}
\end{headerParameter}
\begin{requestBody}
file data
\end{requestBody}
\begin{routeResponse}{application/json}
\responseItem{201}{Created}{minimal manifest entry as request body}
\responseItem{400}{Bad Request}{reference is not well formed}
% \responseItem{403}{Forbidden}{encrypted content but no decryption key in  reference}
\responseItem{408}{Request Timeout}{retrieval of file to append to times out} 
% \responseItem{420}{Enhance Your Calm}{recovery initiated but request timed out}
% \responseItem{413}{Pay,load Too Large}{}
\end{routeResponse}
\end{apiRoute}



\begin{apiRoute}{GET}{/files/\param{reference}}{Retrieve file by reference as in \ref{def:file-retrieval}}
{
}
{ }

\begin{routeParameter} 
\routeParamItem{reference}{string}
\end{routeParameter}
\begin{routeResponse}{application/json}
\responseItem{200}{OK}{file contents streamed in response body}
\responseItem{400}{Bad Request}{reference is not well formed}
% \responseItem{403}{Forbidden}{encrypted content but no decryption key in reference}
\responseItem{408}{Request Timeout}{timeout retrieving file} 
\responseItem{416}{Range Not Satisfiable}{offset in range query out of range}
% \responseItem{420}{Enhance Your Calm}{recovery initiated but request timed out}
\end{routeResponse}
\end{apiRoute}


\subsection{Manifest \statusgreen}\label{spec:api:manifest}




\begin{apiRoute}{GET}{/manifest/\param{reference}/\param{path}}{Lookup entry by path in manifest, see \ref{def:manifests-lookup}}
{
}
{ }

\begin{routeParameter} 
\routeParamItem{reference}{hex string}
\routeParamItem{path}{string}
\end{routeParameter}
\begin{routeResponse}{application/json}
\responseItem{200}{OK}{manifest entry in response body}
\responseItem{400}{Bad Request}{reference or path is not well formed}
% \responseItem{403}{Forbidden}{encrypted content but no decryption key in reference}
\responseItem{404}{Not Fund}{path does not exist}
\responseItem{408}{Request Timeout}{timeout retrieving referenced manifest}
% \responseItem{420}{Enhance Your Calm}{recovery initiated but request timed out}
\end{routeResponse}
\end{apiRoute}



\begin{apiRoute}{DELETE}{/manifest/\param{reference}/\param{path} }{Delete entry on path in referenced manifest, see \ref{def:manifest-remove}}
{
}
{ }

\begin{routeParameter} 
\routeParamItem{reference}{hex string}
\routeParamItem{path}{string}
\end{routeParameter}
\begin{routeResponse}{application/json}
\responseItem{200}{OK}{reference to new manifest}
\responseItem{400}{Bad Request}{reference or path is not well formed}
% \responseItem{403}{Forbidden}{encrypted content but no decryption key in reference}
\responseItem{404}{Not Found}{path to delete does not exist}
\responseItem{408}{Request Timeout}{timeout retrieving referenced manifest}
% \responseItem{420}{Enhance Your Calm}{recovery initiated but request timed out}
\end{routeResponse}
\end{apiRoute}



\begin{apiRoute}{PUT}{/manifest/\param{reference}/\param{path}}{Update manifest as in \ref{def:manifest-update}}
{
}
{ }

\begin{routeParameter} 
\routeParamItem{reference}{hex string}
\routeParamItem{path}{(hex) string}
\end{routeParameter}
\begin{requestBody}
reference to add on path
\end{requestBody}
\begin{routeResponse}{application/json}
\responseItem{200}{OK}{reference to new manifest}
\responseItem{400}{Bad Request}{reference is not well formed}
% \responseItem{403}{Forbidden}{encrypted content but no decryption key in reference}
\responseItem{404}{Not Found}{path does not exist}
\responseItem{408}{Request Timeout}{timeout retrieving referenced manifest}
% \responseItem{420}{Enhance Your Calm}{recovery initiated but request timed out}

\end{routeResponse}
\end{apiRoute}




\begin{apiRoute}{POST}{/manifest/\param{one}/\param{other}}{Merge manifests as in \ref{def:manifest-merge}}
{
}
{ }

\begin{routeParameter} 
\routeParamItem{one}{hex string - reference to manifest to merge}
\routeParamItem{other}{hex string}
\end{routeParameter}
\begin{routeResponse}{application/json}
\responseItem{201}{Created}{reference in response body}
\responseItem{400}{Bad Request}{reference is not well formed}
% \responseItem{403}{Forbidden}{encrypted content but no decryption key in reference}
\responseItem{408}{Request Timeout}{timeout retrieving referenced manifest}
% \responseItem{420}{Enhance Your Calm}{recovery initiated but request timed out}
\end{routeResponse}
\end{apiRoute}

\subsection{High level storage API \statusyellow}\label{spec:api:high-level-storage}


\begin{apiRoute}{GET}{/bzz/\param{host}/\param{path}}{Download file}
{
}
{ }

\begin{routeParameter} 
\routeParamItem{host}{string - reference or ENS domain}
\routeParamItem{path}{string}
\end{routeParameter}
\begin{headerParameter} 
\headerParamItem{SWARM-TAG}{hex string}
\headerParamItem{SWARM-STAMP}{hex string}
\headerParamItem{SWARM-ENCRYPTION}{hex string}
\headerParamItem{SWARM-PIN}{bool}
\headerParamItem{SWARM-PARITIES}{integer}
\end{headerParameter}
\begin{routeResponse}{application/json}
\responseItem{200}{OK}{file streamed in response body}
\responseItem{400}{Bad Request}{host/reference or path is not well formed}
\responseItem{401}{Unauthorized}{access denied: AC unlock failed}
% \responseItem{403}{Forbidden}{encrypted content but no decryption key in reference}
\responseItem{404}{Not Found}{host cannot be resolved or Path does not exist}
\responseItem{408}{Request Timeout}{timeout retrieving referenced manifest}
\responseItem{416}{Range Not Satisfiable}{offset in range query out of range}
% \responseItem{420}{Enhance Your Calm}{recovery initiated but request timed out}
\end{routeResponse}
\end{apiRoute}



\begin{apiRoute}{PUT}{/bzz/\param{host}/\param{path}}{Append upload to file referenced, add new entry to path, returns new manifest}
{
}
{ }

\begin{routeParameter} 
\routeParamItem{host}{string - reference or ENS domain}
\routeParamItem{path}{string}
\end{routeParameter}
\begin{headerParameter} 
\headerParamItem{SWARM-TAG}{hex string}
\headerParamItem{SWARM-STAMP}{hex string}
\headerParamItem{SWARM-ENCRYPTION}{hex string}
\headerParamItem{SWARM-PIN}{bool}
\headerParamItem{SWARM-PARITIES}{integer}
\end{headerParameter}
\begin{requestBody}
file/collection data
\end{requestBody}
\begin{routeResponse}{application/json}
\responseItem{201}{Created}{new manifest root reference in response body}
\responseItem{400}{Bad Request}{host/reference is not well formed}
\responseItem{401}{Unauthorized}{access denied: AC unlock failed}
% \responseItem{403}{Forbidden}{encrypted content but no decryption key in reference}
\responseItem{404}{Not Found}{host cannot be resolved or path does not exist}
\responseItem{408}{Request Timeout}{timeout retrieving referenced manifest}
% \responseItem{420}{Enhance Your Calm}{recovery initiated but request timed out}
\end{routeResponse}
\end{apiRoute}




\begin{apiRoute}{POST}{/bzz/\param{host}/\param{path}}{Upload file or collection, returns new manifest root reference}
{
}
{ }


\begin{routeParameter} 
\routeParamItem{host}{string - reference or ENS domain}
\routeParamItem{path}{string}
\end{routeParameter}
\begin{headerParameter} 
\headerParamItem{SWARM-TAG}{hex string}
\headerParamItem{SWARM-STAMP}{hex string}
\headerParamItem{SWARM-ENCRYPTION}{hex string}
\headerParamItem{SWARM-PIN}{bool}
\headerParamItem{SWARM-PARITIES}{integer}
\end{headerParameter}
\begin{requestBody}
file/collection data
\end{requestBody}
\begin{routeResponse}{application/json}
\responseItem{201}{Created}{new manifest root reference in response body}
\responseItem{400}{Bad Request}{host/reference is not well formed}
\responseItem{401}{Unauthorized}{access denied: AC unlock failed}
% \responseItem{403}{Forbidden}{encrypted content but no decryption key in reference}
\responseItem{404}{Not found}{host cannot be resolved or path does not exist}
\responseItem{408}{Request Timeout}{timeout retrieving referenced manifest}
% \responseItem{420}{Enhance Your Calm}{recovery initiated but request timed out}
\end{routeResponse}
\end{apiRoute}




\section{Access Control \statusgreen}\label{spec:format:access-control}

\begin{apiRoute}{POST}{/access/\param{address} }{Lock ACT for address as in \ref{def:ac-api}}
{
}
{ }

\begin{routeParameter} 
\routeParamItem{address}{hex string}
\end{routeParameter}
\begin{routeResponse}{application/json}
\responseItem{201}{Created}{root access manifest reference in response body}
\responseItem{400}{Bad Request}{encrypted content but no decryption key in reference}
\responseItem{403}{Forbidden}{encrypted content but no decryption key in reference}
\responseItem{404}{Not Found}{}
\responseItem{408}{Request Timeout}{timeout retrieving referenced manifest}
\responseItem{420}{Enhance your calm}{recovery initiated but request timed out}
\end{routeResponse}
\end{apiRoute}



\begin{apiRoute}{GET}{/access/\param{address} }{Unlock ACT for address as in \ref{def:ac-api}}
{
}
{ }

\begin{routeParameter} 
\routeParamItem{address}{hex string}
\end{routeParameter}
\begin{routeResponse}{application/json}
\responseItem{200}{OK}{}
\responseItem{400}{Bad Request}{address not well formed}
\responseItem{401}{Unauthorized}{access denied: AC unlock failed}
\responseItem{403}{Forbidden}{encrypted content but no decryption key in reference}
\responseItem{408}{Request Timeout}{timeout retrieving referenced manifest}
\responseItem{420}{Enhance your calm}{recovery initiated but request timed out}
\end{routeResponse}
\end{apiRoute}




\begin{apiRoute}{PUT}{/access/\param{root}/\param{pubkey}}{Add entry for pubkey to the  ACT referred in the root access manifest \ref{def:act-api}}
{
}
{ }

\begin{routeParameter} 
\routeParamItem{root}{hex string - reference to root access manifest}
\routeParamItem{pubkey}{hex string - public key of grantee}
\end{routeParameter}
\begin{routeResponse}{application/json}
\responseItem{201}{Created}{reference to new manifest root in response body}
\responseItem{400}{Bad Request}{address or public key not well formed}
\responseItem{401}{Unauthorized}{access denied: creating session key failed}
\responseItem{403}{Forbidden}{encrypted content but no decryption key in reference}
\responseItem{408}{Request Timeout}{timeout retrieving referenced manifest}
\responseItem{420}{Enhance your calm}{recovery initiated but request timed out}
\end{routeResponse}
\end{apiRoute}


\begin{apiRoute}{DELETE}{/access/\param{root}/\param{pubkey}}{Remove entry for pubkey from ACT referred in the root access manifest, see \ref{def:act-api}}
{
}
{ }

\begin{routeParameter} 
\routeParamItem{root}{hex string - reference to root access manifest}
\routeParamItem{pubkey}{hex string - public key of grantee}
\end{routeParameter}
\begin{routeResponse}{application/json}
\responseItem{201}{Created}{reference to new manifest root in response body}
\responseItem{400}{Bad Request}{address or public key not well formed}
\responseItem{401}{Unauthorized}{access denied: creating session key failed}
\responseItem{403}{Forbidden}{encrypted content but no decryption key in reference}
\responseItem{408}{Request Timeout}{timeout retrieving referenced manifest}
\responseItem{420}{Enhance your calm}{recovery initiated but request timed out}
\end{routeResponse}
\end{apiRoute}


\section{PSS \statusyellow}

\subsection{PSS message\statusgreen}
\label{spec:format:pss-messsage}
\subsection{Direct pss message with trojan chunk}

Pss has two fundamental types, a message and a trojan chunk structure which wraps the encrypted serialised message and contains a nonce that is mined to make the resulting chunk's content address (BMT hash) to match the targets.


\begin{definition}[Basic types: topic, targets, recipient, message and trojan]\label{def:pss-message}
\begin{lstlisting}[language=buzz1]
// /pss


define type topic        as [segment size]byte       // obfuscated topic matcher
define type targets      as [][]byte       // overlay prefixes 
define type recipient    as crypto/pubkey

// pss message
define type message 
    seal    segment            
    payload [!:4030]byte    // varlength padded to 4030B
    
// trojan chunk
// the nonce 
define type trojan 
    nonce   segment           // the nonce to mine 
    key  pubkey               // compressed format 
    message [4064]byte        // encrypted msg 
\end{lstlisting}
\end{definition}


The message is encoded in a way that allows integrity checking and at the same time obfuscates the topic. The operation to package the payload with a topic is called \emph{sealing}


\begin{definition}[Sealing/unsealing the message]\label{def:pss-sealing}
\begin{lstlisting}[language=buzz1]
// /pss

define function seal @payload []byte
    with topic
as
    @seal = hash @payload and @topic // obfuscate topic
        xor @topic          
    return message{ @seal, @payload }

define function unseal message
    with topic 
as
    @seal = hash @message payload and @topic 
    if @topic == @seal xor @message seal then // check 
        return @payload 
    return nil
    
\end{lstlisting}
\end{definition}

Functions \lstinline{wrap/unwrap} transform between message and trojan chunk. \lstinline{wrap} takes an optional recipient public key to asymmetrically encrypt the message.
The targets are a list of overlay address prefixes derived from overlay addresses of recipients, with length specified to guarantee that a chunk matching it will end up with the recipient solely as a result if  push-syncing.   

\begin{definition}[Wrapping/unwrapping]\label{def:wrap}
\begin{lstlisting}[language=buzz1]
// /pss

define function wrap message 
    for recipient
    to  targets
as 
    @msg = @message 
        (crypto/encrypt for @recipient if @recipient) 

    @nonce = crypto/mine @n such that
        @targets any is prefix of
            trojan{@n, @msg} as chunk address 
    trojan{@nonce, @msg} as chunk 

define function unwrap chunk
    for recipient
as
    @chunk bytes 
        (crypto/decrypt  for @recipient if @recipient)
            as message

\end{lstlisting}
\end{definition}

When a chunk arrives at the node,  \lstinline{pss/deliver} is called as a hook by the storage component.
First the message is unwrapped using the recipient private key and unsealed with all the topics API clients subscribed to. If the unsealing is successful, message integrity as well as topic matching is proven so the payload is written into the stream registered for the topic in question.

\begin{definition}[Incoming message handling]\label{def:delivery}
\begin{lstlisting}[language=buzz1]
// /pss

// mailbox is a handler type, expects payload
// sent sealed with the topic to be delivered via the stream 
define type mailbox
    topic
    deliveries stream of []byte 
    
define context mailboxes as []mailbox

define function deliver chunk
    @msg = @chunk unwrap for context recipient
    mailboxes each @mailbox 
        @payload = unseal @msg  with @mailbox topic
        if @payload then 
            write @msg payload 
                to @mailbox deliveries 
    

\end{lstlisting}
\end{definition}


\begin{definition}[pss API: send]\label{def:send}
\begin{lstlisting}[language=buzz1]
// /pss

define function send @payload []byte
    about topic
    for recipient
    ?to    targets
has api POST on "/pss/<recipient>/<topic>(?targets=<targets>)"
    with @payload as body
as 
    targets = lookup.targets for @recipient if no @targets
    context tag = tag/tag{}
    seal @payload with @topic        // seal with topic
        wrap for @recipient          // encrypt if given recipient
            to @targets              // mine nonce and returns trojan chunk
                store                // to be sent by push-sync
    return tag                       // tag to monitor status 
    
\end{lstlisting}
\end{definition}

\begin{definition}[pss API: receive]\label{def:receive}
\begin{lstlisting}[language=buzz1]
// /pss

define function receive about topic 
    on uint64 @channel
has api POST on "/pss/subscribe/<topic>(?on=<channel>)"
as 
    @stream = open @channel
    context mailboxes append= mailbox{ @topic, @stream }
    
define function cancel topic
    on @channel uint64
has api DELETE on "/pss/subscribe/<topic>(?on=<channel>)"
as
    context mailboxes any ch
\end{lstlisting}
\end{definition}

\subsection{Envelopes}

\begin{definition}[Envelope]\label{def:pss-envelope}
\begin{lstlisting}[language=buzz1]
// /pss

define type envelope
    id  [segment size]byte
    sig crypto/signature
    ps  postage/stamp   
    
\end{lstlisting}
\end{definition}

\subsection{Update notifications \statusred}\label{spec:format:update-notifications}
%\input{specs/format/update-notifications.tex}

\subsection{Chunk recovery  \statusyellow}\label{spec:format:recovery}
\input{specs/format/recovery.tex}

\section{Postage stamps \statusorange}\label{spec:format:postage-stamps}
\input{specs/format/postage-stamps.tex}
% \section{Honey token and multi-chain support}\label{spec:format:honey}
%\input{specs/format/honey.tex}


\chapter{Protocols}\label{spec:protocol}

\section{Introduction \statusorange}\label{spec:protocol:intro}

\subsection{Underlay network \statusgreen}

The \lstinline{libp2p} networking stack provides all the required properties for the Swarm underlay network, as laid out in \ref{sec:underlay-transport}.

\begin{enumerate}
\item Addressing is facilitated through the use of \emph{multi addresses}, known as underlay addresses, assigned to each node. Every node can have multiple underlay addresses depending on the configured network listening addresses and transports.
\item Dialing is provided over \lstinline{libp2p} supported network transports.
\item Listening is provided by \lstinline{libp2p} supported network transports.
\item Live connections are established and maintained between two peers, ensuring the ability to send and receive messages.
\item Channel security is ensured through \lstinline{TLS} and the \lstinline{libp2p secio} stream security transport.
\item Protocol multiplexing is provided by the \lstinline{libp2p mplex} stream multiplexer protocol.
\item Delivery guarantees are provided by using \lstinline{libp2p} bidirectional streams to validate the responses from peers upon message transmission.
\item Serialization is not enforced by \lstinline{libp2p}, as allows for byte streams, providing flexibility for each protocol to choose the most appropriate serialization method. The recommended serialization approach is using \lstinline{Protobuf} with \lstinline{varint} delimited messages in streams.
\end{enumerate}

\subsection{Protocols and streams \statusgreen}

Communication between peers is organised in protocols as logical units under a unique name that may define one or more \emph{streams}. \lstinline{libp2p} provides streams as the basic channel for communication. Streams are full-duplex channels of bytes, multiplexed over a singe connection between two peers.

Every stream defines:

\begin{itemize}
\item a version that follows semantic versioning in semver form
\item data serialization definitions
\item sequence of data passing between peers over a full-duplex stream
\end{itemize}

Streams are identified by \lstinline{libp2p} case-sensitive protocol IDs. The following convention is used construct stream identifiers:

\begin{lstlisting}
/swarm/ProtocolName/ProtocolVersion/StreamName
\end{lstlisting}

\begin{itemize}
\item All stream IDs are prefixed with \lstinline{/swarm}.
\item \lstinline{ProtocolName} is an arbitrary string that identifies the protocol.
\item \lstinline{ProtocolVersion} is a string in semver form that is used to specify compatibility between protocol implementations over time.
\item \lstinline{StreamName} is an arbitrary string that identifies a stream defined as part of the protocol.
\end{itemize}

\subsection{Data exchange sequences \statusgreen}

A data passing sequence must be synchronous under one opened stream. Multiple streams can be opened at the same time that are multiplexed over the same connection exchanging data independently and asynchronously. Streams may use different data exchange sequences such as:

\begin{itemize}
\item \emph{single message sending} - not waiting for the response by the peer if it is not needed before closing the stream.
\item \emph{multiple message sending} - a series of data that is sent to a peer without reading from it before closing the stream.
\item \emph{request/response} - requires a single response for a single request before closing the stream.
\item \emph{multiple requests/response cycles} - require a synchronous response after every request before closing the stream.
\item \emph{exact message sequence} -  requires multiple message types over a single stream in an exact order (see the handshake protocol in \ref{spec:protocol:hive}).
\end{itemize}

Streams have predefined sequences that are kept as simple as possible for a single purpose. For complex message exchanges, multiple streams should be used.

Streams may be short lived for immediate data exchange or communication, or long lived for notifications if needed.

\subsection{Stream headers}

A Swarm specific requirement for all libp2p streams is to exchange Header protobuf messages on every stream initialization between two peers. This message encapsulates a stream scoped information that needs to be exchanged before any stream specific data or messages are exchanged. Headers are sequences of key value pairs, where keys are arbitrary strings and values are byte arrays that do not impose any specific encoding. Every key may use appropriate encoding for the data that it relates to.

\begin{definition}[Header message]\label{def:headers-message}

\begin{lstlisting}[language=protobuf3]
syntax = "proto3";

package pb;

message Headers {
    repeated Header headers = 1;
}

message Header {
  string key = 1;
  bytes value = 2;
}
\end{lstlisting}
\end{definition}

On every stream initialization, the peer that creates it, is sending Headers message regardless if it contains header values or not. The receiving node must read this message and respond with response header using the same message type. This makes the header exchange sequence finished and any other stream data can be transmitted depending on the protocol.

Standard header key names are defined here:

\begin{enumerate}
\item tracing-span-context
\end{enumerate}


\subsection{Encapsulation of context for tracing \statusgreen}

P2P Stream scoped tracing span context is exchanged by using stream headers. Header key "tracing-span-context" is reserved for binary encoded tracing span context data. This context should be used in tracing messages. The stream initiator node should provide tracing span context to the responding node. This context is optional and all nodes must function the same as span context is provided by other nodes or not regardless if the node has tracing configured or not.


% \section{Swarm protocol basics\statusgreen}\label{spec:protocol:basics}
\section{Bzz handshake protocol \statusgreen}\label{spec:protocol:bzz}


The bzz handshake protocol is the protocol that is always run after two peers are connected and before any other protocols are established. It communicates information about the  peer's address, network ID and light node capability.

The handshake protocol defines only one stream and three messages:

\begin{definition}[Bzz handshake protocol  messages]\label{def:bzz-messages}

\begin{lstlisting}
// ID: /swarm/handshake/1.0.0/handshake

syntax = "proto3";

message Syn {
    bytes Address = 1;
    int32 NetworkID = 2;
    bool Light = 3;
}

message SynAck {
    Syn Syn = 1;
    Ack Ack = 2;
}

message Ack {
    bytes Address = 1;
}
\end{lstlisting}
\end{definition}

This message sequence is inspired by the TCP three way handshake to ensure message deliverability.

Upon connection a requesting peer constructs a new handshake stream and sends a \lstinline{Syn} message with its Overlay address, Network ID and Lightnode capability flag, and waits for \lstinline{SynAck} response message from the responding peer. In  the  \lstinline{SynAck} message responder sends it own \lstinline{Syn} message as well as an acknowledgement with the received Overlay address. After the requesting peer receives the \lstinline{SynAck} message from the responding peer and validates that the received \lstinline{Ack} information in it is correct, it sends an \lstinline{Ack} message itself as a confirmation to the responding peer. The stream is closed by the responding peer after it receives the \lstinline{Ack} message.

The connection must be terminated if network IDs are do not match or if the  exact  order of messages is not followed.

The bzz address is verified and overlay, underlay and signature are extracted.
light is a boolean field indicating whether the node is operating as a light (as opposed to full) node.

After the handshake,  each peer should remember the following data about the other:

\begin{itemize}
    \item the overlay address - used in forwarding  (see  \ref{spec:strategy:forwarding}),
    \item the underlay address - used for dialing, passed to the underlay network protocol when the connectivity driver needs to connect to the peer (see \ref{spec:strategy:connection}),
    \item the bzz address signature - needed by the hive protocol to pass information about the node to other peers (see \ref{spec:protocol:hive}),
    \item whether the peer is a light node.
\end{itemize}

% \subsection{Encapsulation of price information \statusred}



\section{Hive discovery  \statusgreen}\label{spec:protocol:hive}
The \gloss{hive protocol} enables nodes to exchange information about other peers that are relevant to them in order to bootstrap their connectivity  (see \ref{sec:bootstrapping}) . The information communicated are both overlay and underlay addresses of the known remote peers (see  \ref{spec:format:bzzaddress}). The overlay address serves to select peers to achieve the connectivity pattern needed for the desired network topology. Underlay address is needed to establish the peer connections by dialing selected peers.

\subsection{Streams and messages \statusgreen}


The protocol specifies one stream with two messages:

\begin{definition}[Hive protocol messages]\label{def:hive-messages}

\begin{lstlisting}
// /swarm/hive/1.0.0/peers
syntax = "proto3";

package hive;

message Peers {
    repeated BzzAddress peers = 1;
}

message BzzAddress {
    bytes Underlay = 1;
    bytes Signature = 2;
    bytes Overlay = 3;
}

\end{lstlisting}
\end{definition}

During the lifetime of connection, nodes can broadcast newly received peers to their peers. This is done by sending the \lstinline{Peers} message over the \\\lstinline{/swarm/hive/1.0.0/peers} stream.

Upon receiving a peers message, nodes are meant to store the peer information in their \gloss{address book}, i.e., a data structure containing info about peers known to the node. The address book is meant to be used to suggest peers  to a connectivity manager according to a connection strategy (\ref{spec:strategy:connection}) in order to bootstrap kademlia topology%
% (\ref{sec:kademlia-connectivity})%
. The address book is meant to be persisted across sessions.

\subsubsection{Sending side}

A stream with the appropriate id is created and a \lstinline{Peers} message is sent over the stream. There is no response to this message. The sending node should wait for the receiving side to close its side of the stream, before closing the stream themselves and moving on.

\subsubsection{Receiving side}

When the stream is created, receiving node should wait for a \lstinline{Peers} message. After receiving the message, node should close its side of the stream to let the sender node know that the message was received, and move on with processing. If the new node was not known, it should also be forwarded to all connected peers closer to peer address then the node themselves.

\section{Retrieval  \statusorange}\label{spec:protocol:retrieval}
\input{specs/protocol/retrieval.tex}

\section{Push-syncing  \statusorange}\label{spec:protocol:push-sync}
\input{specs/protocol/push-sync.tex}

\section{Pull-syncing \statusorange}\label{spec:protocol:pull-sync}
\input{specs/protocol/pull-sync.tex}

\section{SWAP settlement protocol \statusorange}\label{spec:protocol:swap}


\chapter{Strategies \statusorange}\label{spec:strategy}


The strategies clients follow will have a fundamental effect on the behaviour of the network and if they end up going against the preconceived design, the project may very well fail to suit user expectations. Such scenarios can easily turn fatal to the project, which is why it is instructive to err on the side of caution when change (or initially suggest) strategies for node behaviour.
The scope of strategies are defined as those aspects of the intended 'protocol' which cannot be directly observed or easily verified. As an example contrast the very act of forwarding an incoming retrieve request (strategy) with the act of using correctly formatted and serialised messages when doing so (protocol constraint). The latter can be immediately detected and be responded to by disconnecting and blacklisting offenders. In contrast, whether a node does forwarding in a way conducive to the desired network outcome is subtle to detect. 

Since we choose to work with the narrowest possible assumption of profit-maximising node operators, the choice of strategies ought to be course grained enough to evaluate the consequences of the options. In particular we must not allow for scenarios when even vaguely rational deviations from the recommended strategy have catastrophic effects.

In general incentives should be in place to guarantee that behaviour that is detrimental to the service incurs a risk of subsequent loss, deterrent upfront cost or opens up reciprocal  vulnerability. 

The constraints put on strategic choice are crucial in terms of rendering the game theory feasible to simulations and experiments or express them with simple enough analytical models to aid reasoning. 

\section{Connection  \statusorange}\label{spec:strategy:connection}

All new (previously not known) peers found in the \lstinline{Peers} message received from either of the two streams as well as newly connected peers that dialled in should be automatically broadcast to all connected peers that are closer  to them than the node's base overlay address. Formally, 
node $s$ (sender) notifies an existing peer $r$ (recipient) about peer $p$ if $\mathit{PO}(s, r) = \mathit{PO}(s, p)$. 

\subsection{Connectivity and its constraints}

Without resource constraints the optimal connectivity for a proper storer node is full  connectivity, i.e. when a node has all other nodes in the network as their peer. The chequebook contract also prefers a small number of peer connections to be maintained and other network overhead costs for keepalive connections or network socket shortage also implicates that 
above a certain network size full connectivity is prohibitive.

The relative gain is maximised if throughput on all keepalive connections are maximised. If network contention prevents a node from forwarding to a peer instantaneously, another peer must to be chosen. 
Assuming uniformity in the network 
%(see appendix \ref{sec:distribution}) 
throughput maximisation in the context of limited number of peer connections can be achieved with a connectivity pattern where each kademlia bin has a constant cardinality of $2^b$ and peers in the bin are balanced, i.e., match each distinct bit prefix of length $b$.
Therefore, connectivity strategy can be formulated as follows:







\subsection{Light node connection strategy}

Light nodes in the context of connection topology are nodes that do not have kademlia connectivity. In the extreme case, a light node should be able to get away with a single connection. The lack of full connectivity is indicated to the node's peers so that no retrieve requests or  push sync requests are sent to them. Note that if a node falsely indicates its status, that should cause minimal disruption. 

A lightnode wants to identify its neighboorhood so that it connects to at least one of the $R$ storer nodes closest to it (where $R$ is the redundancy parameter determining the minimum neighbourhood size).





\section{Forwarding  \statusorange}\label{spec:strategy:forwarding}
\input{specs/strategy/forwarding.tex}

%peer selection and pricing
\section{Pricing  \statusorange}\label{spec:strategy:pricing}
\input{specs/strategy/pricing-retrieval.tex}

\section{Accounting and settlement  \statusorange}\label{spec:strategy:swap}
\input{specs/strategy/swap.tex}

\section{Push-syncing  \statusorange}\label{spec:strategy:push-sync}
\input{specs/strategy/push-sync.tex}

\section{Pull-syncing  \statusorange}\label{spec:strategy:pull-sync}
\input{specs/strategy/pull-sync.tex}

\section{Garbage collection \statusorange}\label{spec:strategy:garbage-collection}
\input{specs/strategy/garbage-collection.tex}

\chapter{API-s}\label{spec:api}


\begin{definition}[HTTP status codes used in swarm]
\begin{lstlisting}
\end{lstlisting}
\begin{tabular}{l|p{0.25\linewidth}|p{0.6\linewidth}}
103 & Checkpoint & returns temporary root hash for resumable uploads
\\\hline
200 & OK &
\\
201 & Created & returned by POST requests upon successful creation of file/manifest/tag/stamp/\\
204 & No Content & returned by DELETE requests upon successful  deletion\\
209 & Sent & returned by pss POST upon successful send
\\\hline
400 & Bad request & returned if request or its parameters are not well formed or missing
\\
401 & Unauthorized & returned by access control if authentication fails
\\
402$^{*}$ & Payment Required & returned if no swap balance or missing/invalid postage stamp
% \\
% 403 & Forbidden & returned if retrieved chunk is encrypted  but the reference has no decryption key
\\     
404 & Not Found &
returned if a local prerequisite is not found or manifest path does not exist.
\\
405$^{*}$& Method Not Allowed & HTTP verb  not allowed for this endpoint
\\
406$^{*}$
%
& Not Acceptable & No format acceptable by the {ACCEPT} header explicit in the request. 
\\
408 & Request Timeout & Retrieve requests fallback error after TTL passed
\\
% 411 & Length Required & returned by chunk upload API if length of file uploaded is beyond limit
% 412 & Precondition Failed (RFC 7232)
% \\
413 & Payload Too Large  &
Payload size exceeds maximum chunk size or span given (chunk API)
% \\
% 414 & URI Too Long  & manifest path  $>32 $
\\
416 & Range Not Satisfiable  & offset in range query out of range
% \\
% 420 & Enhance your calm & returned when recovery was initiated but retrieval timed out
\\
422 & Unprocessable Entity & returned by the blockchain external API if eth api returns an error or by single owner chunk post API if owner/id pair already exists
% 417 & Expectation Failed
% 421 & Misdirected Request (RFC 7540)
% 451 & Unavailable For Legal Reasons (RFC 7725)
\end{tabular}

\footnotesize{$^{*}$Generic errors detectable before endpoint API call so not documented.}
\end{definition}


\section{External API requirements\statusorange}\label{spec:api:external}

\subsection{Signer}\label{spec:api:signer}
% \input{specs/api/signer.tex}

% \UseRawInputEncoding
\begin{apiRoute}{GET}{/sign/\param{id}/\param{document}}{ECDSA signature}
 {
}
{ }
\begin{routeParameter} 
\routeParamItem{id}{hex string - eth address}
\routeParamItem{document}{(hex) string -  document to sign (is prefixed and hashed before signing)}
\end{routeParameter} \begin{routeResponse}{application/json}
\responseItem{200}{OK}{}
\responseItem{401}{Unauthorised}{failed authentication on existing identity} 
\responseItem{404}{Not Found}{unknown identity} 
\end{routeResponse} 
\end{apiRoute}

\begin{apiRoute}{GET}{/dh/\param{id}/\param{pubkey}}{Diffie-Hellman shared secret}
 {
}
{ }
\begin{routeParameter} \routeParamItem{id}{hex string - eth address}
\routeParamItem{pubkey}{hex string - represents the remote party in the shared secret arrangement}
\end{routeParameter} \begin{routeResponse}{application/json}
\responseItem{200}{OK}{} 
\responseItem{401}{Unauthorised}{failed authentication on existing identity}
\responseItem{404}{Not Found}{unknown identity}
\end{routeResponse} \end{apiRoute}


\subsection{Blockchain \statusgreen}\label{spec:api:blockchain}
% \input{specs/api/blockchain.tex}

\begin{apiRoute}{GET}{/eth/\param{contract}/\param{function}/\param{args}}
{ethereum API call}
 {
}
{ }
\begin{routeParameter}
\routeParamItem{contract}{hex string, eth address of contract}
\routeParamItem{function}{endpoint within contract}
\routeParamItem{args}{arguments for the eth API call}
\end{routeParameter} \begin{routeResponse}{application/json}
\responseItem{200}{ok}{}  \responseItem{400}{Bad request}{unknown contract or function endpoint given} 
\responseItem{404}{Not Found}{unknown contract or function endpoint} \responseItem{422}{Unprocessable entity}{incorrect ABI,  error by eth API}
\end{routeResponse} \end{apiRoute}

\begin{apiRoute}{POST}{/eth/\param{contract}/\param{function}/\param{args}}
{ethereum API send transaction}
 {
}
{ }
\begin{routeParameter}
\routeParamItem{contract}{hex string, eth address of contract}
\routeParamItem{function}{endpoint within contract}
\routeParamItem{args}{arguments for the eth API call}
\end{routeParameter} \begin{routeResponse}{application/json}
\responseItem{200}{OK}{transaction data in response body}
\responseItem{400}{Bad Request}{unknown contract or function endpoint given} 
\responseItem{401}{Unauthorised}{failure signing transaction}
\responseItem{404}{Not Found}{unknown contract or function endpoint} 
\responseItem{422}{Unprocessable entity}{incorrect ABI,  error by eth API}
\end{routeResponse} \end{apiRoute}


\subsection{User input}\label{spec:api:input}


\begin{apiRoute}{GET}{/input/\param{id}}
 {
}
{ }
\begin{routeParameter}
\routeParamItem{id}{hex string - eth address of the persona the input is expected from}
\routeParamItem{note}{the question to be answered or instruction to select}
\end{routeParameter} \begin{routeResponse}{application/json}
\responseItem{200}{ok}{}
\end{routeResponse} \end{apiRoute}



\section{Storage API \statusyellow}\label{spec:api:storage}
\subsection{Chunks}

\begin{apiRoute}{GET}{/chunks/\{reference\}}{Retrieve chunk as in \ref{def:retrieve}}
{
}
{ }

\begin{routeParameter} 
\routeParamItem{reference}{hex string}
\end{routeParameter}
\begin{routeResponse}{application/json}
\responseItem{200}{OK}{chunk data as response body}
% \responseItem{403}{Forbidden}{chunk encrypted but no decryption key in reference}
\responseItem{408}{Request Timeout}{retrieval of chunk times out}
% \responseItem{420}{Enhance Your Calm}{recovery initiated but request timed out} 
\end{routeResponse}
\end{apiRoute}




\begin{apiRoute}{POST}{/chunks/(?span=\{span\})}{Create chunk as in \ref{def:store}}
{
}
{ }

\begin{queryParameter} 
\queryParamItem{span}{integer}
\end{queryParameter}

\begin{headerParameter} 
\headerParamItem{SWARM-TAG}{hex string}
\headerParamItem{SWARM-STAMP}{hex string}
\headerParamItem{SWARM-ENCRYPTION}{hex string}
\headerParamItem{SWARM-PIN}{bool}
\end{headerParameter}
\begin{requestBody}
chunk data
\end{requestBody}
\begin{routeResponse}{application/json}
\responseItem{204}{Created}{reference in response body}
\responseItem{400}{Bad Request}{span parameter not well formed. }
\responseItem{413}{Payload Too Large}{payload size exceeds span value or 4096} 

\end{routeResponse}
\end{apiRoute}




\begin{apiRoute}{GET}{/soc/\param{owner}/\param{id}}{Retrieve single owner chunk as in \ref{def:soc-retrieve} }
{
}
{ }

\begin{routeParameter} 
\routeParamItem{owner}{eth address of single owner}
\routeParamItem{id}{identifier within owner namespace}
\end{routeParameter}
\begin{routeResponse}{application/json}
\responseItem{200}{OK}{a single owner chunk payload in response body}
\responseItem{400}{Bad request}{if owner, id or key is not well formed}
% \responseItem{403}{Forbidden}{single owner chunk encrypted but no decryption key given}
\responseItem{408}{Request Timeout}{retrieval of soc chunk times out} 
% \responseItem{420}{Enhance Your Calm}{recovery initiated but request timed out}
\end{routeResponse}
\end{apiRoute}




\begin{apiRoute}{POST}{/soc/\param{owner}/\param{id}?span=\param{span}\&sig=\param{sig}}{Create new single owner chunk as in \ref{def:soc-store}. The signature (of the id with the content address of wrapped chunk) can be optionally provided, if missing, the default  signer is called. }
{
}
{ }

\begin{routeParameter} 
\routeParamItem{owner}{eth address of owner}
\routeParamItem{id}{identifier - designator within owner namespace}
\end{routeParameter}
\begin{headerParameter} 
\headerParamItem{SWARM-TAG}{hex string}
\headerParamItem{SWARM-STAMP}{hex string}
\headerParamItem{SWARM-PIN}{bool}
\end{headerParameter}
\begin{queryParameter} 
\queryParamItem{span}{integer}
\queryParamItem{sig}{hex string  of rsv format signature}
\end{queryParameter}
\begin{requestBody}
chunk data
\end{requestBody}
\begin{routeResponse}{application/json}
\responseItem{201}{Created}{reference in response body}
\responseItem{400}{Bad Request}{owner,  id, signature, data or span is not well formed} 
\responseItem{401}{Unauthorized}{signing fails} 
\responseItem{404}{Not Found}{owner keypair is not found} 
\responseItem{413}{Payload Too Large}{payload size exceeds span value or 4096} 
\responseItem{422}{Unprocessable Entity}{owner/id pair already exists} 
\end{routeResponse}
\end{apiRoute}

\subsection{Files \statusgreen}\label{spec:api:file}

\begin{apiRoute}{POST}{/files/}{Path description as in \ref{def:swarm-hash}}
{
}
{ }

\begin{headerParameter} 
\headerParamItem{SWARM-TAG}{hex string}
\headerParamItem{SWARM-STAMP}{hex string}
\headerParamItem{SWARM-ENCRYPTION}{hex string}
\headerParamItem{SWARM-PIN}{bool}
\headerParamItem{SWARM-PARITIES}{integer}
\end{headerParameter}
\begin{requestBody}
file data
\end{requestBody}
\begin{routeResponse}{application/json}
\responseItem{201}{Created}{Minimal manifest entry as response body}


\end{routeResponse}
\end{apiRoute}






\begin{apiRoute}{PUT}{/files/\param{reference} }{Append body to file, returns new reference. Note that any intermediate chunk of a file is a file.}
{
}
{ }

\begin{routeParameter} 
\routeParamItem{reference}{hex string}
\end{routeParameter}
\begin{headerParameter} 
\headerParamItem{SWARM-TAG}{hex string}
\headerParamItem{SWARM-STAMP}{hex string}
\headerParamItem{SWARM-ENCRYPTION}{hex string}
\headerParamItem{SWARM-PIN}{bool}
\headerParamItem{SWARM-PARITIES}{integer}
\end{headerParameter}
\begin{requestBody}
file data
\end{requestBody}
\begin{routeResponse}{application/json}
\responseItem{201}{Created}{minimal manifest entry as request body}
\responseItem{400}{Bad Request}{reference is not well formed}
% \responseItem{403}{Forbidden}{encrypted content but no decryption key in  reference}
\responseItem{408}{Request Timeout}{retrieval of file to append to times out} 
% \responseItem{420}{Enhance Your Calm}{recovery initiated but request timed out}
% \responseItem{413}{Pay,load Too Large}{}
\end{routeResponse}
\end{apiRoute}



\begin{apiRoute}{GET}{/files/\param{reference}}{Retrieve file by reference as in \ref{def:file-retrieval}}
{
}
{ }

\begin{routeParameter} 
\routeParamItem{reference}{string}
\end{routeParameter}
\begin{routeResponse}{application/json}
\responseItem{200}{OK}{file contents streamed in response body}
\responseItem{400}{Bad Request}{reference is not well formed}
% \responseItem{403}{Forbidden}{encrypted content but no decryption key in reference}
\responseItem{408}{Request Timeout}{timeout retrieving file} 
\responseItem{416}{Range Not Satisfiable}{offset in range query out of range}
% \responseItem{420}{Enhance Your Calm}{recovery initiated but request timed out}
\end{routeResponse}
\end{apiRoute}


\subsection{Manifest \statusgreen}\label{spec:api:manifest}




\begin{apiRoute}{GET}{/manifest/\param{reference}/\param{path}}{Lookup entry by path in manifest, see \ref{def:manifests-lookup}}
{
}
{ }

\begin{routeParameter} 
\routeParamItem{reference}{hex string}
\routeParamItem{path}{string}
\end{routeParameter}
\begin{routeResponse}{application/json}
\responseItem{200}{OK}{manifest entry in response body}
\responseItem{400}{Bad Request}{reference or path is not well formed}
% \responseItem{403}{Forbidden}{encrypted content but no decryption key in reference}
\responseItem{404}{Not Fund}{path does not exist}
\responseItem{408}{Request Timeout}{timeout retrieving referenced manifest}
% \responseItem{420}{Enhance Your Calm}{recovery initiated but request timed out}
\end{routeResponse}
\end{apiRoute}



\begin{apiRoute}{DELETE}{/manifest/\param{reference}/\param{path} }{Delete entry on path in referenced manifest, see \ref{def:manifest-remove}}
{
}
{ }

\begin{routeParameter} 
\routeParamItem{reference}{hex string}
\routeParamItem{path}{string}
\end{routeParameter}
\begin{routeResponse}{application/json}
\responseItem{200}{OK}{reference to new manifest}
\responseItem{400}{Bad Request}{reference or path is not well formed}
% \responseItem{403}{Forbidden}{encrypted content but no decryption key in reference}
\responseItem{404}{Not Found}{path to delete does not exist}
\responseItem{408}{Request Timeout}{timeout retrieving referenced manifest}
% \responseItem{420}{Enhance Your Calm}{recovery initiated but request timed out}
\end{routeResponse}
\end{apiRoute}



\begin{apiRoute}{PUT}{/manifest/\param{reference}/\param{path}}{Update manifest as in \ref{def:manifest-update}}
{
}
{ }

\begin{routeParameter} 
\routeParamItem{reference}{hex string}
\routeParamItem{path}{(hex) string}
\end{routeParameter}
\begin{requestBody}
reference to add on path
\end{requestBody}
\begin{routeResponse}{application/json}
\responseItem{200}{OK}{reference to new manifest}
\responseItem{400}{Bad Request}{reference is not well formed}
% \responseItem{403}{Forbidden}{encrypted content but no decryption key in reference}
\responseItem{404}{Not Found}{path does not exist}
\responseItem{408}{Request Timeout}{timeout retrieving referenced manifest}
% \responseItem{420}{Enhance Your Calm}{recovery initiated but request timed out}

\end{routeResponse}
\end{apiRoute}




\begin{apiRoute}{POST}{/manifest/\param{one}/\param{other}}{Merge manifests as in \ref{def:manifest-merge}}
{
}
{ }

\begin{routeParameter} 
\routeParamItem{one}{hex string - reference to manifest to merge}
\routeParamItem{other}{hex string}
\end{routeParameter}
\begin{routeResponse}{application/json}
\responseItem{201}{Created}{reference in response body}
\responseItem{400}{Bad Request}{reference is not well formed}
% \responseItem{403}{Forbidden}{encrypted content but no decryption key in reference}
\responseItem{408}{Request Timeout}{timeout retrieving referenced manifest}
% \responseItem{420}{Enhance Your Calm}{recovery initiated but request timed out}
\end{routeResponse}
\end{apiRoute}

\subsection{High level storage API \statusyellow}\label{spec:api:high-level-storage}


\begin{apiRoute}{GET}{/bzz/\param{host}/\param{path}}{Download file}
{
}
{ }

\begin{routeParameter} 
\routeParamItem{host}{string - reference or ENS domain}
\routeParamItem{path}{string}
\end{routeParameter}
\begin{headerParameter} 
\headerParamItem{SWARM-TAG}{hex string}
\headerParamItem{SWARM-STAMP}{hex string}
\headerParamItem{SWARM-ENCRYPTION}{hex string}
\headerParamItem{SWARM-PIN}{bool}
\headerParamItem{SWARM-PARITIES}{integer}
\end{headerParameter}
\begin{routeResponse}{application/json}
\responseItem{200}{OK}{file streamed in response body}
\responseItem{400}{Bad Request}{host/reference or path is not well formed}
\responseItem{401}{Unauthorized}{access denied: AC unlock failed}
% \responseItem{403}{Forbidden}{encrypted content but no decryption key in reference}
\responseItem{404}{Not Found}{host cannot be resolved or Path does not exist}
\responseItem{408}{Request Timeout}{timeout retrieving referenced manifest}
\responseItem{416}{Range Not Satisfiable}{offset in range query out of range}
% \responseItem{420}{Enhance Your Calm}{recovery initiated but request timed out}
\end{routeResponse}
\end{apiRoute}



\begin{apiRoute}{PUT}{/bzz/\param{host}/\param{path}}{Append upload to file referenced, add new entry to path, returns new manifest}
{
}
{ }

\begin{routeParameter} 
\routeParamItem{host}{string - reference or ENS domain}
\routeParamItem{path}{string}
\end{routeParameter}
\begin{headerParameter} 
\headerParamItem{SWARM-TAG}{hex string}
\headerParamItem{SWARM-STAMP}{hex string}
\headerParamItem{SWARM-ENCRYPTION}{hex string}
\headerParamItem{SWARM-PIN}{bool}
\headerParamItem{SWARM-PARITIES}{integer}
\end{headerParameter}
\begin{requestBody}
file/collection data
\end{requestBody}
\begin{routeResponse}{application/json}
\responseItem{201}{Created}{new manifest root reference in response body}
\responseItem{400}{Bad Request}{host/reference is not well formed}
\responseItem{401}{Unauthorized}{access denied: AC unlock failed}
% \responseItem{403}{Forbidden}{encrypted content but no decryption key in reference}
\responseItem{404}{Not Found}{host cannot be resolved or path does not exist}
\responseItem{408}{Request Timeout}{timeout retrieving referenced manifest}
% \responseItem{420}{Enhance Your Calm}{recovery initiated but request timed out}
\end{routeResponse}
\end{apiRoute}




\begin{apiRoute}{POST}{/bzz/\param{host}/\param{path}}{Upload file or collection, returns new manifest root reference}
{
}
{ }


\begin{routeParameter} 
\routeParamItem{host}{string - reference or ENS domain}
\routeParamItem{path}{string}
\end{routeParameter}
\begin{headerParameter} 
\headerParamItem{SWARM-TAG}{hex string}
\headerParamItem{SWARM-STAMP}{hex string}
\headerParamItem{SWARM-ENCRYPTION}{hex string}
\headerParamItem{SWARM-PIN}{bool}
\headerParamItem{SWARM-PARITIES}{integer}
\end{headerParameter}
\begin{requestBody}
file/collection data
\end{requestBody}
\begin{routeResponse}{application/json}
\responseItem{201}{Created}{new manifest root reference in response body}
\responseItem{400}{Bad Request}{host/reference is not well formed}
\responseItem{401}{Unauthorized}{access denied: AC unlock failed}
% \responseItem{403}{Forbidden}{encrypted content but no decryption key in reference}
\responseItem{404}{Not found}{host cannot be resolved or path does not exist}
\responseItem{408}{Request Timeout}{timeout retrieving referenced manifest}
% \responseItem{420}{Enhance Your Calm}{recovery initiated but request timed out}
\end{routeResponse}
\end{apiRoute}




\subsection{Tags \statusyellow}\label{spec:api:tags}

\begin{apiRoute}{POST}{/tags}{Create new tag}
{
}
{ }

\begin{routeResponse}{application/json}
\responseItem{201}{Created}{ID in response body}
\end{routeResponse}
\end{apiRoute}




\begin{apiRoute}{GET}{/tags?offset=\param{offset}\&length=\param{length}}{Get all tags}
{
}
{ }

\begin{queryParameter} 
\queryParamItem{offset}{integer}
\queryParamItem{length}{integer}
\end{queryParameter}
\begin{routeResponse}{application/json}
\responseItem{200}{OK}{json array of tag details struct}
\end{routeResponse}
\end{apiRoute}




\begin{apiRoute}{GET}{/tags/\param{id}}{View details tag with given ID}
{
}
{ }

\begin{routeParameter} 
\routeParamItem{id}{hex string}
\end{routeParameter}
\begin{routeResponse}{application/json}
\responseItem{200}{OK}{json struct in response body}
\responseItem{400}{Bad Request}{ID not well formed}
\responseItem{404}{Not Found}{tag with ID does not exists.}
\end{routeResponse}
\end{apiRoute}



\begin{apiRoute}{DELETE}{/tags/\param{id}}{Path description}
{
}
{ }

\begin{routeParameter} 
\routeParamItem{id}{hex string}
\end{routeParameter}
\begin{routeResponse}{application/json}
\responseItem{204}{No Content}{successful deletion}
\responseItem{400}{Bad Request}{ID not well formed}
\responseItem{404}{Not Found}{tag with ID does not exists.}

\end{routeResponse}
\end{apiRoute}



\subsection{Pinning
\statusyellow}\label{spec:api:pinning}

\begin{apiRoute}{GET}{/pin/?offset=\param{offset}\&length=\param{length}}{List pinned content and metadata}
{
}
{ }

\begin{queryParameter} 
\queryParamItem{offset}{integer}
\queryParamItem{length}{integer}
\end{queryParameter}
\begin{routeResponse}{application/json}
\responseItem{200}{OK}{json array in response body}
\end{routeResponse}
\end{apiRoute}


\begin{apiRoute}{GET}{/pin/\param{id}}{View pinned content and metadata}
{
}
{ }

\begin{routeParameter} 
\routeParamItem{id}{hex string}
\end{routeParameter}
\begin{routeResponse}{application/json}
\responseItem{200}{OK}{pin in response body}
\responseItem{400}{Bad Request}{ID not well formed}
\responseItem{404}{Not Found}{Pin with ID does not exists.}
\end{routeResponse}
\end{apiRoute}




\begin{apiRoute}{PUT}{/pin/\param{id}}{Pin an already uploaded content}
{
}
{ }

\begin{routeParameter} 
\routeParamItem{id}{hex string}
\end{routeParameter}
\begin{routeResponse}{application/json}
\responseItem{200}{OK}{json struct in response body}
\responseItem{400}{Bad Request}{ID not well formed}
\responseItem{404}{Not Found}{Pin with ID does not exists.}
\end{routeResponse}
\end{apiRoute}




% \begin{apiRoute}{PUT}{/pin/\{id\}}{Path description?}
% {
% }
% { }

% \begin{routeParameter} 
% \routeParamItem{id}{}
% \end{routeParameter}
% \begin{routeResponse}{application/json}
% \begin{routeResponseItem}{200}{Ok}
% \begin{routeResponseItemBody}
  
% \end{routeResponseItemBody}
% \end{routeResponseItem}
% \end{routeResponse}
% \end{apiRoute}




\begin{apiRoute}{DELETE}{/pin/\param{id}}{Remove pinning from content}
{
}
{ }

\begin{routeParameter} 
\routeParamItem{id}{hex string}
\end{routeParameter}
\begin{routeResponse}{application/json}
\responseItem{204}{No Content}{successful deletion}
\responseItem{400}{Bad Request}{ID not well formed}
\responseItem{404}{Not found}{pin with ID does not exists}
\end{routeResponse}
\end{apiRoute}



\subsection{Swap and chequebook\statusorange}\label{spec:api:swap}
\input{specs/api/swap.tex}

\subsection{Postage stamps \statusorange}\label{spec:api:postage}

\begin{apiRoute}{GET}{/stamp?offset=\param{offset}\&length=\param{length}}{View all postage stamps}
{
}
{ }
\begin{queryParameter} 
\queryParamItem{offset}{integer}
\queryParamItem{length}{integer}
\end{queryParameter}

\begin{routeResponse}{application/json}
\responseItem{200}{OK}{}
\end{routeResponse}
\end{apiRoute}



\begin{apiRoute}{GET}{/stamp/\param{id}}{View postage stamp with id}
{
}
{ }

\begin{routeParameter} 
\routeParamItem{id}{}
\end{routeParameter}
\begin{routeResponse}{application/json}
\responseItem{200}{OK}{}
\responseItem{400}{Bad Request}{ID not well formed}
\responseItem{404}{Not Found}{stamp with ID does not exists.}
\end{routeResponse}
\end{apiRoute}




\begin{apiRoute}{PUT}{/stamp/\param{id}}{Top up postage stamp with id}
{
}
{ }

\begin{routeParameter} 
\routeParamItem{id}{}
\end{routeParameter}
\begin{queryParameter} 
\queryParamItem{amount}{integer}
\end{queryParameter}

\begin{routeResponse}{application/json}
\responseItem{200}{ok}{}
\responseItem{400}{Bad Request}{ID not well formed}
\responseItem{404}{Not Found}{stamp with ID does not exists.}
\end{routeResponse}
\end{apiRoute}




\begin{apiRoute}{DELETE}{/stamp/\param{id}}{Drain and expire stamp with id}
{
}
{ }

\begin{routeParameter} 
\routeParamItem{id}{}
\end{routeParameter}
\begin{routeResponse}{application/json}
\responseItem{200}{OK}{}
\responseItem{400}{Bad Request}{ID not well formed}
\responseItem{404}{Not Found}{stamp with ID does not exists}
\end{routeResponse}
\end{apiRoute}


\begin{apiRoute}{POST}{/stamp/}{Create a new postage stamp, return ID in response body}
{
}
{ }

\begin{routeParameter} 
\end{routeParameter}
\begin{routeResponse}{application/json}
\responseItem{201}{Created}{ID in response body}
\end{routeResponse}
\end{apiRoute}



\subsection{Access Control  \statusgreen}\label{spec:api:access-control}

\begin{apiRoute}{POST}{/access/\param{address} }{Lock ACT for address as in \ref{def:ac-api}}
{
}
{ }

\begin{routeParameter} 
\routeParamItem{address}{hex string}
\end{routeParameter}
\begin{routeResponse}{application/json}
\responseItem{201}{Created}{root access manifest reference in response body}
\responseItem{400}{Bad Request}{encrypted content but no decryption key in reference}
\responseItem{403}{Forbidden}{encrypted content but no decryption key in reference}
\responseItem{404}{Not Found}{}
\responseItem{408}{Request Timeout}{timeout retrieving referenced manifest}
\responseItem{420}{Enhance your calm}{recovery initiated but request timed out}
\end{routeResponse}
\end{apiRoute}



\begin{apiRoute}{GET}{/access/\param{address} }{Unlock ACT for address as in \ref{def:ac-api}}
{
}
{ }

\begin{routeParameter} 
\routeParamItem{address}{hex string}
\end{routeParameter}
\begin{routeResponse}{application/json}
\responseItem{200}{OK}{}
\responseItem{400}{Bad Request}{address not well formed}
\responseItem{401}{Unauthorized}{access denied: AC unlock failed}
\responseItem{403}{Forbidden}{encrypted content but no decryption key in reference}
\responseItem{408}{Request Timeout}{timeout retrieving referenced manifest}
\responseItem{420}{Enhance your calm}{recovery initiated but request timed out}
\end{routeResponse}
\end{apiRoute}




\begin{apiRoute}{PUT}{/access/\param{root}/\param{pubkey}}{Add entry for pubkey to the  ACT referred in the root access manifest \ref{def:act-api}}
{
}
{ }

\begin{routeParameter} 
\routeParamItem{root}{hex string - reference to root access manifest}
\routeParamItem{pubkey}{hex string - public key of grantee}
\end{routeParameter}
\begin{routeResponse}{application/json}
\responseItem{201}{Created}{reference to new manifest root in response body}
\responseItem{400}{Bad Request}{address or public key not well formed}
\responseItem{401}{Unauthorized}{access denied: creating session key failed}
\responseItem{403}{Forbidden}{encrypted content but no decryption key in reference}
\responseItem{408}{Request Timeout}{timeout retrieving referenced manifest}
\responseItem{420}{Enhance your calm}{recovery initiated but request timed out}
\end{routeResponse}
\end{apiRoute}


\begin{apiRoute}{DELETE}{/access/\param{root}/\param{pubkey}}{Remove entry for pubkey from ACT referred in the root access manifest, see \ref{def:act-api}}
{
}
{ }

\begin{routeParameter} 
\routeParamItem{root}{hex string - reference to root access manifest}
\routeParamItem{pubkey}{hex string - public key of grantee}
\end{routeParameter}
\begin{routeResponse}{application/json}
\responseItem{201}{Created}{reference to new manifest root in response body}
\responseItem{400}{Bad Request}{address or public key not well formed}
\responseItem{401}{Unauthorized}{access denied: creating session key failed}
\responseItem{403}{Forbidden}{encrypted content but no decryption key in reference}
\responseItem{408}{Request Timeout}{timeout retrieving referenced manifest}
\responseItem{420}{Enhance your calm}{recovery initiated but request timed out}
\end{routeResponse}
\end{apiRoute}



% \subsection{Recovery\statusorange}\label{spec:api:recovery}

\section{Communications  \statusorange}\label{spec:api:communications}
\input{specs/api/comms.tex}

\subsection{PSS \statusyellow}\label{spec:api:trojan}

\begin{apiRoute}{POST}{/pss/send/\param{topic}(?targets=\param{targets}\&recipient=\param{recipient})}{Send private message with topic to targets, encrypted for recipient if public key given. If public key parameter missing, the message is encrypted using the topic as as the key for the asymmetric encryption so that whoever knows and expects messages on the topic can read,  see \ref{def:send}}. 
{
}
{ }

\begin{routeParameter} 
\routeParamItem{topic}{string}
\end{routeParameter}
\begin{queryParameter} 
\queryParamItem{recipient}{hex string - recipient public key for encryption}
\queryParamItem{targets}{hex string - comma separated list of targets, these correspond to alternative target overlays for the same recipient and the first mined trojen chunk  matching any  target will be sent.} 
\end{queryParameter} 
\begin{headerParameter} 
\headerParamItem{SWARM-TAG}{hex string, to monitor delivery status}
\headerParamItem{SWARM-STAMP}{hex string}
\end{headerParameter}
\begin{requestBody}
the message payload plaintext
\end{requestBody}
\begin{routeResponse}{application/json}
\responseItem{209}{Sent}{Tag ID to monitor delivery with}
\responseItem{400}{Bad Request}{Topic, targets or recipient not well formed.}
\end{routeResponse}
\end{apiRoute}




\begin{apiRoute}{POST}{/pss/subscribe/\param{topic}/(?on=\param{channel})}{Subscribe to messages with topic to be delivered on given channel, see \ref{def:receive}}
{
}
{ }

\begin{routeParameter} 
\routeParamItem{topic}{string}
\end{routeParameter}
\begin{queryParameter} 
\queryParamItem{on}{hex string - channel ID}
\end{queryParameter} \begin{routeResponse}{application/json}
\responseItem{201}{Created}{}
\responseItem{400}{Bad Request}{topic or channel not well formed.}
\end{routeResponse}
\end{apiRoute}

 
\begin{apiRoute}{DELETE}{/pss/subscribe/\param{topic}/(?on=\param{channel})}{Unsubscribe for topic on channel, see \ref{def:receive}}
{
}
{ }

\begin{routeParameter} 
\routeParamItem{topic}{string}
\end{routeParameter}
\begin{queryParameter} 
\queryParamItem{on}{hex string - channel ID}
\end{queryParameter} \begin{headerParameter} 
\headerParamItem{SWARM-TAG}{hex string}
\headerParamItem{SWARM-STAMP}{hex string}
\end{headerParameter}
\begin{routeResponse}{application/json}
\responseItem{204}{No Content}{successful unsubscribe}
\responseItem{400}{Bad Request}{topic or channel not well formed.}
\end{routeResponse}
\end{apiRoute}




\subsection{Feeds \statusorange}\label{spec:api:feeds}



% \subsection{The pss URL scheme}

