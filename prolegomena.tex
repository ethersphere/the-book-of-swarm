\chapter{Prolegomena \statusgreen}
\green{}
\section*{Intended audience \statusgreen}
The primary aim of this book is to capture the wealth of output of the first phase of the Swarm project and serve as a compendium for teams and individuals participating in bringing Swarm to life in the forthcoming stages.

The book is intended for technically inclined readers who are interested in using Swarm in their development stack and wish to better understand the motivation and design decisions behind the technology. Researchers, academics and decentralisation experts are invited to check our reasoning and audit the sanity of our design. Core developers or developers from the wider eco-system who build components, tooling or entire client implementations should benefit from the specifications as well as the explanations.

\section*{Structure of the book \statusgreen}

The book has three major parts. The first, Prelude (\ref{part:preface}), explains the motivation by describing the historical context and sets the stage for a fair data economy. It then presents the Swarm vision.

The second, Design and Architecture (\ref{part:designarchitecture}), contains a detailed exposition of the design and architecture of Swarm. This part aspires to cover all areas relevant to the core functionality of Swarm.

The third part, Specifications (\ref{part:specifications}), provides the formal specification of components, which is meant to serve as the handbook for Swarm client developers.

The book is complete with index, glossary of terms and acronyms, and an appendix (\ref{part:appendix}) containing formalised arguments (\ref{sec:formalisation}), an implementor's guide (\ref{sec:implementor-guide}) and a section about the project itself, it's history, as well as the ecosystem emerging around it (\ref{sec:about}).

\section*{How to use the book \statusgreen}

The Prelude together with the Design and Architecture part can be read as one continuous narrative. Those wishing to jump right into the technology can start with the Design and Architecture part, skipping the prelude.

A Swarm client developer can start from any particular component spec in the Specifications part and work one's way back to Design and Architecture via the in-text references if wider context is needed or one is interested in the justification of some of the choices apparent in the specs.

