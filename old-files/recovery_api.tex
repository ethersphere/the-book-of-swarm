
The \texttt{recovery} API supports the creation of recovery chunks (see \ref{sec:reupload}). For this a POST request needs to be sent with a JSON structure in the request body. The request URL specifies a hash referencing a pinned file or collection while the recovery JSON in the body supports the following attributes:

\begin{itemize}
\item \texttt{publisher} - The publisher's public key used as the single  owner of the recovery chunk address.
\item \texttt{hosts} - The root hash of the manifest that collects the recovery hosts (aka pinners) overlay addresses to match recovery addresses against. The addresses are partial, i.e., the prefix of the host overlay with byte length corresponding to $\mathtt{depth} / 8+1$.  
\item \texttt{depth} - The minimum length in bits of the prefix shared by the recovery address and a recovery host overlay. It expresses the required proximity needed to land the address in a host's neighbourhood. 
\item \texttt{count} - The number of recovery chunks created for each original chunk. 
\end{itemize}

As the file/collection is traversed, for each chunk, recovery chunks are generated using the publisher's public key and an incremented index. The \texttt{hosts} attribute is set to a hash that references the manifest which represents the recovery hosts (aka pinners) addresses. The first few recovery addresses that fall within the proximity of any pinner in the manifest are calculated; the number of recovery chunks is given by the \texttt{count} attribute, and the minimal length of prefix shared by the recovery address and a pinner overlay is given by the \texttt{depth} attribute.