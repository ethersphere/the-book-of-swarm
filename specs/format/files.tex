
This table gives an overview of data sizes a chunk span represents, depending on the level of recursion.

\begin{table}[h]
\begin{tabular}{|l|l|l|l|}
\hline
Level & Span in chunks & Span in bytes & $--human$ \\
\hline
0 (data) & 1 & 4,096 & (4KB) \\
1 & 128 & 524,288 & (500KB) \\
2 & 16,384 & 67,108,864 & (67MB) \\
3 & 2,097,152 & 8,589,934,592 & (8.5GB) \\
4 & 268,435,456 & 1,099,511,627,776 & (1.1TB) \\
5 & 34,359,738,368 & 140,737,488,355,328 & (140TB) \\
6 & 4,398,046,511,104 & 18,014,398,509,481,984 & (18PB) \\
7 & 562,949,953,421,312 & 2,305,843,009,213,693,952 & (2.3EB) \\
\hline
\end{tabular}
\caption{Some Chunk Span data size boundaries}
\end{table}

\subsubsection{Calculating the Swarm Hash}

To calculate the Swarm Hash the following method is used:

\begin{enumerate}
	\item Determine the \texttt{Chunk Reference} of each \texttt{Chunk} into an \emph{intermediate result} \footnote{Remember, use the actual size of represented data}
	\item If the $log_{128}$ of the number of \texttt{Chunks} minus one is an even number, the last \texttt{Chunk Reference} should be stored and not included in the following step
	\item Otherwise, if the $log_{128}$ of the number of \texttt{Chunks} is an even number, remove the stored \texttt{Chunk Reference} and concatenate it to the \emph{intermediate result}
	\item If the data size of the \texttt{Chunk References} of the \texttt{Chunks} (including the stored \texttt{Chunk Reference} is greater than \texttt{Segment Size}, group all the \texttt{Chunk References} in the \emph{intermediate result} and repeat from step 1 with this as the data
\end{enumerate}

Note that the \texttt{Swarm Hash} of data up to a single chunk in total size is identical to its \texttt{BMT Hash}.

\subsection{Chunk Reference}

The Chunk storage in Swarm is \emph{content addressed}. This means that the identifier of each individual \texttt{Chunk} in the store is a function of the content itself.

This identifier always contains the address of the \texttt{Chunk}. This address is a \texttt{Swarm Overlay Address}, which means that \emph{nodes} and \emph{content} have comparable identifiers. \ref{sec:distance-and-proximity}.

Retrieval and interpretation of a \texttt{Chunk} from the store \emph{may} require more information than merely its identifier in the store. The entity encaspulating this information is called the \texttt{Chunk Reference}.

In its simplest form, the \texttt{Chunk Reference} is the same as the address of the \texttt{Chunk}. 

\subsection{Encrypted Chunk References}

Currently, the only other defined \texttt{Chunk Reference} is used with an intra-node convenience crypto scheme.

Here all references are double the \texttt{Segment Size}. The first \texttt{Segment Size} bytes is a key to decrypt the latter \texttt{Segment Size} bytes into an actual valid address that can be retrieved from the store.

The symmetric key provided by the caller when adding the content is used as the seed to derive all decryption keys for the \texttt{Intermediate Chunks} \ref{intermediate-chunks}. It is also used as the encryption key for the \texttt{Swarm Hash} \ref{swarm-hash}.

\todo{Describe the crypto algo}