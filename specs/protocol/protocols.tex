
\subsection{Underlay network \statusgreen}

The \lstinline{libp2p} networking stack provides all the required properties for the Swarm underlay network, as laid out in \ref{sec:underlay-transport}.

\begin{enumerate}
\item Addressing is facilitated through the use of \emph{multi addresses}, known as underlay addresses, assigned to each node. Every node can have multiple underlay addresses depending on the configured network listening addresses and transports.
\item Dialing is provided over \lstinline{libp2p} supported network transports.
\item Listening is provided by \lstinline{libp2p} supported network transports.
\item Live connections are established and maintained between two peers, ensuring the ability to send and receive messages.
\item Channel security is ensured through \lstinline{TLS} and the \lstinline{libp2p secio} stream security transport.
\item Protocol multiplexing is provided by the \lstinline{libp2p mplex} stream multiplexer protocol.
\item Delivery guarantees are provided by using \lstinline{libp2p} bidirectional streams to validate the responses from peers upon message transmission.
\item Serialization is not enforced by \lstinline{libp2p}, as allows for byte streams, providing flexibility for each protocol to choose the most appropriate serialization method. The recommended serialization approach is using \lstinline{Protobuf} with \lstinline{varint} delimited messages in streams.
\end{enumerate}

\subsection{Protocols and streams \statusgreen}

Communication between peers is organised in protocols as logical units under a unique name that may define one or more \emph{streams}. \lstinline{libp2p} provides streams as the basic channel for communication. Streams are full-duplex channels of bytes, multiplexed over a singe connection between two peers.

Every stream defines:

\begin{itemize}
\item a version that follows semantic versioning in semver form
\item data serialization definitions
\item sequence of data passing between peers over a full-duplex stream
\end{itemize}

Streams are identified by \lstinline{libp2p} case-sensitive protocol IDs. The following convention is used construct stream identifiers:

\begin{lstlisting}
/swarm/ProtocolName/ProtocolVersion/StreamName
\end{lstlisting}

\begin{itemize}
\item All stream IDs are prefixed with \lstinline{/swarm}.
\item \lstinline{ProtocolName} is an arbitrary string that identifies the protocol.
\item \lstinline{ProtocolVersion} is a string in semver form that is used to specify compatibility between protocol implementations over time.
\item \lstinline{StreamName} is an arbitrary string that identifies a stream defined as part of the protocol.
\end{itemize}

\subsection{Data exchange sequences \statusgreen}

A data passing sequence must be synchronous under one opened stream. Multiple streams can be opened at the same time that are multiplexed over the same connection exchanging data independently and asynchronously. Streams may use different data exchange sequences such as:

\begin{itemize}
\item \emph{single message sending} - not waiting for the response by the peer if it is not needed before closing the stream.
\item \emph{multiple message sending} - a series of data that is sent to a peer without reading from it before closing the stream.
\item \emph{request/response} - requires a single response for a single request before closing the stream.
\item \emph{multiple requests/response cycles} - require a synchronous response after every request before closing the stream.
\item \emph{exact message sequence} -  requires multiple message types over a single stream in an exact order (see the handshake protocol in \ref{spec:protocol:hive}).
\end{itemize}

Streams have predefined sequences that are kept as simple as possible for a single purpose. For complex message exchanges, multiple streams should be used.

Streams may be short lived for immediate data exchange or communication, or long lived for notifications if needed.

\subsection{Stream headers}

A Swarm specific requirement for all libp2p streams is to exchange Header protobuf messages on every stream initialization between two peers. This message encapsulates a stream scoped information that needs to be exchanged before any stream specific data or messages are exchanged. Headers are sequences of key value pairs, where keys are arbitrary strings and values are byte arrays that do not impose any specific encoding. Every key may use appropriate encoding for the data that it relates to.

\begin{definition}[Header message]\label{def:headers-message}

\begin{lstlisting}[language=protobuf3]
syntax = "proto3";

package pb;

message Headers {
    repeated Header headers = 1;
}

message Header {
  string key = 1;
  bytes value = 2;
}
\end{lstlisting}
\end{definition}

On every stream initialization, the peer that creates it, is sending Headers message regardless if it contains header values or not. The receiving node must read this message and respond with response header using the same message type. This makes the header exchange sequence finished and any other stream data can be transmitted depending on the protocol.

Standard header key names are defined here:

\begin{enumerate}
\item tracing-span-context
\end{enumerate}


\subsection{Encapsulation of context for tracing \statusgreen}

P2P Stream scoped tracing span context is exchanged by using stream headers. Header key "tracing-span-context" is reserved for binary encoded tracing span context data. This context should be used in tracing messages. The stream initiator node should provide tracing span context to the responding node. This context is optional and all nodes must function the same as span context is provided by other nodes or not regardless if the node has tracing configured or not.
