
\subsection{Underlay network}

The \texttt{libp2p} networking stack provides all required properties for the swarm underlay network laid out in \ref{sec:underlay-transport}.

\begin{enumerate}
\item Addressing is provided in the form of so called \emph{multi address} for every node, which is referred here as the underlay address. Every node can have multiple underlay addresses depending on transports and network listening addresses that are configured.
\item Dialing is provided over \texttt{libp2p} supported network transports.
\item Listening is provided by \texttt{libp2p} supported network transports.
\item Live connections are established between two peers and kept open for accepting or sending messages.
\item Channel security is provided with \texttt{TLS} and \texttt{libp2p secio} stream security transport.
\item Protocol multiplexing is provided by \texttt{libp2p mplex} stream multiplexer protocol.
\item Delivery guarantees are provided by using \texttt{libp2p} bidirectional streams to validate the response from the peer on sent message.
\item Serialization is not enforced by \texttt{libp2p}, as it provides byte streams allowing flexibility for every protocol to choose the most appropriate serialization. The recommended serialization is \texttt{Protobuf} with \texttt{varint} delimited messages in streams.
\end{enumerate}

\subsection{Protocols and streams}

Communication between peers is organised in protocols as logical units under a unique name that may define one or more \emph{streams}. \texttt{libp2p} provides streams as the basic channel for communication. Streams are full-duplex channels of bytes, multiplexed over a singe connection between two peers.

Every stream defines:

\begin{itemize}
\item a version that follows semantic versioning in semver form
\item data serialization definitions
\item sequence of data passing between peers over a full-duplex stream
\end{itemize}

Streams are identified by \texttt{libp2p} case-sensitive protocol IDs. The following convention is used construct stream identifiers:

\begin{lstlisting}[basicstyle=\ttfamily]
/swarm/ProtocolName/ProtocolVersion/StreamName
\end{lstlisting}

\begin{itemize}
\item All stream IDs are prefixed with \texttt{/swarm}.
\item \texttt{ProtocolName} is an arbitrary string that identifies the protocol.
\item \texttt{ProtocolVersion} is a string in semver form that is used to specify compatibility between protocol implementations over time.
\item \texttt{StreamName} is an arbitrary string that identifies a stream defined as part of the protocol.
\end{itemize}

\subsection{Data exchange sequences}

A data passing sequence must be synchronous under one opened stream. Multiple streams can be opened at the same time that are multiplexed over the same connection exchanging data independently and asynchronously. Streams may use different data exchange sequences such as:

\begin{itemize}
\item \emph{single message sending} - not waiting for the response by the peer if it is not needed before closing the stream.
\item \emph{multiple message sending} - a series of data that is sent to a peer without reading from it before closing the stream.
\item \emph{request/response} - requires a single response for a single request before closing the stream.
\item \emph{multiple requests/response cycles} - require a synchronous response after every request before closing the stream.
\item \emph{exact message sequence} -  requires multiple message types over a single stream in an exact order (see the handshake protocol in \ref{spec:protocol:hive}).
\end{itemize}

Streams have predefined sequences that are kept as simple as possible for a single purpose. For complex message exchanges, multiple streams should be used.

Streams may be short lived for immediate data exchange or communication, or long lived for notifications if needed.
