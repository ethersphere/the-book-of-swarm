Forwarding strategy refers to the process a node follows pertaining to relayed messages and responses in Forwarding Kademlia as relevant for chunk retrieval and upload.

Note that for retrieval as well as push-syncing, the protocol does not specify just allows forwarding of retrieve requests and chunk deliveries, respectively.
Whether a node forwards an incoming message and where it forwards them, should be guided by the incentives in order to attain stability.

As described in \ref{sec:pricing}, nodes  formulate and  advertise their prices for each PO bin separately. 
The optimal pricing strategy will reflect the PO bins in that the prices for closer bins will monotonically decrease. 

With sufficiently 
    
 
\subsubsection{Retries}
%disconnection

Both push-sync and retrieval messages are using backwarding (i.e., pass-back responses), and the compensation for forwarding only gets accounted for once the response is sent. This incentivises nodes to watch on peer connections where the downstream message was sent. In particular, if downstream peer disconnects  before the response in received, the forwarding should be repeated and tried on another peer.


%price-driven strategy

% acyclicity


 %\subsubsection{}

% multiple requests


