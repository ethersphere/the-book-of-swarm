
## Abstract

Depending on the capability settings at any given moment, each module in Swarm may have different connectivity requirements.

Since the actual interpretation of capabilities is confined to each module and opaque between modules, the connectivity driver cannot and should not be able to decide which type of connectivity particular capability settings require.

This proposal adds a method for modules to signal what connectivity they require to operate.


## Motivation

To make it possible to determine and maintain a minimum of required connectivity.

This proposal does not change the Swarm protocol.


## Specification

*TODO: Should refer to SWIP as part of node implemnter specification for connectivity cases when it is available*

Connectivity configurations for Swarm can be broken down in three concrete cases:

* **Dispatcher** - `n` peers in a far bin providing broad access
  - Send message
  - Retrieve chunk
* **Receiver** - `n` peers in neighborhood providing routing endpoint
  - Receive message
* **Saturation** - `n` peers in every bin
  - Pull syncing

In cases where the node is a consumer node and/or resources are severely constrained, it should be possible to provide an adaptive mode of operation that overrides the default peer cardinality requirements. This can be achieved by adding a flag to request a "minimal" configuration.

Thus, all relevant connectivity cases can be summarized in a table, where:

* `f` is the saturation threshold of the far bin
* `n` is the neighborhood threshold
* `i` is the saturation threshold of intermediate bins.

|far bit|nbh bit|min bit| | far peers | nbh peers | bin peers | 
|---|---|---|---|---|---|---|
| X |   |   | | `f` | 0 | 0 |
| X |   | X | |  `1` | 0 | 0 |
|   | X |   | |  0 | `n` | 0 | 
|   | X | X | |  0 | `1` | 0 | 
| X | X |   | |  `f` | `n` | `b` |
| X | X | X | | `1` | `1` | 0 |


## Test Cases

* Connectivity correctness test for all cases and cardinalities in **Specification** above.
* `Capabilities` holds the minimal sum of 3 different capabilities representing all cases


## Implementation

The three bits outlined in the **Specification** can be introduced to each individual `Capability` in the `Capabilities` array. They can be encapsulated as a new object tentatively named `CapabilityConnectivity`:

```
CapabilityConnectivity {
	far	bool
	near	bool
	min	bool
}
```

The Capability struct then becomes:

```
Capability {
	Id	uint64
	Cap	[]bool
	connect CapabilityConnectivity
}
```

Since individual `Capability` settings are registered and modified through the encapsulating `Capabilities` object, it is possble for this component to digest the at any time required minimum of connectivity (the `OR` of all `CapabilityConnectivity` settings in all `Capability` objects). 

The result can be represented by the same struct. `Capabilities` then become:

```
Capabilites {
	Caps	[]Capability
	connect CapabilityConnectivity
}
```

### Serialization

A node can use the `depth` parameter of the `discovery` protocol to control the radius of peers suggested to it. Therefore, this connectivity request information can remain node-internal and does not need to be transmitted to peers.

## Simple Summary
Peer suggestion inside a bin is been done randomly (or the first in line). Instead of doing that, it is better to suggest
a peer that covers a bigger address gap space.

## Abstract
This PR propose an improved method for suggesting peers for connecting. Whenever a peer of a given bin needs to be added,
instead of taking the first `callable`, the method suggest the peer which covers a bigger address space gap. 
An address space gap is defined as a set of addresses in which the current node doesn't have a connected peer.

## Motivation
Improve the suggested peer algorithm, trying to cover a bigger part of the address space and allowing the node to be 
more efficient when searching for the nearest peer to a certain address.

## Specification
<!--The technical specification should describe the syntax and semantics of any new feature. The specification should be detailed enough to allow competing, interoperable implementations for the current Swarm platform and future client implementations.-->
In order to optimize Kademlia load balancing, performance and peer suggestion, we define the concept of `address space gap`
or simply `gap`. 
A `gap` is a portion of the overlay address space in which the current node does not know any peer. It could be represented
as a range of addresses: `0xxx`, meaning `0000-0111`

The `proximity order of a gap` or `gap po` is the proximity order of that address space with respect to the nearest peer(s)
in the kademlia connected table (and considering also the current node address). For example if the node address is `0000`, 
the gap of addresses `1xxx` has proximity order 0. However the proximity order of the gap `01xx` has po 1.
  
The `size of a gap` is defined as the number of addresses that could fit in it. If the area of the whole address space is 1,
the `size of a gap` could be defined from the `gap po` as `1 / 2 ^ (po + 1)`. For example, our previous `1xxx` gap has a size of
`1 / (2 ^ 1) = 1/2`. The size of `01xx` is `1 / (2 ^ 2) = 1/4`. 

## Rationale
In order to increment performance of content retrieval and delivery the node should minimize the size of its gaps, because this 
means that it knows peers near almost all addresses. If the minimum `gap` in the kademlia table is 4, it means that whatever 
look up or forwarding done will be at least 4 po far away. On the other hand, if the node has a 0 po `gap`, it means that
for half the addresses, the next jump will be still 0 po away!.


### Gaps for peer suggestion
The current kademlia bootstrap algorithm tries to fill in the bins (or po spaces) until some level of saturation is reached.
In the process of doing that, the `gaps` will diminish, but not in the optimal way.

For example, if the node address is `00000000`, it is connected only with one peer in bin 0 `10000000` and the known 
addresses for bin 0 are: `10000001` and `11000000`. The current algorithm will take the first `callable` one, so 
for example, it may suggest `10000001` as next peer. This is not optimal, as the biggest `gap` in bin 0 will still be 
po 1 => `11xxxxxx`. If however, the algorithm is improved searching for a peer which covers a bigger `gap`, `11000000` would
be selected and now the biggest `gaps` will be po2 => `111xxxx` and `101xxxx`.

Additionally, even though the node does not have an address in a particular `gap`, it could still select the furthest away 
from the current peers so it covers a bigger `gap`. In the previous example with node `00000000` and one peer already connected
`10000000`, if the known addresses are `10000001` and `1001000`, the best suggestion would be the last one, because it is po 3
from the nearest peer as opposed to `10000001` that is only po 7 away. The best case will cover a `gap` of po 3 size 
(1/16 of area or 16 addresses) and the other one just po 7 size (1/256 area or 1 address).

### Gaps and load balancing
One additional benefit in considering `gaps` is load balancing. If the target addresses are distributed randomly 
(although address popularity is another problem that can also be studied from the `gap` perspective), the request will
be automatically load balanced if we try to connect to peers covering the bigger `gaps`. Continuing with our example,
if in bin 0 we have peers `10000000` and `10000001` (Fig. 1), almost all addresses in space `1xxxxxxx`, that is, half of the 
addresses will have the same distance from both peers. If we need to send to some of those address we will need to use
one of those peers. This could be done randomly, always the first or with some load balancing accounting to use the least
used one. 
![Fig. 1](./../assets/swip-suggest-by-address-gap/address_space_gaps-lb-1.png)
Fig.1 - Closer peers needs an external Load Balancing mechanism

This last method will still be useful, but if the `gap` filling strategy is used, most probably both peers will
be separated enough that they never compete for an address and a natural load balancing will be made among them (for example,
`10000000` and `11000000` will be used each for half the addresses in bin 0 (Fig. 2)).
![Fig. 2](./../assets/swip-suggest-by-address-gap/address_space_gaps-lb-2.png)
Fig.2 - Peers chosen by space address gap have a natural load balancing

### Further improvements
 Instead of the size of a gap, maybe it could be more interesting to see the ratio between size and number of current 
 peers serving that gap. If we have `n` current peers that are equidistant to a particular gap of size `s`,
the load of each of these peers will be on average `s/n`. 
We can define a gap's `temperature` as that number `s/n`. When looking for new peers to connect, instead of looking for
bigger gaps we could look for `hotter` gaps.
For example, if in our first example, we can't find a peer in `11xxxxxx` and we instead, used the best peer, we could end
with the configuration in Fig. 3.
![Fig. 3](./../assets/swip-suggest-by-address-gap/address_space_gaps-lb-3.png)
Fig. 3 - Comparing gaps temperature

Here we still have `11xxxxxx` as the biggest gap (po=1, size 1/4), same size as `01xxxxxx`. But if we consider temperature,
`01xxxxxx` is hotter because is served only by our node `00000000`, being its temperature  `(1/4)/ 1 = 1/4`. However,
`11xxxxxx` is now served by two peers, so its temperature is `(1/4) / 2 = 1/8`, and that will mean that we will select
`01xxxxxx` as the hotter one.

There is a way of implementing temperature calculation so its cost  is the same as looking for biggest gap. Temperature
can be calculated on the fly as the gap is found using a `pot`.

Other metrics could be considered in the temperature, as recently number of requests per address space, performance of
current peers...

## Backwards Compatibility
This will be totally backwards compatible with the current implementation as it only improves the suggestion of the 
next peer to connect to.

## Implementation
A preliminary implementation is done in [PR #1869](https://github.com/ethersphere/swarm/pull/1869) 
The search for gaps can be done easily using a proximity order tree or `pot`. Traversing the bins of a node, a `gap` is
found if there is some of the po's missing starting from furthest (left). In each level the starting po to search for is the
parent po (not 0, because in the second level, under a node of po=0, the minimum po that could be found is 1).

Implementation of the function that looks for the bigger Gap in a `pot` can be seen in
`pot.BiggestAddressGap`. That function returns the biggest gap in the form of a po and
a node under the gap can be found.

This function is used in `kademlia.suggestPeerInBinByGap`, which returns a BzzAddress in a particular bin which fills
 up the biggest address gap. This function is not used in `SuggestPeer`, but it will be enough to replace the call to 
 `suggestPeerInBin` with the new one.

The method is implemented but is not currently attached to the main code (so it is only used in unit tests). 
Attaching it is just replacing the call to the current function. 
