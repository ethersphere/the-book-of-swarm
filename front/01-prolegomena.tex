\chapter{Prolegomena \statusgreen}
\green{}
\section*{Intended audience \statusgreen}
The primary aim of this book is to capture the wealth of output of the first phase of the Swarm project. It serves as a compendium for teams and individuals participating in bringing Swarm to life in the forthcoming stages.

The book is intended for technically inclined readers who are interested in incorporating Swarm into their development stack and understanding the motivation and design choices behind the technology. Moreover, we extend an invitation to researchers, academics, and decentralisation experts to review our reasoning and audit the coherence of Swarm's overall design. For core developers and those contributing to the wider ecosystem by building components, tooling, or client implementations, this book offers concrete specifications and insights into the thought process behind them.

\section*{Structure of the book \statusgreen}

The book comprises two major parts. The Prelude (\ref{part:preface}) delves into the motivation behind the Swarm project by describing the historical context and laying the foundation for a fair data economy. We then present the Swarm vision.

The second part, Design and Architecture (\ref{part:designarchitecture}), offers a comprehensive and in-depth exploration of the design and architecture of the swarm. This part covers all areas relevant to Swarm's core functionality.

To complement the main content, the book includes valuable resources such as an index, a glossary defining technical terms and acronyms, and an appendix, providing readers with a well-rounded and complete compendium.

\section*{How to use the book \statusgreen}

The Prelude and Design and Architecture sections have been seamlessly combined to form a cohesive and continuous narrative. For those wishing to jump right into the technology, they can start with the Design and Architecture part, skipping the Prelude.

On the other hand, Swarm client developers can use the book as background reading to gain a comprehensive understanding of the specifications. This approach proves valuable when seeking a wider context or when interested in the justification for the choices made in the development process.
