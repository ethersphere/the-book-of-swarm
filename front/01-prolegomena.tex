\chapter{Prolegomena \statusgreen}
\green{}
\section*{Intended audience \statusgreen}
The primary aim of this book is to capture the wealth of output of the first phase of the Swarm project, and to serve as a compendium for teams and individuals participating in bringing Swarm to life in the forthcoming stages.

The book is intended for the technically inclined reader who is interested in using Swarm in their development stack and wishes to better understand the motivation and design decisions behind the technology. Researchers, academics and decentralisation experts are invited to check our reasoning and audit the consistency of Swarm's overall design. Core developers and developers from the wider ecosystem who build components, tooling or client implementations, should benefit from the concrete specifications we present, as well as from the explanation of the thoughts behind them.

\section*{Structure of the book \statusgreen}

The book has two major parts. The Prelude (\ref{part:preface}), explains the motivation by describing the historical context, setting the stage for a fair data economy. We then present the Swarm vision.

The second part, Design and Architecture (\ref{part:designarchitecture}), contains a detailed exposition of the design and architecture of Swarm. This part covers all areas relevant to Swarm's core functionality.

The index, glossary for technical terms and acronyms, and an appendix complete the compendium.

\section*{How to use the book \statusgreen}

The first two parts – the Prelude and  Design and Architecture – can be read as one continuous narrative. Those wishing to jump right into the technology can start with the Design and Architecture part, skipping the Prelude.

A Swarm client developer can use the book as background reading for the specs whenever wider context is needed or one is interested in the justification for the choices.
